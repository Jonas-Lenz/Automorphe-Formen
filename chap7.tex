\chapter{Fourieranalysis}
Sei $G$ eine abelsche, lokal kompakte Hausdorffgruppe, $\mathrm{d}x$ ein Haarmaß auf $G$
sowie $\hat{G}$ die duale Gruppe.
\begin{defi}
Die \emph{Fouriertransformierte} von $f \in \mathit{L}^1(G)$ ist definiert durch
\begin{align*}
\hat{f}\colon \hat{G} &\to \C\\
\chi &\mapsto \hat{f}(\chi)=\int_G f(x)\overline{\chi(x)}~\mathrm{d}x.
\end{align*}
\end{defi}

\begin{bsp}
Für $G=(\R,+)$ erhalten wir die bekannte Fouriertransformation aus der Integrationstheorie.
Wir wissen bereits, dass $\hat{G} \cong G$ und somit können wir die Fouriertransformation als Funktion $\R \to \C$ auffassen.
Wir erhalten so die klassische Formel für die Fouriertransformation.
\begin{align*}
\hat{f}(y)=\int_\R f(x)\overline{\mathrm{e}_{\infty}(yx)}~\mathrm{d}x.
\end{align*}
\end{bsp}

Im Folgenden werden wir Schwartzfunktionen betrachten.
\begin{defi}
Eine Funktion $f\colon \R \to \C$ heißt \emph{Schwartzfunktion}, wenn $f \in \CC^\infty$ und $Pf^{(n)}$ für alle Polynome $P$ und alle $n \in \N$ beschränkt ist.
\end{defi}
\begin{bsp}
Die Funktion $f(x)=\mathrm{e}^{-x^2}$ ist eine Schwartzfunktion, außerdem alle $\CC^\infty$ Funktionen, die kompakten Träger haben.
\end{bsp}
Wir erweitern diese Definition auf $\Q_p$.
\begin{defi}
Sei $p<\infty$. Eine Funktion $f \colon \Q_p \to \C$ heißt \emph{Schwartzfunktion}, falls sie lokal konstant ist und kompakten Träger hat.
Wir definieren $\Sq$ als die Menge aller Schwartzfunktionen auf $\Q_p$.
Dies ist ein komplexer Vektorraum.
\end{defi}
\begin{bsp}
Für $a \in \Q_p$ und $m\in \Z$ ist $\chi_{a+p^m\Z_p}$ eine Schwartzfunktion.
\end{bsp}

Tatsächlich sind alle Schwartzfunktionen quasi von dieser Form.

\begin{prop}
Sei $f \in \Sq$. Dann ist $f$ eine endliche Linearkombination von Funktionen der Form $\chi_{a+p^m\Z_p}$.
\end{prop}
\begin{proof}
Sei $f \in \Sq$. Da $f$ lokal konstant ist, ist $f^{-1}(0)$ offen.
Insbesondere ist somit $\Q_p \setminus f^{-1}(0)$ abgeschlossen.
Also ist $\Q_p\setminus f^{-1}(0)=\supp(f)$ kompakt.
Wir können es durch die offenen Mengen $f^{-1}(z)$ für $z \not =0$ überdecken.
Aufgrund der Komapktheit reichen endlich viele dieser Mengen $f^{-1}(z)$ und $f$ hat endliches Bild.
Jede der offene Mengen $f^{-1}(z)$ ist eine Vereinigung offener Bälle.
Also ist $\supp(f)$ die Vereinigung von endlich vielen Bällen auf denen $f$ konstant ist.
\end{proof}


\begin{defi}
Für $f\in \mathit{L}^1(\Q_p)$ definieren wir die Fouriertransformation durch
\begin{align*}
	\hat{f}(y)=\int_{\Q_p} f(x)\ex_p(-xy)~\mathrm{d}x.
\end{align*}
\end{defi}

Ziel ist es die Riemannsche Zeta-Funktion meromorph fortzusetzen.
Für die analytische Fortsetzung benötigen wir die Maschinerie, die wir bisher aufgebaut haben.

Wir sammeln nun einige Eigenschaften der Fouriertransformation, die wir für $\Q_\infty$ bereits aus der Integrationstheorie kennen.

\begin{prop}
Sei $\nu \leq \infty$ und $f \in \mathcal{S}(\Q_\nu)$.
\begin{enumerate}[label=\roman*)]
\item Für $a \in \Q_\nu$ und $g(x)=\ex_\nu(ax)f(x)$ gilt
\begin{align*}
	\hat{g}(x)=\hat{f}(x-a).
\end{align*}
\item Für $a \in\Q_\nu$ und $g(x)=f(x+a)$ gilt
\begin{align*}
	\hat{g}(x)=\ex_\nu(ax)\hat{f}(x).
\end{align*}
\item Für $a \in \Q_\nu^\ast$ und $g(x)=f(ax)$ gilt
\begin{align*}
	\hat{g}(x)=\frac{1}{\abs{a}_\nu}\hat{f}\left(\frac{x}{a}\right).
\end{align*}
\end{enumerate}
\end{prop}
\begin{proof}
Die Beweise funktionieren analog zum bereits Bekannten, für iii) verwendet man Korollar 3.12.
\end{proof}

Wir zeigen nun die lokale Inversionformel für die Fouriertransformation.

\begin{prop}
Sei $f \in \mathcal{S}(\Q_\nu)$. Dann gilt $\hat{f} \in \mathcal{S}(\Q_\nu)$ sowie
\begin{align*}
\hat{\hat{f}}(x)=f(-x).
\end{align*}
\end{prop}
\begin{proof}
Für $\nu=\infty$ ist dies aus Integrationstheorie bekannt.
Sei nun $p<\infty$. Wir führen den Beweis in mehreren Schritten.
Zunächst zeigen wir, dass $f=\chi_{\Z_p}$ ein Fixpunkt der Fouriertransformation ist.
Es gilt
\begin{align*}
\hat{f}(y)&=\int_{\Q_p}\chi_{\Z_p}(x)\ex_p(-xy)~\mathrm{d}x\\
&=\int_{\Z_p}  \ex_p(-xy)~\mathrm{d}x\\
&=\int_{\Z_p} \psi(x)~\mathrm{d}x,
\end{align*}
wobei $\psi =\ex_p(-xy) \colon \Z_p \to T$ ein Charakter ist.
Dieser ist genau dann trivial, wenn $y \in \Z_p$.
In diesem Fall gilt $\hat{f}(y)=1$.
Sei nun $y \not \in \Z_p$. Dann existiert $t \in \Z_p$ mit $\psi(t)\not =1$ und
\begin{align*}
\int_{\Z_p}\psi_(x)~ \mathrm{d}x=\int_{\Z_p}\psi(x+t)~\mathrm{d}x=\psi(t)\int_{\Z_p} \psi(x)~\mathrm{d}x.
\end{align*}
Da $\psi(t)\not =1$, ist dies nur für $\hat{f}(y)=0$ möglich.
Also ist $\chi_{\Z_p}$ ein Fixpunkt der Fouriertransformation.

Als nächstes betrachten die Fouriertransformation von $f=\chi_{p^n \Z_p}$ und zeigen, dass diese $p^{-n}\chi_{p^{-n} \Z_p}$ ist.
Es gilt $f(x)=\chi_{p^n \Z_p}(x)= \chi_{\Z_p}(\frac{x}{p^n})$.
Mit Proposition 7.9 iii) angewendet mit $a=\frac{1}{p^n}$ folgt
\begin{align*}
\hat{f}(x)=\frac{1}{p^n}\chi_{\Z_p}(p^nx)=\frac{1}{p^n}\chi_{p^{-n} \Z_p}(x).
\end{align*}
Nun betrachten $f(x)=\chi_{a+p^n\Z_p}(x)$ und zeigen, dass
\begin{align*}
\hat{f}(y)=p^{-n} \ex_p(-ay)\chi_{p^{-n}\Z_p}(y)
\end{align*}
gilt.
Wir wollen 7.9 i) verwenden und bemerken, dass $f(x)=\chi_{p^n\Z_p}(x-a)$ gilt.
Somit folgt die Behauptung.
Da $\ex_p(-ay)$ lokal konstant ist, folgt $\ex_p(-ay)\chi_{p^{-n}\Z_p}(y) \in \Sq$.
Also bildet die Fouriertransformation $\Sq$ auf $\Sq$ ab.
Es bleibt lediglich die Inversionsformel zu zeigen.
Sei dafür $f(x)=p^{-n} \ex_p(-ax)\chi_{p^{-n}\Z_p}(x)$.
Dann gilt
\begin{align*}
\hat{f}(y)&=p^{-n}\hat{\chi}_{p^{-n}\Z_p} (y+a)\\
&=p^{-n}p^n \chi_{p^n \Z_p}(y+a)\\
&=\chi_{-a+p^n\Z_p}(y)\\
&=\chi_{a+^pn\Z_p}(-y).
\end{align*}
Da für $f=\chi_{a+p^n\Z_p}$ die behauptete Formel gilt und jede Schwartzfunktion eine endliche Summe solcher Funktionen ist, folgt die Behauptung.
\end{proof}

Nach der lokalen Fourieranalysis machen wir nun globale Fourieranalysis auf den Adelen.
Zunächst betrachten wir die endlichen Adele.

\begin{defi}
Eine Funktion $f \colon \A_f \to \C$ heißt \emph{Schwartzfunktion}
falls $f$ lokal konstant ist und kompakten Träger hat.
Wir bezeichnen den komplexen Vektorraum aller Schwartzfunktionen auf $\A_f$ mit $\mathcal{S}(\A_f)$.
\end{defi}

\begin{bsp}
Für $a \in \A_f$ und $\Z \ni m>0$ ist die Funktion $\chi_{a+m\hat{Z}}$ eine Schwartzfunktion.
\end{bsp}

\begin{prop}
Sei $ f\in \mathcal{S}(\A_f)$. Dann ist $f$ eine endliche Linearkombination
von Funktionen der Form $\chi_{a+m\hat{Z}}$.
\end{prop}

\begin{defi}
Sei $f\in \mathit{L}^1(\A)$. Dann ist die Fouriertransformation von $f$ definiert durch
\begin{align*}
\hat{f}(y)=\int_\A f(x)\overline{\ex(xy)}~\mathrm{d}x.
\end{align*}
\end{defi}

\begin{defi}
Eine Funktion $f \colon \A \to \C$ heißt \emph{Schwartzfunktion}, falls $f$ eine endliche Linearkombination von Funktionen der Form
$t(x)=g(x_\infty)h(x_f)$ für $x=(x_\infty,x_f)$ ist.
\end{defi}

\begin{defi}
Seien $g\in \mathcal{S}(\Rn)$ und $h\in \mathcal{S}(\A_f)$.
Eine Funktion der Form
\begin{align*}
f \colon A &\to \C\\
x &\mapsto \prod_{\nu \leq \infty} f_\nu(x_\nu)
\end{align*}
mit $f_\nu \in \mathcal{S}(\Q_\nu)$ für alle $\nu \leq \infty$
sowie $f_p=\chi_{\Z_p}$ für fast alle $p$
heißt \emph{einfache Schwartzfunktion}.
\end{defi}

\begin{lem}
Jede Funktion $f \in \mathcal{S}(\A)$ ist eine endliche Linearkombination einfacher Schwartzfunktionen.
\end{lem}

Für $f \in \mathcal{S}(\A)$ ist $f$ integrierbar und die lokale Inversionformel impliziert das folgende Theorem.

\begin{thm}
Sei $f \in \mathcal{S}(\A)$. Dann gilt $\hat{f}\in \mathcal{S}(\A)$ und
\begin{align*}
\hat{\hat{f}}(x)=f(-x).
\end{align*}
\end{thm}

\begin{thm}[Poissonsche Summenformel]
Sei $f\in \mathcal{S}(\A)$. Dann gilt
\begin{align*}
\sum_{t \in\Q} f(t)=\sum_{t \in \Q}\hat{f}(t)
\end{align*}
und insbesondere sind beide Summen absolut konvergent.
\end{thm}
\begin{proof}
Es genügt, die Behauptung für $f(x)=f_\infty(x_\infty)\chi_{a+N\hat{Z}}(x_f)$ zu zeigen.
Da $\Q\subseteq \A_f$ dicht ist, existiert $t\in \Q$ mit $t \in a+N\hat{Z}$.
Dann gilt auch $t+N\hat{Z}=a+N\hat{Z}$ sowie
\begin{align*}
\sum_{x\in \Q} f(x)&=\sum_{x\in \Q} f_\infty(x_\infty)\chi_{a+N\hat{Z}}(x_f) \\
&=\sum_{x\in \Q} f_\infty(x_\infty)\chi_{t+N\hat{Z}}(x_f)\\
&=\sum_{x\in \Q \cap (t+N\hat{Z})} f_\infty(x_\infty)\\
&=\sum_{y\in \Z} f_\infty(Ny+t).
\end{align*}
Dies impliziert die absolute Konvergenz beider Reihen.
Die Gleichheit kann aus der klassischen Poissonsche Summationsformel gefolgert werden.
Sie kann ebenfalls folgendermaßen bewiesen werden.
Die Funktion
\begin{align*}
F \colon \A &\to \C\\
x &\mapsto \sum_{t\in \Q} f(x+t)
\end{align*}
ist unter $\Q$ invariant und kann daher als Funktion auf $\A/\Q$ aufgefasst werden.
Sie hat eine Fourierentwicklung durch die Charaktere der dualen Gruppe $\widehat{\A/\Q}=\Q$, d.h.\,
\begin{align*}
F(x)=\sum_{\nu \in \Q} a_\nu \ex(\nu x)
\end{align*}
mit
\begin{align*}
a_\nu=\int_{\A/\Q} F(x)\ex(-\nu x)~\mathrm{d}x.
\end{align*}
Es gilt
\begin{align*}
a_\nu&=\int_{\A/\Q} \sum_{t \in \Q} f(x+t) \ex(-\nu x)~\mathrm{d}x\\
&= \sum_{t\in \Q} \int_{\A /\Q} f(x+t)\ex(-\nu x)~\mathrm{d}x\\
&= \sum_{t \in \Q} \int_{A /\Q} f(x+t)\ex(-\nu(x+t))~\mathrm{d}x\\
&=\int_A f(x) \ex(-\nu x)~\mathrm{d}x\\
&=\hat{f}(\nu).
\end{align*}
Also folgt
\begin{align*}
\sum_{t \in \Q} f(x+t)=\sum_{\nu \in \Q} \hat{f}(\nu)\ex(\nu x).
\end{align*}
Für $x=0$ folgt die Behauptung.
\end{proof}