\chapter{Die $p$-adischen Zahlen}
Für $x\in \Q$ definieren wir
\begin{align*}
\abs{x}_\infty\coloneqq\begin{cases}
x &\text{ falls } x\geq 0,\\
-x &\text{ falls } x<0.
\end{cases}
\end{align*}
Dann ist $\abs{\cdot}_\infty$ ein Absolutbetrag.

Sei $p>0$ eine Primzahl. Wir schreiben $x\in \Q^\ast$ als $x=p^n \frac{a}{b}$ mit $p \nmid ab$ und definieren
\begin{align*}
\abs{x}_p \coloneqq p^{-n}
\end{align*}
und setze $\abs{0}_p=0$.
Dann ist der $p$-adische Betrag $\abs{\cdot}_p$ ein nicht-archimedischer Absolutbetrag auf $\Q$.
Es gilt
\begin{align*}
\abs{x+y}_p\leq \max{(\abs{x}_p,\abs{y}_p)}.
\end{align*}
Ist $\abs{x}_p \not= \abs{y}_p$ so gilt Gleichheit.
\begin{bsp}

\begin{enumerate}[label=\roman*)]
\item Die Folge $1,p,p^2,\dots$ konvergiert $p$-adisch gegen $0$, da $d_p(0,p^n)=p^{-n}\to 0$.
\item Die Folge $1,\frac{1}{10},\frac{1}{10^2},\dots$ ist keine Cauchy-Folge bezüglich $\abs{\cdot}_p$.
\end{enumerate}
\end{bsp}

\begin{thm}[Ostrowski]
Ein Absolutbetrag auf $\Q$ ist äquivalent zu $\abs{\cdot}_\infty$ oder zu einem $\abs{\cdot}_p$ für eine Primzahl $p$.
\end{thm}

\begin{satz}[Produktformel]
Sei $x\in \Q^\ast$. Dann gilt
\begin{align*}
\abs{x}_\infty \prod_p \abs{x}_p=1.
\end{align*}
\end{satz}

\begin{bem}
Die Primzahlen $2,3,\dots$ werden als \emph{endliche} Primzahlen bezeichnet und $\infty$ als \emph{unendliche} Primzahl.
Wir bezeichnen endliche Primzahlen meist mit $p$ und beliebige Primzahlen mit $\nu$.
\end{bem}

\begin{defi}
Wir fixieren eine endliche Primzahl $p$ und definieren
\begin{align*}
K(x,r)\coloneqq\{y \in \Q \mid d_p(x,y)<r\},\\
\overline{K}(x,r) \coloneqq \{y\in \Q \mid d_p(x,y)\leq r\}.
\end{align*}
\end{defi}
