% Makros

\newcommand{\N}{\mathds{N}} % natuerliche Zahlen
\newcommand{\Z}{\mathds{Z}} % ganze Zahlen
\newcommand{\Q}{\mathds{Q}} % rationale Zahlen
\newcommand{\R}{\mathds{R}} % reelle Zahlen
\newcommand{\C}{\mathds{C}} % komplexe zahlen
\newcommand{\A}{\mathbb{A}} % Adele
\newcommand{\CC}{\mathrm{C}} %Differenzierbarkeit
\newcommand{\Sq}{\mathcal{S}(\Q_p)} %Schwartzraum auf Q_p

\DeclareMathOperator{\Ret}{Re} % Guter Realteil
\DeclareMathOperator{\Imt}{Im} % Guter Imaginaerteil
\DeclareMathOperator{\diag}{diag} %diagonal matrix
\DeclareMathOperator{\supp}{supp}%traeger
\DeclareMathOperator{\loc}{loc}
\DeclareMathOperator{\dist}{dist}%abstand
\DeclareMathOperator{\diam}{diam}%Durchmesser
\DeclareMathOperator{\tr}{tr}%Trace
\DeclareMathOperator{\GL}{GL}%Allgemeine Lineare Gruppe
\DeclareMathOperator{\SL}{SL}%spezielle Lineare Gruppe
\DeclareMathOperator{\SO}{SO} %spezielle orthogonale Gruppe
\DeclareMathOperator{\Oo}{O} %orthongonale Gruppe
\DeclareMathOperator{\End}{End} %Endomorphismen
\newcommand{\Po}{\mathcal{P}} %Potenzmenge
\DeclareMathOperator*{\Product}{\widehat{\prod}}

\newcommand{\ex}{\mathrm{e}}

\newcommand{\norm}[1]{\|#1\|} %Norm
\newcommand{\abs}[1]{|#1|} %Betrag
\newcommand{\Rn}{{\mathds{R}^n}} %Rn
\newcommand{\Srn}{\mathcal{S}(\Rn)} %Schwartzraum
\newcommand{\Ssrn}{\mathcal{S}'(\Rn)} %temperierte Distributionen
\newcommand{\F}{\mathcal{F}} %Fourier transform
\newcommand{\Lr}{\mathrm{L}} %Lp raeume
\newcommand*\Laplace{\mathop{}\!\mathbin\bigtriangleup} %Laplaceoperator
\newcommand*\grad{\mathop{}\!\mathbin\bigtriangledown} %Gradient

%Differentialoperatoren
\let\divsymb=\div % rename builtin command \div to \divsymb
\renewcommand{\div}[1]{\mathrm{div\,} #1} % for divergence

% Umgebungen f�r Definitionen, S�tze, usw.

\theoremstyle{definition}
\newtheorem{defi}{Definition}[chapter]
\newtheorem{bsp}[defi]{Beispiel}


\theoremstyle{plain}
\newtheorem{thm}[defi]{Theorem}
\newtheorem{satz}[defi]{Satz}
\newtheorem{lem}[defi]{Lemma}
\newtheorem{bem}[defi]{Bemerkung}
\newtheorem{prop}[defi]{Proposition}
\newtheorem{cor}[defi]{Korollar}


\theoremstyle{remark}
\newtheorem*{Bemerkung}{Bemerkung}


\def\Satzrefname{Satz}