\chapter{Die Gruppe $\GL_2(\A)$}
Wir erinnern zunächst an einige Eigenschaften der Idele und Adele.
Die endlichen Adele $A_f$ sind definiert durch
\begin{align*}
A_f&=\hat{\prod_{p<\infty}}^\Z_p \Q_p\\
&=\{(x_2,x_3,\dots) \mid x_p \in \Q_p, x_p\in \Z_p \text{ für fast alle } p\}.
\end{align*}
Wir haben $\A_f$ mit eingeschränkten Produkttopolgie ausgestattet.
Dann ist
\begin{align*}
\hat{Z}=\prod_{p <\infty} \Z_p \subseteq \A_f
\end{align*}
kompakt.
Die Adele sind definiert durch $\A=\A \times \A_f$.
Weiterhin haben wir definiert
\begin{align*}
\A_f^\ast= \hat{\prod_{p<\infty}}^{\Z_p^\ast} \Q_p^\ast.
\end{align*}
Dann ist $\hat{Z}^\ast=\prod_{p<\infty} \Z_p^\ast \subseteq \A_f^\ast$ kompakt.
Die Idele sind durch $\A^\ast=\R^\ast \times \A_f^\ast$ definiert.
Außerdem ist durch
\begin{align*}
\Q &\hookrightarrow \A_f\\
a&\mapsto (a,a,\dots)
\end{align*}
eine Einbettung definiert.
\begin{prop}[Schwache Approximation]
$\Q$ ist dicht in $\A_f$.
\end{prop}
\begin{proof}
Dies folgt mit dem Chinesischen Restsatz.
\end{proof}
\begin{prop}[Starke Approximation für $\A$]
$D=[0,1)\times \hat{Z}\subseteq \A$ ist ein Fundamentalbereich für die
Operation von $\Q$ auf $\A$, das heißt es gilt
\begin{align*}
\A=\bigcup_{\alpha\in \Q} (\alpha+D)
\end{align*}
als disjunkte Vereinigung.
\end{prop}

\begin{prop}[Starke Approximation für $\A^*$]
$D=(0,\infty)\times\hat{\Z}^*$ ist ein Fundamentalbereich für $\Q^\ast \setminus \A^\ast$, das heißt es gilt
\begin{align*}
\A^\ast=\bigcup_{\alpha\in \Q_p^\ast} \alpha D
\end{align*}
als disjunkte Vereinigung.
\end{prop}
\begin{proof}
Für $x\in \A_f^\ast$ gilt $\abs{x}x\in \hat{Z}^\ast$.
\end{proof}

Für einen Ring $R$ beschreibe $\GL_2(R)$ die Gruppe aller invertierbaren $2\times 2$-Matrizen.
Es gilt
\begin{align*}
\GL_2(\A_f)&=\hat{\prod_{p<\infty}}^{K_p} \GL_2(\Q_p)
\end{align*}
wobei $K_p=\GL_2(\Z_p)$ ist.
Wir statten daher $\GL_2(\A_f)$ mit der eingeschränkten Produkttopologie aus.
Definiere
\begin{align*}
\SL_2(\A_f)\coloneqq \{g \in \GL_2(\A_f) \mid \det(g)=1\}
\end{align*}
sowie
\begin{align*}
\SL_2(\Q)&\hookrightarrow \SL_2(\A_f)\\
g &\mapsto (g,g,\dots).
\end{align*}
Die folgende Aussage ist falsch, wenn man $\SL_2$ durch $\GL_2$ ersetzt.
\begin{prop}
$\SL_2(\Q)$ ist dicht in $\SL_2(\A_f)$.
\end{prop}
\begin{proof}
Sei $X$ der Abschluss von $\SL_2(\Q)$ in $\SL_2(\A_f)$.
Da Multiplikation in $\SL_2(\A_f)$ stetig ist, ist $X$ eine Gruppe.
Wir wissen bereits, dass $\Q$ dicht in $\A_f$ ist.
Also enthält $X$ alle Matrizen der Form
\begin{align*}
\begin{pmatrix}
1&x\\
0&1
\end{pmatrix},\begin{pmatrix}
1&0\\
y&1
\end{pmatrix}
\end{align*}
für $x,y\in \A_f$.
Für eine fixe Primzahl $p$ enthält $X$ alle Elemente der Form $g=(g_2,g_3,\dots)$
mit
\begin{align*}
g_p=\begin{pmatrix}
1&x_p\\
0&1
\end{pmatrix},\begin{pmatrix}
1&0\\
y_p&1
\end{pmatrix}
\end{align*}
mit $x_p,y_p \in \Q_p$
sowie
\begin{align*}
g_q=\begin{pmatrix}
1&0\\
0&1
\end{pmatrix}
\end{align*}
für $q\not =p$.
Die Matrizen der Form
\begin{align*}
\begin{pmatrix}
1&x_p\\
0&1
\end{pmatrix},\begin{pmatrix}
1&0\\
y_p&1
\end{pmatrix}
\end{align*}
mit $x_p,y_p \in \Q_p$ erzeugen $\SL_2(\Q_p)$.
Daher enthält $X$ alle Matrizen der Form $g=(g_2,g_3,\dots)$ mit $g_p \in \SL_2(\Q_p)$ and $g_p=\begin{pmatrix}
1&0\\
0&1
\end{pmatrix}$ für fast alle $p$.
Diese sind dicht in $\SL_2(\A_f)$.
\end{proof}

Es gilt $\GL_2(\A)=\GL_2(\R)\times \GL_2(\A_f)$.
\begin{prop}[Starke Approximation für $\GL_2(\A)$]
Sei $D_\infty$ ein Fundamentalbereich für $\GL_2(\Z)\setminus \GL_2(\R)$,
das heißt $D_\infty$ enthält einen Repräsentanten für jedes Element in
\begin{align*}
\GL_2(\Z)\setminus \GL_2(\R).
\end{align*}
Ein Fundamentalbereich für $\GL_2(\Q)\setminus \GL_2(\A)$ ist durch
\begin{align*}
D\coloneqq D_\infty \GL_2(\hat{Z})
\end{align*}
gegeben.
\end{prop}
\begin{proof}
Wir zeigen, dass jedes Element $g\in \GL_2(\A)$ von der Form $g=\gamma d$ mit $\gamma \in \GL_2(\Q)$ und $d\in D$ geschrieben werden kann.
Wir schreiben dazu $\det(g)=\underbrace{\alpha}_{\in \Q^\ast} \underbrace{x}_{\in (0,\infty) \hat{Z}}$ nach der starken Approximation für Idele.
Definiere
\begin{align*}
g'=\begin{pmatrix}
\alpha^{-1}&0\\
0&1
\end{pmatrix}
g
\begin{pmatrix}
x^{-1}&0\\
0&1
\end{pmatrix} \in \SL_2(\A).
\end{align*}
Sei nun $U$ eine offene Umgebung der $1$ in $\SL_2(\R)$.
Dann ist
\begin{align*}
g'(U \cdot \SL_2(\hat{\Z})
\end{align*}
eine offene Umgebung von $g'$ in $\SL_2(\A)$.
Aufgrund der Dichtheit von $\SL_2(\Q)$ in $\SL_2(\A_f)$ enthält diese ein Element der Form
\begin{align*}
\underbrace{\gamma_\infty}_{\in \GL_2(\R)} \underbrace{\gamma}_{\in \SL_2(\Q)}.
\end{align*}
Dann gilt
\begin{align*}
\gamma_\infty \gamma =g' \gamma_\infty' \gamma'
\end{align*}
für geeignete $\gamma_\infty',\gamma'$,
sodass
\begin{align*}
g&=\begin{pmatrix}
\alpha&0\\
0&1
\end{pmatrix} g' \begin{pmatrix}
x&0\\
0&1
\end{pmatrix}\\
&=\begin{pmatrix}
\alpha&0\\
0&1
\end{pmatrix} (\underbrace{\gamma_\infty \gamma_\infty'^{-1}}_{\in \GL_2(\R)} ) (\gamma \gamma'^{-1} \begin{pmatrix}
x&0\\
0&1
\end{pmatrix}\\
&=\underbrace{\begin{pmatrix}
\alpha&0\\
0&1
\end{pmatrix} \gamma}_{\in \GL_2(\Q)} \underbrace{(\gamma_\infty \gamma'^{-1}) \begin{pmatrix}
x_\infty&0\\
0&1
\end{pmatrix}} _{\in \GL_2(\R)}
\underbrace{\gamma'^{-1} \begin{pmatrix}
x_f&0\\
0&1
\end{pmatrix}}_{\in \GL_2(\hat{Z})}.
\end{align*}
Es folgt, dass $g$ eine Zerlegung der Form
\begin{align*}
g=\mu d_\infty h
\end{align*}
besitzt.
Die Eindeutigkeit der Zerlegung verbleibt als Übung.
\end{proof}

Nun betrachten wir die Iwasawa Zerlegung für Adele.
\begin{prop}[Adelische Iwasawa Zerlegung]
Sei $g\in \GL_2(\A)$.
Dann kann $g$ eindeutig als
\begin{align*}
g=\begin{pmatrix}
1&x\\
0&1
\end{pmatrix}
\begin{pmatrix}
y&0\\
0&1
\end{pmatrix}
\begin{pmatrix}
r&0\\
0&r
\end{pmatrix}
k
\end{align*}
mit $x\in \A$, $r,y\in \A^\ast$, $k\in K=\Oo_2(\R)\GL_2(\hat{Z})$ und $y_\infty,r_\infty>0$  geschrieben werden.
\end{prop}
\begin{proof}
Dies folgt durch Zusammensetzen der bisherigen Resultate.
\end{proof}

$\GL_2(\C)$ operiert auf auf $\mathbb{P}^1(\C)=\C\cup \{\infty\}$ durch
\begin{align*}
\begin{pmatrix}
a&b\\
c&d
\end{pmatrix}z=\frac{az+b}{cz+d}.
\end{align*}
Dies Operation ist transitiv.
Für $M=\begin{pmatrix}
a&b\\
c&d
\end{pmatrix} \in \GL_2(\R)$ gilt
\begin{align*}
\Imt(Mz)=\frac{\det(M)}{\abs{cz+d}^2} \Imt(z).
\end{align*}
Wir definieren eine Operation von $\GL_2(\R)$ auf $H$ durch
\begin{align*}
\begin{pmatrix}
a&b\\
c&d
\end{pmatrix} z&=\frac{az+b}{cz+d}~~\text{falls } \det(\begin{pmatrix}
a&b\\
c&d
\end{pmatrix})>0\\
\begin{pmatrix}
a&b\\
c&d
\end{pmatrix} z&=\overline{\frac{az+b}{cz+d}}~~\text{falls } \det(\begin{pmatrix}
a&b\\
c&d
\end{pmatrix})<0.
\end{align*}
Die Operation von $\GL_2(\R)$ auf $H$ wird von $M \in \SL_2(\R)$ und $\begin{pmatrix}
-1&0\\
0&1
\end{pmatrix}$
erzeugt.
Der Stabilisator von $\mathrm{i}$ ist $\O_2(\R)\cdot \R^\ast$ und die Abbildung
\begin{align*}
\GL_2(\R)/\GL_2(\R)_\mathrm{i} &\to H\\
g \GL_2(\R)_\mathrm{i} &\mapsto g\mathrm{i}
\end{align*}
ist ein Isomorphismus.
Ein Fundamentalbereich für die Operation von $\GL_2(\Z)$ auf $H$ ist durch
\begin{align*} 
\{z\in \C \mid -\frac{1}{2} \leq \Ret(z)\leq 0 \mid \abs{z}\geq1\}
\end{align*}
gegeben.

\begin{prop}
Sei $g \in \GL_2(\A)$. Dann hat $g$ eine eindeutige Darstellung der Form
\begin{align*}
g=\gamma \underbrace{\begin{pmatrix}
y&x\\
c0&1
\end{pmatrix}}_{\in \GL_2(\R)} \underbrace{\begin{pmatrix}
r&0\\
0&r
\end{pmatrix}}_{\in \GL_2(\R)} k
\end{align*}
mit $\gamma \in \GL_2(\Q)$ (diagonal eingebettet) sowie $x,y\in \R$
mit $x+\mathrm{i}y \in \{z \in \C \mid -\frac{1}{2} \leq \Ret(z) \leq 0, \abs{z}\geq 1\}$, $r \in \R_{>0}$ und $k\in K=\Oo_2(\R) \GL_2(\hat{\Z})$.
\end{prop}
\begin{proof}
Wir wissen bereits aufgrund der starken Approximation, dass $g \in GL_2(\A)$ eindeutig in der Form
\begin{align*}
g=\underbrace{\gamma}_{\in \GL_2(\Q)} \underbrace{d_\infty}_{\in D_\infty} \underbrace{h}_{\in \GL_2(\hat{Z})}
\end{align*}
mit $D_\infty=\GL_2(\Z)\setminus \GL_2(\R)$.
Aus
\begin{align*}
H &\overset{\to}{\cong} \GL_2(\R) / \Oo_2(\R)\cdot \R^\ast\\
x+\mathrm{i}y &\mapsto \begin{pmatrix}
y&x\\
0&1
\end{pmatrix} \Oo_2(\R)\cdot \R^\ast
\end{align*}
erhalten wir
\begin{align*}
\GL_2(\Z)\setminus H =\GL_2(\Z) \setminus \GL_2(\R) / \Oo_2(\R)\cdot \R^\ast
\end{align*}
%TODO links operiert \GL_2(\Z) durch "fractional transformation" rechts durch Multiplikation von links.
sodass 
\begin{align*}
D_\infty&=\GL_2(\Z) \setminus \GL_2(\R)
&=\left\lbrace \begin{pmatrix}
y&x\\
0&1
\end{pmatrix} \mid x+\mathrm{i}y \in C\right\rbrace \Oo_2(\R) \R^\ast
\end{align*}
mit $C=\{z \in \C \mid -\frac{1}{2}\leq  \Ret(z) \leq 0, \abs{z}\geq 1\}$.
\end{proof}

\begin{prop}
Sei $g \in \GL_2(\A)$. Dann kann $g$ eindeutig in der Form
\begin{align*}
g=k_1 a k_2
\end{align*}
mit $k_1,k_2 \in K=\Oo_2(\R) \GL_2(\hat{Z})$ und
\begin{align*}
a=\left\lbrace \begin{pmatrix}
yr&0\\
0&r
\end{pmatrix}, \begin{pmatrix}
2^{m_2}&0\\
0&2^{n_2}
\end{pmatrix},\dots \right\rbrace
\end{align*}
und $y,r \in \R^\ast_{>0}$.
\end{prop}