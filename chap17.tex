\chapter{Die Gruppe ´$\GL_2(\A)$}
Wir erinnern zunächst an einige Eigenschaften der Idele und Adele.
Die endlichen Adele $A_f$ sind definiert durch
\begin{align*}
A_f&=\hat{\prod_{p<\infty}}^\Z_p \Q_p\\
&=\{(x_2,x_3,\dots) \mid x_p \in \Q_p, x_p\in \Z_p \text{ für fast alle } p\}.
\end{align*}
Wir haben $\A_f$ mit eingeschränkten Produkttopolgie ausgestattet.
Dann ist
\begin{align*}
\hat{Z}=\prod_{p <\infty} \Z_p \subseteq \A_f
\end{align*}
kompakt.
Die Adele sind definiert durch $\A=\A \times \A_f$.
Weiterhin haben wir definiert
\begin{align*}
\A_f^\ast= \hat{\prod_{p<\infty}}^{\Z_p^\ast} \Q_p^\ast.
\end{align*}
Dann ist $\hat{Z}^\ast=\prod_{p<\infty} \Z_p^\ast \subseteq \A_f^\ast$ kompakt.
Die Idele sind durch $\A^\ast=\R^\ast \times \A_f^\ast$ definiert.
Außerdem ist durch
\begin{align*}
\Q &\hookrightarrow \A_f\\
a&\mapsto (a,a,\dots)
\end{align*}
eine Einbettung definiert.
\begin{prop}[Schwache Approximation]
$\Q$ ist dicht in $\A_f$.
\end{prop}
\begin{proof}
Dies folgt mit dem Chinesischen Restsatz.
\end{proof}
\begin{prop}[Starke Approximation für $\A$]
$D=[0,1)\times \hat{Z}\subseteq \A$ ist ein Fundamentalbereich für die
Operation von $\Q$ auf $\A$, das heißt es gilt
\begin{align*}
\A=\bigcup_{\alpha\in \Q} (\alpha+D)
\end{align*}
als disjunkte Vereinigung.
\end{prop}

\begin{prop}[Starke Approximation für $\A^*$]
$D=(0,\infty)\times\hat{\Z}^*$ ist ein Fundamentalbereich für $\Q^\ast \setminus \A^\ast$, das heißt es gilt
\begin{align*}
\A^\ast=\bigcup_{\alpha\in \Q_p^\ast} \alpha D
\end{align*}
als disjunkte Vereinigung.
\end{prop}
\begin{proof}
Für $x\in \A_f^\ast$ gilt $\abs{x}x\in \hat{Z}^\ast$.
\end{proof}

Für einen Ring $R$ beschreibe $\GL_2(R)$ die Gruppe aller invertierbaren $2\times 2$-Matrizen.
Es gilt
\begin{align*}
\GL_2(\A_f)&=\hat{\prod_{p<\infty}}^{K_p} \GL_2(\Q_p)
\end{align*}
wobei $K_p=\GL_2(\Z_p)$ ist.
Wir statten daher $\GL_2(\A_f)$ mit der eingeschränkten Produkttopologie aus.
Definiere
\begin{align*}
\SL_2(\A_f)\coloneqq \{g \in \GL_2(\A_f) \mid \det(g)=1\}
\end{align*}
sowie
\begin{align*}
\SL_2(\Q)&\hookrightarrow \SL_2(\A_f)\\
g &\mapsto (g,g,\dots).
\end{align*}
Die folgende Aussage ist falsch, wenn man $\SL_2$ durch $\GL_2$ ersetzt.
\begin{prop}
$\SL_2(\Q)$ ist dicht in $\SL_2(\A_f)$.
\end{prop}
\begin{proof}
Sei $X$ der Abschluss von $\SL_2(\Q)$ in $\SL_2(\A_f)$.
Da Multiplikation in $\SL_2(\A_f)$ stetig ist, ist $X$ eine Gruppe.
Wir wissen bereits, dass $\Q$ dicht in $\A_f$ ist.
Also enthält $X$ alle Matrizen der Form
\begin{align*}
\begin{pmatrix}
1&x\\
0&1
\end{pmatrix},\begin{pmatrix}
1&0\\
y&1
\end{pmatrix}
\end{align*}
für $x,y\in \A_f$.
Für eine fixe Primzahl $p$ enthält $X$ alle Elemente der Form $g=(g_2,g_3,\dots)$
mit
\begin{align*}
g_p=\begin{pmatrix}
1&x_p\\
0&1
\end{pmatrix},\begin{pmatrix}
1&0\\
y_p&1
\end{pmatrix}
\end{align*}
mit $x_p,y_p \in \Q_p$
sowie
\begin{align*}
g_q=\begin{pmatrix}
1&0\\
0&1
\end{pmatrix}
\end{align*}
für $q\not =p$.
Die Matrizen der Form
\begin{align*}
\begin{pmatrix}
1&x_p\\
0&1
\end{pmatrix},\begin{pmatrix}
1&0\\
y_p&1
\end{pmatrix}
\end{align*}
mit $x_p,y_p \in \Q_p$ erzeugen $\SL_2(\Q_p)$.
Daher enthält $X$ alle Matrizen der Form $g=(g_2,g_3,\dots)$ mit $g_p \in \SL_2(\Q_p)$ and $g_p=\begin{pmatrix}
1&0\\
0&1
\end{pmatrix}$ für fast alle $p$.
Diese sind dicht in $\SL_2(\A_f)$.
\end{proof}

Es gilt $\GL_2(\A)=\GL_2(\R)\times \GL_2(\A_f)$.
\begin{prop}[Starke Approximation für $\GL_2(\A)$]
Sei $D_\infty$ ein Fundamentalbereich für $\GL_2(\Z)\setminus \GL_2(\R)$,
das heißt $D_\infty$ enthält einen Repräsentanten für jedes Element in
\begin{align*}
\GL_2(\Z)\setminus \GL_2(\R).
\end{align*}
Ein Fundamentalbereich für $\GL_2(\Q)\setminus \GL_2(\A)$ ist durch
\begin{align*}
D\coloneqq D_\infty \GL_2(\hat{Z})
\end{align*}
gegeben.
\end{prop}
\begin{proof}
Wir zeigen, dass jedes Element $g\in \GL_2(\A)$ von der Form $g=\gamma d$ mit $\gamma \in \GL_2(\Q)$ und $d\in D$ geschrieben werden kann.
\end{proof}