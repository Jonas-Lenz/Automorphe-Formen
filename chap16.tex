\chapter{Zerlegungen von $\GL_2(\Q_p)$}
Sei
\begin{align*}
\GL_2(\Q_p)&=\left\lbrace \begin{pmatrix}
a&b\\
c&d
\end{pmatrix} \mid a,b,c,d \in \Q_p, \det \begin{pmatrix}
a&b\\
c&d
\end{pmatrix} \not =0\right\rbrace\\
&=\left\lbrace \begin{pmatrix}
a&b\\
c&d
\end{pmatrix} \mid a,b,c,d\in \Q_p \abs{ad-bc}_p \not =0\right\rbrace
\end{align*}
sowie
\begin{align*}
K_p=\GL_2(\Z_p^\ast)&=\left\lbrace \begin{pmatrix}
a&b\\
c&d
\end{pmatrix}\mid a,b,c,d \in \Z_p, ad-bc \in \Z_p^\ast\right\rbrace\\
&= \left\lbrace \begin{pmatrix}
a&b\\
c&d
\end{pmatrix} \mid a,b,c,d\in \Z_p, \abs{ad-bc}_p=1\right\rbrace.
\end{align*}

\begin{prop}
$K_p$ ist eine maximale kompakte Untergruppe von $\GL_2(\Q_p)$.
\end{prop}
Der Beweis verbleibt als Übungsaufgabe

\begin{prop}[Iwasawa Zerlegung]
Sei $g \in \GL_2(\Q_p)$. Dann kann $g$ eindeutig in der Form
\begin{align*}
g=\begin{pmatrix}
1&x\\
0&1
\end{pmatrix} \begin{pmatrix}
p^m&0\\
0&p^n
\end{pmatrix}
k
\end{align*}
mit $x\in \Q_p$ sowie $k\in K_p$.
\end{prop}
\begin{proof}
Sei $g=\begin{pmatrix}
a&b\\
c&d
\end{pmatrix}$. Wir nehmen zuerst an, dass $\abs{c}_p\leq \abs{d}_p$ gilt.
Dann folgt
\begin{align*}
\begin{pmatrix}
a&b\\
c&d
\end{pmatrix}\begin{pmatrix}
1&0\\
-\frac{c}{d}&1
\end{pmatrix}
=\begin{pmatrix}
\ast&\ast\\
0&\ast
\end{pmatrix}
\end{align*}
und $\abs{\frac{c}{d}}_p=1$ das heißt $\frac{c}{d}\in \Z_p$.
Also folgt
\begin{align*}
\begin{pmatrix}
a&b\\
c&d
\end{pmatrix}=\begin{pmatrix}
r&s\\
0&u
\end{pmatrix}
\underbrace{\begin{pmatrix}
1&0\\
-\frac{c}{d}&1
\end{pmatrix}^{-1}}_{\in K_p}
\end{align*}
mit $ru\not =0$.
Es gilt
\begin{align*}
\begin{pmatrix}
r&s\\
0&u
\end{pmatrix}=\begin{pmatrix}
1&\frac{s}{u}\\
0&1
\end{pmatrix}
\begin{pmatrix}
r&0\\
0&u
\end{pmatrix}
\end{align*}
sowie
\begin{align*}
\begin{pmatrix}
r&0\\
0&u
\end{pmatrix}=\begin{pmatrix}
\varepsilon p^m&0\\
0&\eta p^n
\end{pmatrix}=
\begin{pmatrix}
p^m&0\\
0&p^n
\end{pmatrix}
\begin{pmatrix}
\varepsilon&0\\
0&\eta
\end{pmatrix}
\end{align*}
mit $\eta,\varepsilon \in \Z_p^\ast$.
Also folgt
\begin{align*}
\begin{pmatrix}
a&b\\
c&d
\end{pmatrix}=\begin{pmatrix}
1&\frac{s}{u}\\
0&1
\end{pmatrix} \begin{pmatrix}
p^m&0\\
0&p^n
\end{pmatrix}\begin{pmatrix}
\varepsilon&0\\
0&\eta
\end{pmatrix}\begin{pmatrix}
1&0\\
-\frac{c}{d}&1
\end{pmatrix}^{-1}.
\end{align*}
Dies beweist die Behauptung für den Fall $\abs{c}_p\leq \abs{d}_p$.
Falls $\abs{c}_p\geq \abs{d}_p$ gilt, dann folgt $\frac{d}{c}\in \Z_p$ sowie
\begin{align*}
\begin{pmatrix}
a&b\\
c&d
\end{pmatrix}
\begin{pmatrix}
1&-\frac{d}{c}\\
0&1
\end{pmatrix}=
\begin{pmatrix}
\ast&\ast\\
\ast&0
\end{pmatrix}
\end{align*}
sodass Multiplikation mit $\begin{pmatrix}
0&1\
1&0
\end{pmatrix}\in K$
zum vorherigen Fall führt.
\end{proof}

\begin{prop}[Cartan Zerlegung]
Sei $g\in \GL_2(\Q_p)$. Dann gilt
\begin{align*}
g=k_1 \begin{pmatrix}
p^m&0\\
0&p^n
\end{pmatrix}k_2
\end{align*}
mit $k_1,k_2\in K_p$.
\end{prop}
