\chapter{Idele}
\begin{defi}
Die Einheitengruppe $\A^\ast$ von $\A$ heißt Gruppe der \emph{Idele}.
Es gilt
\begin{align*}
\A =\R \times \A_f, \A_f=\hat{\prod}_{p <\infty}^{\Z_p} \Q_p\\
\A^\ast =\R^\ast \times \A_f^\ast, \A_f^\ast=\{(a_p)_{p<\infty} \mid a_p \in \Q_p^\ast \forall p, a_p \in \Z_p^\ast \text{ a.e. } p\}=\hat{\prod}^{\Z_p^\ast} \Q_p^\ast.
\end{align*}
%TODO schoen untereinander machen
\end{defi}
Wir statten $\A_f^\ast$ mit der eingeschränkten Produkttopologie aus.
$\hat{\Z}=\prod_{p<\infty} \Z_p$ ist offen und kompakt in $\A_f$,
$\hat{\Z}^\ast=\prod_{p<\infty} \Z_p^\ast$ ist offen und kompakt in $\A_f^\ast$ bezüglich der eingeschränkten Produkttopologie.
$\hat{Z}^\ast$ ist nicht offen bezüglich der Topologie von $\A_f^\ast$, die von $\A_f$ induziert wird.
Die eingeschränkte Produkttopologie auf $\A_f^\ast$ ist echt feiner als die Topologie, die von $\A_f$ induziert wird.
Wir statten $\A^\ast=\R^\ast\times \A_f^\ast$ mit der Produkttopologie aus. 

\begin{thm}
Dann sind $(A_f^\ast,\cdot)$ und $(\A^\ast,\cdot)$ lokal kompakte Hausdorffgruppen.
\end{thm}

Wir normieren das Haarmaß auf $\A_f^\ast$ sodass $\hat{\Z}^\ast$ Maß $1$ hat und schreiben $\mathrm{d}^\ast x_f$ für dieses Maß.
Auf $\R^\ast$ ist $\frac{\mathrm{d}t}{\abs{t}}$ ein Haarmaß.
Wir statten $\A^\ast$ mit dem Produktmaß aus, also mit dem Haarmaß
\begin{align*}
\mathrm{d}^\ast x=\frac{\mathrm{d}t}{\abs{t}}\cdot \mathrm{d}^\ast x_f.
\end{align*}

\begin{defi}
Eine \emph{einfache Funktion auf $\A^\ast$} ist eine Funktion der Form
\begin{align*}
f=\prod_{\nu \leq \infty}f_\nu
\end{align*}
mit $f_\nu\in C_c(\Q_\nu^\ast)$ für alle $\nu\leq \infty$ und $f_p=\chi_{\Z_p^\ast}$ für fast alle $p<\infty$.
Für eine solche Funktion definieren wir das Integral über
\begin{align*}
\int_{\A^\ast} f(x)~\mathrm{d}^\ast x=\prod_{\nu \leq \infty} \int_{\Q_\nu^\ast} f_\nu(x)~\mathrm{d}x.
\end{align*}
\end{defi}

Die Abbildung
\begin{align*}
\Q^\ast &\to \A^\ast\\
x&\mapsto (x,x,\dots)
\end{align*}
ist eine Einbettung von $\Q^\ast$ nach $\A^\ast$.

\begin{prop}
$\Q^\ast$ ist eine diskrete und abgeschlossene Untergruppe von $\A^\ast$.
\end{prop}
\begin{proof}
Für $a=(a_\nu)_{\nu \leq \infty} \in \A^\ast$ definieren wir die Norm
\begin{align*}
\abs{a}\coloneqq \prod_{\nu \leq \infty} \abs{a_\nu}_\nu,
\end{align*}
wobei nur endlich viele Faktoren ungleich $1$ sind.
Die Abbildung
\begin{align*}
\A^\ast &\to \R_{>0}^\ast\\
a &\mapsto \abs{x}
\end{align*}
ist ein stetiger Gruppenhomomorphismus.
Der Kern
\begin{align*}
A'\coloneqq \{a \in \A^\ast \mid \abs{a}=1\}
\end{align*}
ist eine abgeschlossene Untergruppe von $\A^\ast$.
Für $x \in \Q^\ast$ gilt
\begin{align*}
\prod_{\nu \leq \infty} \abs{x}_\nu=1.
\end{align*}
Also ist $\Q^\ast$ eine diskrete abgeschlossene Untergruppe von $A'$.
\end{proof}

Der Quotient $A'/\Q^\ast$ ausgestattet mit der Quotiententopologie ist eine lokal kompakte Hausdorffgruppe.
Wir definieren
\begin{align*}
\hat{\Z}^\ast &\to \A'\\
(x_p)_{p<\infty} &\mapsto (1,x_2,x_3,\dots).
\end{align*}

\begin{prop}
Die kanonische Abbildung $f \colon \hat{\Z}^\ast \to \A'/\Q^\ast$
ist ein Isomorphismus topologischer Gruppen.
\end{prop}
\begin{proof}
Wir zeigen zunächst, dass $f$ injektiv ist.
Sei dafür $f((x_p)_{p<\infty})=f((y_p)_{p<\infty})$. Dann gilt
\begin{align*}
(1,x_2,x_3,\dots)=(r (1,y_2,y_3,\dots)
\end{align*}
für ein $r\in \Q^\ast$.
Die erste Komponente impliziert aber $r=1$. Also folgt die Injektivität.
Für die Surjektivität betrachten wir $a=(a_\infty,a_2,a_3,\dots)\in \A'$. Dann gilt
\begin{align*}
1&=\abs{a_\infty}_\infty \prod_{p<\infty}\abs{a_p}_p\\
&=a_\infty \frac{\abs{a_\infty}_\infty}{a_\infty}\prod_{p<\infty} \abs{a_p}_p\\
&\eqqcolon a_\infty \cdot r.
\end{align*}
Definiere $b=(r,a_2,a_3,\dots)$. Dann gilt $b \in \hat{\Z}^\ast$, da
\begin{align*}
\abs{r a_p}_p&=\abs{\frac{\abs{a_\infty}_\infty}{a_\infty}\prod_{q<\infty} \abs{a_q}_q a_p}_p\\
&=\abs{\abs{a_p}_p a_p}_p \abs{\prod_{p\not =q <\infty} \abs{a_q}_q}_p=1
\end{align*}
sowie
$f(b)=ra$.
Die Stetigkeit verbleibt als Übungsaufgabe.
\end{proof}

\begin{cor}
$\A'/\Q^\ast$ ist kompakt.
\end{cor}

Der Isomorphismus $\hat{\Z}=\lim_{\Leftarrow} (\Z/N\Z)$ impliziert
\begin{align*}
\hat{\Z}^\ast=\lim_{\Leftarrow} \left(\Z/N\Z\right)^\ast=A'/\Q^\ast.
\end{align*}