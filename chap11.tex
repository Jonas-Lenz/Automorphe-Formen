\chapter{Automorphe Formen auf $\GL(2,\A)$}

\begin{defi}
Eine Funktion $\Phi \colon \GL(2,\A) \to \C$ heißt \emph{glatt},
wenn für jedes $g_0 \in \GL(2,\A)$ eine offene Umgebung $U$ und eine
glatte Funktion $\Phi_\infty^U \colon \GL(2,\R) \to \C$ existiert, sodass
\begin{align*}
\Phi(g)=\Phi_\infty^U(g_\infty)
\end{align*}
für alle $g \in U$ gilt.
\end{defi}

\begin{defi}
Sei $g \in \GL(2,\A)$ mit Einträgen $a=(a_\infty,a_2,\dots)$ usw.
Definiere
\begin{align*}
\norm{g}=\prod_{\nu \leq \infty} \max\{\abs{a_\nu}_\nu, \abs{b_\nu}_\nu, \abs{c_\nu}_\nu, \abs{d_\nu}_\nu, \abs{a_\nu d_\nu-b_\nu c_\nu}_\nu\}.
\end{align*}
Eine Funktion $\Phi \colon  \GL(2,\Q)\ \GL(2,\A) \to \C$ \emph{wächst moderat}, falls Konstanten $B,C>0$ mit
\begin{align*}
\abs{\Phi(g)} \leq C \norm{g}^B
\end{align*}
für alle $g \in \GL(2,\A)$ existieren.
Sei $K=O(2,\R) \prod_{p<\infty} \GL(2,\Z_p)$
die maximale kompakte Untergruppe von $\GL(2,\A)$.
Eine Funktion $\Phi \colon \GL(2,\A)\to \C$
heißt \emph{rechts $K$-endlich}, wenn die Menge
\begin{align*}
\{\phi \circ R_k\mid k\in K\}
\end{align*}
für jedes $g \in \GL(2,\A)$ einen endlich-dimensionalen Vektorraum definiert, wobei $R_k$ die Rechtstranslation mit $k$ ist.
Sei $Z(U(g))$ der Zentralisator der universellen einhüllenden Algebra von $g \in \GL(2,\C)$.
Eine glatte Funktion $\Phi \colon \GL(2,\A) \to \C$ heißt
\emph{$Z(U(g))$-endlich} wenn die Menge
\begin{align*}
\{D \phi\mid D\in Z(U(g))\}
\end{align*}
einen endlich-dimensionalen Vektorraum erzeugt (Vektorraum von Funktionen).
\end{defi}

\begin{defi}
Sei $\omega\colon \Q^\ast \backslash \A^\ast \to T$ ein Charakter.
Eine \emph{automorphe Form} auf $\GL(2,\A)$ mit Charakter $\omega$ ist eine glatte Funktion
\begin{align*}
\Phi \colon \GL(2,\A) \to \C
\end{align*}
mit
\begin{enumerate}[label=(\roman*)]
\item $\Phi(\gamma g)=\Phi(g)$ für alle $\gamma \in \GL(2,\Q)$ und $g \in \GL(2,\A)$,
\item $\Phi(zg)=\omega(z)\Phi(g)$ für alle $z \in \A^\ast$ und $g \in \GL(2,\A)$,
\item $\Phi$ ist rechts $K$-endlich,
\item $\Phi$ ist $Z(U(g))$-endlich,
\item $\Phi$ wächst moderat.
\end{enumerate}
Solch eine Funktion heißt \emph{Spitzenform}, wenn zusätzlich
\begin{align*}
\int_{\Q\ \A} \Phi \left( \begin{pmatrix}
1 &x\\
0 &1
\end{pmatrix}
g\right) ~\mathrm{d}x=0
\end{align*}
für alle $g$ gilt.
\end{defi}

\begin{bem}
Automorphe Formen auf $\GL(2,\A)$ können durch Lifts klassischer Modulformen erhalten werden.
\end{bem}