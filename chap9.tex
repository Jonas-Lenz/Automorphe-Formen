\chapter{Dirichletsche $L$-Reihe}
Sei $\chi \colon (\Z/N\Z)^\ast \to T$ ein Dirichletcharakter modulo $N$.
Wir setzen $\chi$ zu einer Funktion auf $\Z$ durch
\begin{align*}
\chi(a)=\begin{cases}
\chi(a \mod N),~~&(a,N)=1,\\
0, &(a,N)>1.
\end{cases}
\end{align*}
fort und definieren die \emph{$L$-Reihe} von $\chi$ durch
\begin{align*}
L(\chi,s)\coloneqq \sum_{n=1} \frac{\chi(n)}{n^s}.
\end{align*}
Diese Reihe konvergiert für $\Ret(s)>1$ und definiert dort eine holomorphe Funktion.
Es gilt
\begin{align*}
L(\chi,s)=\prod_{p>0} \frac{1}{1-\chi(p)p^{-s}}.
\end{align*}

Erinnerung: Sei $\omega$ ein endlicher Charakter von $\A^\ast/\Q^\ast$.
Wir betrachten $\omega$ als Charakter auf $\A^\ast$.
Dann sind fast alle $\omega_p$ unverzweigt, das heißt $\Z_p^\ast \subseteq \ker(\omega_p)$.
Falls $\omega_p$ verzweigt, so existiert ein minimales $\Z\ni m_p >0$ mit
\begin{align*}
1+K(0,p^{-m_p})\subseteq \ker(\omega_p).
\end{align*}
Der Führer von $\omega$ ist durch $N=\prod_{p \text{verzweigt}} p^{m_p}$ definiert.
Dann definiert $\omega$ durch $\chi(p)=\overline{\omega_p(p)}$ für $p\nmid N$ einen primitiven Dirichletcharakter modulo $N$.
Wenn wir $\chi$ zu einem Charakter auf $\A^\ast/\Q^\ast$ liften, erhalten wir $\omega$ zurück.

\begin{defi}
Sei $\omega$ ein endlicher Charakter auf $\A^\ast/\Q^\ast$ und $f \in \mathcal{S}(A)$.
Wir definieren die twisted $\zeta$-Funktion
\begin{align*}
\zeta(f,\omega,s)\coloneqq \int_{\A^\ast} f(x)\omega(x)\abs{x}^s~\mathrm{d}^\ast x.
\end{align*}
\end{defi}
%TODO wie heißt twisted auf deutsch?!
Wie im letzten Kapitel kann man zeigen, dass das Integral lokal gleichmäßig für $\Ret(s)>1$ konvergiert und dort eine holomorphe Funktion definiert.
Ebenso gilt
\begin{align*}
\zeta(f,\omega,s)=\int_{\A^\ast/\Q^\ast} E(f)(x)\omega(x)\abs{x}^s~\mathrm{d}^\ast x,
\end{align*}
wobei $E(f)$ wie im letzten Kapitel definiert ist (siehe Definition 8.3).

Als Vorbereitungen benötigen wir folgendes Lemma.
\begin{lem}
Sei $K$ eine kompakte Hausdorffgruppe mit Haarmaß $\mathrm{d}x$ und $\chi \colon K\to T$ ein Charakter. Dann gilt
\begin{align*}
\int_K \chi(x)~\mathrm{d}x=\begin{cases}
\operatorname{vol}(K),~~&\text{falls } \chi \text{ trivial ist},\\
0,&\text{sonst.}
\end{cases}
\end{align*}
\end{lem}
%besondere: man kann L-reihen für viele verschiedene objekte definieren, aber diese sind immer gleich
%für modulare Formen hat dies Wiles vor nicht allzulanger Zeit gesagt und ist allgemeiner Teil des Langland Programms.

Nun können wir unseren Satz formulieren.

\begin{thm}
Sei $\omega \not =1$ ein endlicher Charakter von $\A^\ast/\Q^\ast$.
Dann hat $\zeta(f,\omega,s)$ eine holomorphe Fortsetzung auf $\C$ und erfüllt die Funktionalgleichung
\begin{align*}
\zeta(f,\omega,s)=\zeta(\hat{f},\overline{\omega},1-s).
\end{align*}
\end{thm}
\begin{proof}
Wie im Fall $\omega=1$ erhält man
\begin{align*}
\zeta(f,\omega,s)&=\int_{\A^\ast/\Q^\ast} E(f)(x)\omega(x)\abs{x}^s~\mathrm{d}^\ast x\\
&=\int_{\A^\ast/\Q^\ast \abs{x}<1} E(f)(x)\omega(x)\abs{x}^s~\mathrm{d}^\ast x +\int_{\A^\ast/\Q^\ast \abs{x}>1} E(f)(x)\omega(x)\abs{x}^s~\mathrm{d}^\ast x.
\end{align*}
Der zweite Summand konvergiert für alle $s\in \C$ und ist holomorph.
%TODO wieso macht der charakter nichts kaputt?
Für das Integral erhalten wir zuvor mit der Funktionalgleichung für $E(f)$
\begin{align*}
\int_{\A^\ast/\Q^\ast \abs{x}<1} E(f)(x)\omega(x)\abs{x}^s~\mathrm{d}^\ast x \\
=\int_{\abs{x}<1} E(\hat{f})(\frac{1}{x})\omega(x)\abs{x}^{s-1}~\mathrm{d}^\ast x + \hat{f}(0)\int_{\abs{x}<1} \omega(x)\abs{x}^{s-1}~\mathrm{d}^\ast x - f(0) \int_{\abs{x}<1} \omega(x) \abs{x}^s~\mathrm{d}^\ast x.
\end{align*}
Mit der Substitution $y=\frac{1}{x}$ wird das Integral zu
\begin{align*}
\int_{\abs{y}>1} E(\hat{f})(y)\overline{\omega}(y)\abs{y}^{1-s} ~\mathrm{d}^\ast y.
\end{align*}
Als nächstes zeigen wir, dass die beiden anderen Integrale verschwinden. Es gilt
\begin{align*}
\int_{\abs{x}<1}\omega(x)\abs{x}^{s-1}~\mathrm{d}^\ast x&=\int_{A^1/\Q^\ast} \int_0^1 \omega(x)t^{s-1} \frac{\mathrm{d}t}{t}~\mathrm{d}^\ast x\\
&=\int_0^1 t^{s-2}~\mathrm{d}t \int_{A^1/\Q^\ast} \omega(x)~\mathrm{d}x.
\end{align*}
Da wir $\omega$ als nicht trivial vorausgesetzt haben, folgt mit Lemma 9.2, dass das zweite Integral verschwindet.
Das dritte Integral lässt sich genauso behandeln.
Daraus folgt nun
\begin{align*}
\zeta(f,\omega,s)=\int_{\A^\ast/\Q^\ast \abs{x}>1} E(f)(x)\omega(x)\abs{x}^s +E(\hat{f})(x)\overline{\omega}(x)\abs{x}^{1-s}~\mathrm{d}^\ast x.
\end{align*}
Daraus folgt die Behauptung.
\end{proof}

\begin{thm}
Sei $\omega \not =1$ ein endlicher Charakter von $\A^\ast/\Q^\ast$
mit Führer $N$ und sei $\chi$ der zugehörige primitive Dirichletcharakter modulo $N$.
Dann gibt es $f \in \mathcal{S}(\A)$ mit
\begin{align*}
\zeta(f,\omega,s)=L_\infty(\overline{\chi},s) L(\overline{\chi},s),
\end{align*}
wobei $L_\infty(\overline{\chi},s)=\Gamma(\frac{s+\delta}{2})\pi^{-\frac{s+\delta}{2}}$ für $\delta \in \{0,1\}$, sodass $\omega_\infty(-1)=(-1)^\delta=\overline{\chi}(-1)$ gilt.
Für $L^\ast(\overline{\chi},s)=L_\infty(\overline{\chi},s)L(\overline{\chi},s)$ gilt
\begin{align*}
L^\ast(\overline{\chi},s)=(-1)^\delta N^{-s}\overline{\tau(\omega,\mathrm{e})} L^\ast (\overline{\chi},1-s),
\end{align*}
mit
\begin{align*}
\tau_(\omega,\mathrm{e})=\varphi(N)\int_{1/N \hat{Z} ^\ast} \omega(x)\mathrm{e}(x)~\mathrm{d}^\ast x.
\end{align*}
%eulersche phi funktion, e eulersche Zahl?!
\end{thm}
\begin{proof}
Wir schreiben $N=\prod_{p<\infty} p^{n_p}$. Wir definieren $f\in \mathcal{S}(\A)$ durch $f_p=\chi_{\Z_p}$ für $p \nmid N$, $f_p=p^{n_p}(1-\frac{1}{p})\chi_{1+p^{n_p}\Z_p}$ für $p \mid N$ und $f_\infty(t)=t^s \mathrm{e}^{-\pi t^2}$.
Dann gilt
\begin{align*}
\zeta(f,\omega,s)&=\int_{\A^\ast} f(x)\omega(x)\abs{x}^s~\mathrm{d}^\ast x\\
&=\prod_{\nu \leq \infty} \int_{\Q_\nu^\ast} f_\nu(x_\nu)\omega_\nu(x_\nu)\abs{x_\nu}_\nu^s~\mathrm{d}^\ast x_\nu.
\end{align*}
Also reicht es die einzelne Faktoren $\zeta_\nu(f_\nu,\omega_\nu,s)$ zu berechnen.

Sei $p \nmid N$, also unverzweigt. Dann gilt, da $\Z_p\setminus \{0\}=\bigcup_{j=0}^\infty p^j\Z_p^\ast$
\begin{align*}
\zeta_p &=\int_{\Q_p^\ast \cap \Z_p} \omega_p(x_p)\abs{x}_p^s~\mathrm{d}^\ast x_p\\
&=\sum_{j=0}^\infty \int_{p^j\Z_p^\ast} \omega_p(x_p)\abs{x_p}_p^s ~\mathrm{d}^\ast x_p\\
&=\sum_{j=0}^\infty \int_{\Z_p^\ast} \omega_p(p^jy_p)\abs{p^jy_p}_p^s~\mathrm{d}^\ast y_p.
\end{align*}
Da das Maß invariant unter Multiplikation ist folgt, dass sich obiges Integral zu folgendem Ausdruck vereinfacht
\begin{align*}
\sum_{j=0}^\infty \omega_p(p^j)p^{-js} \int_{\Z_p^\ast} \mathrm{d}^\ast y_p\\
=\sum_{j=0}^\infty \overline{\chi}(p)^j p^{-js}=\frac{1}{1-\overline{\chi}(p)p^{-s}}.
\end{align*}
Sei $p \mid N$, also $p$ verzweigt. In diesem Fall erhalten wir
\begin{align*}
\zeta_p&=p^{n_p}(1-\frac{1}{p})\int_{1+p^{n_p}\Z_p} \omega_p(x_p)\abs{x_p}_p^s~\mathrm{d}^\ast x_p\\
&=\varphi(p^{n_p})\int_{1+p^{n_p}\Z_p} \mathrm{d}^\ast x_p =1,
\end{align*}
da $\Z_p^\ast$ die disjunkte Zerlegung
\begin{align*}
\Z_p^\ast=\bigcup_{j \in (\Z/p^{n_p}\Z)^\ast} (j+p^{n_p} \Z_p)
\end{align*}
besitzt.
Für $\nu=\infty$ und $\delta=0$ ist $\omega_\infty$ trivial.
Es gilt
\begin{align*}
\zeta_\infty=\int_{\R^\ast} \mathrm{e}^{-\pi x^2}\abs{x}\frac{\mathrm{d}x}{\abs{x}}=\Gamma(\frac{s}{2})\pi^{-\frac{s}{2}}.
\end{align*}
Für $\delta=1$ ist $\omega_\infty$ der Signumscharakter und es gilt
\begin{align*}
\zeta_\infty &=\int_{\R^\ast} \mathrm{e}^{-\pi x^2}  x \omega_\infty(x)\abs{x}^s\frac{\mathrm{d}x}{\abs{x}}\\
&=\Gamma(\frac{s+1}{2})\pi^{-\frac{s+1}{2}}.
\end{align*}
Insgesamt folgt
\begin{align*}
\zeta(f,\omega,s)=L_\infty(\overline{\chi},s) L(\overline{\chi},s).
\end{align*}
Die Funktionalgleichung von $L^\ast$ folgt aus
\begin{align*}
\zeta(f,\omega,s)=\zeta(\hat{f},\overline{\omega},1-s).
\end{align*}
Dazu berechnen wir die Fouriertransformierte von $f$.
Für $p \nmid N$ gilt $\hat{f_p}=f_p$.

Für $p \mid N$ gilt mit der Substitution $y=1+t$
\begin{align*}
\hat{f_p}(x)&=\int_{\Q_p} f_p(y)\ex_p(-xy)~\mathrm{d}y\\
&=\varphi(p^{n_p})\int_{\Q_p} \chi_{1+p^{n_p}\Z_p}(y)\ex_p(-xy)~\mathrm{d}y\\
&= \varphi(p^{n_p})\int_{\Q_p} \chi_{p^{n_p} \Z_p}(t)\ex_p(-x(1+t))~\mathrm{d}t\\
&=\varphi(p^{n_p})\ex_p(-x)\int_{\Q_p} \chi_{\Z_p}(p^{-n_p}t)\ex_p(-xp^{n_p}p^{-n_p}t)~\mathrm{d}t
%lustige Dinger aus chapter 4, integrand ist g(p^{-n_p}t)
&=\varphi(p^{n_p})\overline{\ex_p(x)} \abs{p^{n_p}}_p \int_{\Q_p} g(t)~\mathrm{d}t\\
&=(1-\frac{1}{p})\overline{\ex_p(x)} \int_{\Z_p} \ex_p(-xp^{n_p}t)~\mathrm{d}t\\
&=(1-\frac{1}{p})\overline{\ex_p(x)} \chi_{p^{-n_p}\Z_p}(x).
\end{align*}
Im Fall $\nu=\infty$ und $\delta=0$ gilt
\begin{align*}
\hat{f}_\infty=f_\infty=(-\mathrm{i})^\delta f_{\infty}.
\end{align*}
Für $\delta=1$ gilt
\begin{align*}
\hat{f}_{\infty}(x)=\int_\R y \ex^{-\pi y^2-2\pi \mathrm{i}xy}\mathrm{d}y=(-\mathrm{i})^\delta f_\infty(x).
\end{align*}
Wir berechnen nun das $\zeta$-Integral
\begin{align*}
\zeta(\hat{f},\overline{\omega},1-s)=\prod_{\nu \leq \infty} \int_{\Q_\nu^\ast} f_\nu(x)\overline{\omega}_nu(x)\abs{x_\nu}_\nu^{1-s}~\mathrm{d}^\ast x_\nu.
\end{align*}
Für $p \nmid N$ gilt
\begin{align*}
\zeta_p(\hat{f}_p,\overline{\omega},1-s)=\frac{1}{1-\chi(p)p^{1-s}}.
\end{align*}
Für $p\mid N$ gilt
\end{proof}