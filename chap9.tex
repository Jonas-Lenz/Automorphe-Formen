\chapter{Dirichletsche $L$-Reihe}
Sei $\chi \colon (\Z/N\Z)^\ast \to T$ ein Dirichletcharakter modulo $N$.
Wir setzen $\chi$ zu einer Funktion auf $\Z$ durch
\begin{align*}
\chi(a)=\begin{cases}
\chi(a \mod N),~~&(a,N)=1,\\
0, &(a,N)>1.
\end{cases}
\end{align*}
fort und definieren die \emph{$L$-Reihe} von $\chi$ durch
\begin{align*}
L(\chi,s)\coloneqq \sum_{n=1} \frac{\chi(n)}{n^s}.
\end{align*}
Diese Reihe konvergiert für $\Ret(s)>1$ und definiert dort eine holomorphe Funktion.
Es gilt
\begin{align*}
L(\chi,s)=\prod_{p>0} \frac{1}{1-\chi(p)p^{-s}}.
\end{align*}

Erinnerung: Sei $\omega$ ein endlicher Charakter von $\A^\ast/\Q^\ast$.
Wir betrachten $\omega$ als Charakter auf $\A^\ast$.
Dann sind fast alle $\omega_p$ unverzweigt, das heißt $\Z_p^\ast \subseteq \ker(\omega_p)$.
Falls $\omega_p$ verzweigt, so existiert ein minimales $\Z\ni m_p >0$ mit
\begin{align*}
1+K(0,p^{-m_p})\subseteq \ker(\omega_p).
\end{align*}
Der Führer von $\omega$ ist durch $N=\prod_{p \text{verzweigt}} p^{m_p}$ definiert.
Dann definiert $\omega$ durch $\chi(p)=\overline{\omega_p(p)}$ für $p\nmid N$ einen primitiven Dirichletcharakter modulo $N$.
Wenn wir $\chi$ zu einem Charakter auf $\A^\ast/\Q^\ast$ liften, erhalten wir $\omega$ zurück.

\begin{defi}
Sei $\omega$ ein endlicher Charakter auf $\A^\ast/\Q^\ast$ und $f \in \mathcal{S}(A)$.
Wir definieren die twisted $\zeta$-Funktion
\begin{align*}
\zeta(f,\omega,s)\coloneqq \int_{\A^\ast} f(x)\omega(x)\abs{x}^s~\mathrm{d}^\ast x.
\end{align*}
\end{defi}
%TODO wie heißt twisted auf deutsch?!
Wie im letzten Kapitel kann man zeigen, dass das Integral lokal gleichmäßig für $\Ret(s)>1$ konvergiert und dort eine holomorphe Funktion definiert.
Ebenso gilt
\begin{align*}
\zeta(f,\omega,s)=\int_{\A^\ast/\Q^\ast} E(f)(x)\omega(x)\abs{x}^s~\mathrm{d}^\ast x,
\end{align*}
wobei $E(f)$ wie im letzten Kapitel definiert ist (siehe Definition 8.3).

Als Vorbereitungen benötigen wir folgendes Lemma.
\begin{lem}
Sei $K$ eine kompakte Hausdorffgruppe mit Haarmaß $\mathrm{d}x$ und $\chi \colon K\to T$ ein Charakter. Dann gilt
\begin{align*}
\int_K \chi(x)~\mathrm{d}x=\begin{cases}
\text{vol}(K),~~&\text{falls } \chi \text{ trivial ist},\\
0,&\text{sonst.}
\end{cases}
\end{align*}
\end{lem}
%besondere: man kann L-reihen für viele verschiedene objekte definieren, aber diese sind immer gleich
%für modulare Formen hat dies Wiles vor nicht allzulanger Zeit gesagt und ist allgemeiner Teil des Langland Programms.