\chapter{Hecke-Petersson Theorie}
%definiere inneres produkt auf cusp 
%die c\alpha\beta heißen strukturkonstanten
Diese Theorie gibt eine natürliche Erklärung der Multiplikativität
der Koeffizienten von
\begin{align*}
\Delta(\tau)&=q \prod_{n=1}^\infty (1-q^n)^{24}\\
&=q-24q^2+252q^3-1472q^4+4830q^5-6048q^6
\end{align*}

%define hecke algebra dafür brauchen wir folgendes

\begin{defi}
Für $\Z \ni N>0$ definiere 
\begin{align*}
\Gamma(N)=\{\begin{pmatrix}
a&b\\
c&d
\end{pmatrix} \in \SL_2(\Z) \mid \begin{pmatrix}
a&b\\
c&d
\end{pmatrix} =\begin{pmatrix}
1&0\\
0&1
\end{pmatrix} \mod N\}.
\end{align*}
$\Gamma(N)$ ist eine normale Untergruppe von $\SL_2(\Z)$ und heißt
\emph{Hauptkongruenzuntergruppe} vom Level $N$.
Sie hat endlichen Index, tatsächlich gilt sogar
\begin{align*}
\abs{\SL_2(Z)/\Gamma(N)} =N^3 \prod_{p \mid N} (1-\frac{1}{p^2}).
\end{align*}
Eine Untergruppe $\Gamma \subseteq \SL_2(Z)$ heißt \emph{Kongruenzuntergruppe},
falls $\Gamma$ die Gruppe $\Gamma(N)$ für ein $N \in \N$ enthält.
\end{defi}

\begin{bsp}
Die beiden wichtigsten Beispiele für Kongruenzuntergruppen sind $\Gamma_0(N)$ %c\not =0
und $\Gamma_1(N)$ %c=o mod N und b egal
\end{bsp}

Wir schreiben $\Sigma=\GL_2(\Q)^+$ für Matrizen mit positiver Determinante.

\begin{prop}
Sei $\Gamma$ eine Kongruenzuntergruppe und $\alpha \in \Sigma$.
Dann existiert ein $M \in \N$ mit
\begin{align*}
\Gamma(M)\subseteq \alpha \Gamma \alpha^{-1}.
\end{align*}
\end{prop}
\begin{proof}
Sei $\Gamma(N)\subseteq \Gamma$.
Wähle $M_1,M_2 \in \Z$, sodass $M_1\alpha,M_2\alpha^{-1} \in M_2(\Z)$ gilt.
%ganzzahlige Einträge
Definiere $M=M_1M_2N$.
%aufpassen, dass M\in \N!
Dann können wir $\gamma \in \Gamma(N)$ schreiben als
\begin{align*}
\gamma=1+Mg
\end{align*}
mit einem $g \in M_2(\Z)$.
Dann gilt
\begin{align*}
\alpha \gamma \alpha^{-1} &=1+M_1M_2N \alpha g\alpha^{-1}\\
&=1+N(M_1,\alpha)g (M_2\alpha^{-1})\\
&\in \Gamma(N).
\end{align*}
Damit folgt die Behauptung, wenn man $\alpha$ durch $\alpha^{-1}$ ersetzt.
\end{proof}

%Hecke algebra mit Doppelnebenklassen

\begin{prop}
Sei $\alpha \in \Sigma$ und $\Gamma$ eine Kongruenzuntergruppe. Dann existiert eine Menge von Repräsentanten $(\gamma_i)$ von
$(\Gamma \cap \alpha^{-1} \Gamma \alpha)\backslash \Gamma$ sodass
\begin{align*}
\Gamma \alpha \Gamma =\bigcup_{i} \Gamma \alpha \gamma_i
\end{align*}
als disjunkte Vereinigung gilt.
Insbesondere ist die Menge $(\gamma_i)$ endlich.
\end{prop}
\begin{proof}
Seien $\gamma,\gamma'\in \Gamma$.
Dann gilt
\begin{align*}
\Gamma \alpha \gamma =\Gamma\alpha \gamma'&\Leftrightarrow \alpha \gamma' \gamma^{-1} \alpha^{-1} \Gamma\\
&\Leftrightarrow \gamma'\gamma^{-1} \in \alpha^{-1} \Gamma \alpha\\
&\Leftrightarrow \gamma' \in (\alpha^{-1} \Gamma \alpha \cap \Gamma)\gamma.
\end{align*}
Dies zeigt die Behauptung.
\end{proof}
%endlichkeit folgt mit argumenten, dass man \Gamma(N)s enthält.

Ein analoges Resultat erhält man für Linksadjunktion.
\begin{bem}
Mit den Voraussetzungen wie eben können wir eine Menge $(\beta_i)$
von Repräsentanten von $\Gamma/(\Gamma \cap \alpha^{-1} \Gamma \alpha)$ finden, die
\begin{align*}
\Gamma \alpha \Gamma)\bigcup \beta_i \alpha \Gamma
\end{align*}
als disjunkte Vereinigung erfüllt.
\end{bem}

Nun haben wir alles beisammen, um die Hecke Algebra definieren zu können.
\begin{defi}
Für $\alpha \in \Sigma$ definieren wir $T_\alpha=\Gamma \alpha \Gamma \in \Gamma \backslash \Sigma /\Gamma$
und die Hecke Algebra
\begin{align*}
H_\Gamma = \{ \sum_{\alpha \in \Gamma \backslash \Sigma /\Gamma} c_\alpha T_\alpha \mid c_\alpha in \Z\}
\end{align*}
wobei alles bis auf endliche $c_\alpha$ verschwinden.
$H_\Gamma$ ist der freie $\Z$-Modul, der durch die $T_\alpha$ erzeugt wird.
\end{defi}

Für $\alpha,\beta \in \Sigma$ gilt
\begin{align*}
(\Gamma \alpha \Gamma)(\Gamma \beta
\Gamma)&=(\cup_i \Gamma \alpha_i)(\cup_j \beta_j \Gamma\\
&=\cup_{i,j} \Gamma \alpha_i \beta_j \Gamma,
\end{align*}
so dass $(\Gamma \alpha \Gamma)(\Gamma \beta \Gamma)$
die endliche Vereinigung von Doppelnebenklassen ist.
Daher können wir $H_\Gamma$ zu einer Algebra machen, in dem wir
\begin{align*}
T_\alpha T_\beta \coloneqq \sum c_{\alpha \beta}^\gamma T_\gamma
\end{align*}
definieren, wobei wir über alle $\gamma \in \Gamma \backslash \Sigma /\Gamma$ summieren, die
\begin{align*}
\Gamma \gamma \Gamma \subseteq (\Gamma \alpha \Gamma)(\Gamma \beta
\Gamma)
\end{align*}
erfüllen.
Die Koeffizienten sind definiert durch
\begin{align*}
c_{\alpha\beta} =\abs{\{(i,j)\mit \Gamma \alpha_i \beta_j=\Gamma \gamma\}}
\end{align*}
wobei die $\alpha_i,\beta_j$ aus den disjunkten Vereinigungen von $\Gamma \alpha \Gamma$ kommen.
Es ist a priori nicht klar, ob dieses Produkt wohldefiniert ist.
Wir zeigen dies nun. Es gilt
\begin{align*}
\abs{\{(i,j)\mid \Gamma \alpha_i \beta_j =\Gamma \gamma\}} &=\abs{\{(i,j)\mid \Gamma \alpha_i =\Gamma \gamma \beta_j^{-1}\}}\\
&=\abs{\{j\mid \Gamma \alpha \Gamma \supseteq \Gamma \gamma\ \beta_j^{-1}\}}\\
&=\abs{\{j \mid \gamma \beta_j^{-1} \in \Gamma \alpha \Gamma\}}\\
&=\abs{\{j \mid \beta_j \in \Gamma \alpha^{-1} \Gamma \gamma\}}\\
&=\abs{\{j \mid \Gamma \beta_j \subseteq \Gamma \alpha^{-1} \Gamma \gamma\}}.
\end{align*}
Dies ist die Anzahl der Nebenklassen der Form $\Gamma \varepsilon$ in $\Gamma \alpha^{-1} \Gamma \gamma \cap \Gamma \beta \Gamma$.
Diese letzte Zahl ist unabhängig von der Wahl der $\alpha_i,\beta_j$.
Sei nun $\Gamma \gamma\Gamma=\Gamma \eta \Gamma$.
Dann gilt $\gamma=\delta'\\delta S$ für geeignete $S,S'\in \Gamma$.
Ebenso gilt
\begin{align*}
\Gamma \alpha^{-1} \Gamma \gamma \cap \Gamma \beta \Gamma&=\Gamma \alpha^{-1} \Gamma \eta \delta \cap \Gamma\beta \Gamma\\
&=(\Gamma \alpha^{-1} \Gamma \eta \cap \Gamma \beta \Gamma)\delta.
\end{align*}
Dies impliziert, dass Multiplikation mit $\delta^{-1}$ Nebenklassen der Form $\Gamma \varepsilon$ in $\Gamma \alpha^{-1} \Gamma \gamma\cap \Gamma \beta \Gamma$ in Nebenklassen der Form $\Gamma \rho$ in $\Gamma \alpha^{-1}\Gamma \eta \cap \Gamma \beta \Gamma$.
Also haben wir die Unabhängigkeit der Anzahl von $\gamma$ gezeigt.

\begin{defi}
Zusammen mit diesem Produkt heißt $H_\Gamma$ die zu $\Gamma$ gehörende \emph{Hecke Algebra}.
\end{defi}

Falls $\Gamma \gamma \Gamma=\cup_{k=1}^l \Gamma \gamma_k$ als disjunkte Vereinigung gilt,
so folgt
\begin{align*}
\abs{\{(i,j)\mid \Gamma \alpha_i \beta_j \Gamma=\Gamma \gamma \Gamma\}}=l \abs{\{(i,j)\mid \Gamma \alpha_i \beta_j =\Gamma \gamma\}}.
\end{align*}

%important theorem dont know much about product
\begin{thm}
Die Hecke Algebra $H_\Gamma$ ist assoziativ.
\end{thm}
\begin{proof}
Ein Beweis kann in dem Buch von Shimura über automorphe Formen gefunden werden
\end{proof}
%TODO buch raussuchen

Wir betrachten nun den Fall $\Gamma =\SL_2(\Z)$.
\begin{prop}
Sei $\alpha \in \Sigma$.
Dann enthält $\Gamma \alpha \Gamma$ eine eindeutige Matrix der Form
\begin{align*}
\begin{pmatrix}
d_1&0\\
0&d_2
\end{pmatrix}
\end{align*}
wobei $d_1,d_2 \in \Q^+$ und $\frac{d_1}{d_2}\in \Z$ erfüllt seien.
\end{prop}
\begin{proof}
Wähle $\Z \ni N>0$ so, dass $N\alpha$ ganzzahlig ist.
Seien $a_1,a_2$ die Spalten von $N\alpha$ und $e_1,e_2$ die Standardbasisvektoren von $\R^2$.
Definiere
\begin{align*}
\Lambda_1 &=\Z e_1+\Z e_2\\
\Lambda_2 &=\Z a_1+\Z a_2\subseteq \Lambda_1.
\end{align*}
Aus dem Elementarteilersatz folgt die Existenz einer Basis $(\xi_1,\xi_2\}$ von $\Lambda_1$ sowie die von positiven Zahlen $D_1,D_2$
$D_2 \mid D_1$, sodass $\{D_1\xi_1,D_2\xi_2\}$ eine $\Z$-Basis von $\Lambda_2$ ist.
Sei $\xi$ die Matrix mit Spalten $\xi_1,\xi_2$.
Da $\Lambda_1$ unimodular ist, folgt
\begin{align*}
\det(\xi)^2=\det(\xi^T \xi)=1.
\end{align*}
Durch Umorientieren können wir $\det(\xi)=1$, also $\xi \in \Gamma$ annehmen.
Aus
\begin{align*}
\Lambda_2&=\Z a_1+\Z a_2\\
&=\Z D_1 \xi_1+\Z D_2 \xi_2
\end{align*}
folgt
\begin{align*}
N\alpha=\xi \begin{pmatrix}
D_1&0\\
0&D_2
\end{pmatrix} \gamma
\end{align*}
für ein geeignetes $\gamma \in \GL_2(\Z)$.
Wie zuvor kann man nun $\det(\gamma)=1$ also $\gamma \in \Gamma$ sehen.
Teilen durch $N$ liefert die Existenz, die Eindeutig verbleibt als Übung.
\end{proof}

%schönes aber sehr technisches Objekt.
%Machen nicht alle beweise, da dies sehr zeitraubend wäre.

\begin{prop}
Sei $\gamma \in \Sigma$.
Dann gibt es $\delta_1,\dots,\delta_n\in \Sigma$ mit
\begin{align*}
\Gamma \gamma \Gamma =\cup_{i=1}^n \Gamma \delta_i=\cup_{i=1}^n \delta_i \Gamma
\end{align*}
wobei die Vereinigungen disjunkt sind.
\end{prop}
\begin{proof}
Wähle eine disjunkte Zerlegung $\Gamma \gamma \Gamma = \cup_{i=1}^n \Gamma \gamma_i$.
Mit der letzten Proposition folgt dann
\begin{align*}
\Gamma \gamma \Gamma=\left(\Gamma \gamma \Gamma\right)^T=\cup_{i=1}^n \gamma_i^T \Gamma.
\end{align*}
Es gilt
\begin{align*}
\Gamma \gamma_i \cap \gamma_j^T \Gamma \not =\emptyset
\end{align*}
für alle $i,j$.
Angenommen, dies wäre falsch, dann gilt mit $\Gamma \gamma_i \subseteq \Gamma \gamma_i \Gamma=\Gamma \gamma \Gamma=\cup_{j=1}^n \gamma_j^T \Gamma$, dass insbesondere
\begin{align*}
\Gamma \gamma_i \subseteq \cup{j\not =i} \gamma_j^T \Gamma
\end{align*}
gilt.
Durch Multiplikation von rechts mit $\Gamma$ folgt
\begin{align*}
\Gamma \gamma \Gamma =\Gamma \gamma_i \Gamma \subseteq \cup_{j\not =i} \gamma_j^T \Gamma
\end{align*}
was ein Widerspruch ist, da die Nebenklassen paarweise disjunkt sind.
Wähle nun $\delta_i \in \Gamma \gamma_i \cap \gamma_i^T \Gamma$.
Dann gilt $\Gamma \delta_i =\Gamma \gamma_i$ und $\delta_i \Gamma =\gamma_i^T \Gamma$,
was uns die gewünschte Zerlegung gibt.
\end{proof}

Damit können wir nun die Kommutativität der Multiplikation zeigen.

\begin{prop}
Seien $\alpha,\beta \in \Sigma$. Dann gilt
\begin{align*}
(\Gamma \alpha \Gamma)(\Gamma \beta \Gamma)=(\Gamma \beta \Gamma)(\Gamma \alpha \Gamma).
\end{align*}
\end{prop}
\begin{proof}
Wir wählen zunächst eine beidseitige disjunkte Zerlegung, das heißt
\begin{align*}
\Gamma \alpha \Gamma &=\cup_{i=1}^n \Gamma \alpha_i=\cup_{i=1}^n \alpha_i \Gamma\\
\Gamma \beta \Gamma &=\cup_{i=1}^n \Gamma \beta_i=\cup_{i=1}^n \beta_i \Gamma.
\end{align*}
Damit gilt
\begin{align*}
(\Gamma \alpha \Gamma)(\Gamma \beta \Gamma)&=\cup_{i,j} \Gamma \alpha_i^T \beta_j^T \Gamma\\
&=\cup_{i,j} (\Gamma \beta_j \alpha_i \Gamma)^T\\
&=\cup_{i,j} (\Gamma \beta_j \alpha_i \Gamma)\\
&=(\Gamma \beta \Gamma)(\Gamma \alpha \Gamma). \qedhere
\end{align*}
\end{proof}

Nun können wir die Kommutativität der Hecke Algebra zeigen.

\begin{thm}
Die Hecke Algebra $H_\Gamma$ ist kommutativ.
\end{thm}
\begin{proof}
Seien $\alpha, \beta \in \Sigma$ und wir wählen beidseite disjunkte Zerlegungen
\begin{align*}
\Gamma \alpha \Gamma &=\cup_{i=1}^n \Gamma \alpha_i=\cup_{i=1}^n \alpha_i \Gamma\\
\Gamma \beta \Gamma &=\cup_{i=1}^n \Gamma \beta_i=\cup_{i=1}^n \beta_i \Gamma.
\end{align*}
Aufgrund der letzten Proposition gilt
\begin{align*}
(\Gamma \alpha \Gamma)(\Gamma \beta \Gamma)=(\Gamma \beta \Gamma)(\Gamma \alpha \Gamma).
\end{align*}
Sei nun $\gamma \in \Sigma$ mit
\begin{align*}
\Gamma \gamma \Gamma\subseteq (\Gamma \alpha \Gamma)(\Gamma \beta \Gamma)
\end{align*}
und der disjunkten Zerlegung
\begin{align*}
\Gamma \gamma \Gamma =\cup_{k=1}^l \Gamma \gamma_k.
\end{align*}
Wir müssen nun zeigen, dass $c_{\alpha \beta}^\gamma$ symmetrisch in $\alpha$ und $\beta$ ist.
Es gilt (mit Proposition 14.9)
\begin{align*}
c_{\alpha \beta}^\gamma &=\abs{\{(i,j)\mid \Gamma \alpha_i \beta_j=\Gamma \gamma\}}\\
&=\frac{1}{l}\abs{\{(i,j)\mid \Gamma \alpha_i \beta_j \Gamma=\Gamma \gamma \Gamma\}}\\
&=\frac{1}{l}\abs{\{(i,j)\mid \Gamma \beta_j^T \alpha_i^T\Gamma =\Gamma \gamma \Gamma\}}\\
&=\abs{\{(i,j)\mid \Gamma \beta_j^T \alpha_i^T=\Gamma \gamma\}}\\
&=c_{\beta \alpha}^\gamma.
\end{align*}
Damit folgt die Behauptung.
\end{proof}

Als nächstes wollen wir die Hecke Algebra auf Modulformen operieren lassen.
\begin{prop}
Sei $f \in M_k$ und $\alpha \in \Sigma$.
Wähle eine disjunkte Zerlegung
\begin{align*}
\Gamma \alpha \Gamma=\cup_{i=1}^n \Gamma \alpha_i.
\end{align*}
Dann ist
\begin{align*}
f\mid_{T_\alpha}=\det(\alpha)^{\frac{k}{2}-1} \sum_{i=1}^n f \mid_{\alpha_i}
\end{align*}
wohldefiniert.
\end{prop}
\begin{proof}
Wähle eine andere disjunkte Zerlegung $\Gamma \alpha \Gamma=\cup_{j=1}^n \beta_j$.
Nach möglicher Umnummerierung gilt $\beta_i=M \alpha_i$ für geeignete $M_i \in \Gamma$.
Daraus folgt
\begin{align*}
f\mid_{\beta_i}=f\mid_{M_i \alpha_i} =(f \mid_{M_i})\mid_{\alpha_i}=f\mid_{\alpha_i}.
\end{align*}
%TODO warum stimmen die gleichheiten, vor allem die letzte?
Also ist die Definition unabhängig von der Zerlegung.
\end{proof}

Das nächste Ziel wird sein, zu zeigen, dass dies wieder eine Modulform definiert.
Dafür benötigen wir folgende Vorbereitung.
\begin{prop}
Sei $\alpha \in \Sigma$. Dann gilt $\alpha \gamma \alpha'$ mit
$\gamma \in \Gamma$ und $\alpha'=r \begin{pmatrix}
a&b\\
0&d
\end{pmatrix}$
wobei $r \in \Q^+,a,b,d\in \Z$ mit $(a,b,d)=1$ sowie $a,d>0$.
\end{prop}
\begin{proof}
Sei $\alpha =\begin{pmatrix}
x&y\\
s&t
\end{pmatrix}$.
Es genügt zu zeigen, dass
\begin{align*}
\gamma \alpha =\begin{pmatrix}
\ast&\ast\\
0&\ast
\end{pmatrix}
\end{align*}
für ein geeignetes $\gamma \in \Gamma$ gilt.
Dies ist für $s=0$ klar.
Für $s\not =0$ wählen wir $c',d'\in \Z$ mit $d'=-c' \frac{x}{s}$.
Setze $c=\frac{c'}{(c',d')}$ und $d=\frac{d'}{(c',d')}$ und ergänze $c,d$ zu einer Matrix $\gamma=\begin{pmatrix}
a&b\\
c&d
\end{pmatrix}\in \Gamma$.
\end{proof}

\begin{prop}
Sei $\alpha \in \Sigma$ und $f \in M_k$.
Dann gilt $f\mid_{T_\alpha} \in M_k$ und ebenso für $S_k$.
\end{prop}
\begin{proof}
Sei $f \in M_k$.
Zunächst zeigen wir, dass $f \mid_{T_\alpha}$ invariant unter $\Gamma$ ist.
Wähle die disjunkte Zerlegung $\Gamma \alpha \Gamma=\cup_{i} \Gamma \alpha_i$ und sei $M \in \Gamma$.
Dann gilt
\begin{align*}
\Gamma \alpha \Gamma M=\cup_i \Gamma \alpha_i M
\end{align*}
sodass
\begin{align*}
(f \mid_{T_\alpha})\mid_M &=(\det(\alpha)^{\frac{k}{2}-1} \sum f\mid_{\alpha_i})\mid_M\\
&=\det(a)^{\frac{k}{2}-1} \sum f\mid_{\alpha_i M}\\
&=f \mid_{T_\alpha}
\end{align*}
gilt.
Aus der Konstruktion ist klar, dass $f\mid_{T_\alpha}$ holomorph auf $H$ ist.
Es bleibt zu zeigen, dass $f\mid_{T_\alpha}$ holomorph im unendlichen ist.
Sei $\alpha \in \Sigma$. Die vorherige Proposition impliziert $\alpha=\gamma \alpha'$
mit $\gamma \in \Gamma$ und $\alpha'=r \begin{pmatrix}
a&b\\
0&d
\end{pmatrix}$
mit geeigneten $r,a,b,d$.
Dann folgt
\begin{align*}
f\mid_{\alpha} (\tau)&=f\mid_{\gamma \alpha'}(\tau)=f\mid_{\alpha'}(\tau)\\
&=f\mid_{\begin{pmatrix}
a&b\\
0&d
\end{pmatrix}} =\left(\frac{a}{d}\right)^{\frac{k}{2}} f\left( \frac{a \tau+b}{d}\right).
\end{align*}
Da $f$ holomorph im unendlichen ist, konvergiert $f(\tau)$ für $\tau \to \mathrm{i}\infty$ und somit auch $f\left(\frac{a\tau+d}{d}\right)$.
Daraus folgt $f\mid_{T_\alpha} \in M_k$.
Der Beweis für $S_k$ funktioniert analog.
\end{proof}

\begin{prop}
Seien $\alpha,\beta \in \Sigma$ und $f \in M_k$.
Dann gilt
\begin{align*}
(f \mid_{T_\alpha}\mid_{T_\beta} =f\mid_{T_\alpha T_\beta}.
\end{align*}
\end{prop}

\begin{prop}
Das Maß
\begin{align*}
\mathrm{d}\mu(\tau)=\frac{\mathrm{d}x\mathrm{d}y}{y^2}, \tau=x+\mathrm{i}y \in H
\end{align*}
ist invariant unter $\GL_2(\R^+)$.
\end{prop}

\begin{defi}
Ein offene Menge $F \subseteq H$ heißt \emph{Fundamentalbereich von $\Gamma$}, falls eine Menge von Repräsentanten von $\Gamma\backslash H$ mit
$F \subset R \subset \overline{F}$ und $\mu(\partial F)=0$ existiert.
Wir definieren $\mu(\Gamma\backslash H)\coloneqq \mu(F)$, wobei $F$ ein beliebiger Fundamentalbereich von $\Gamma$ ist.
Es gilt
\begin{align*}
\mu(\Gamma \backslash H)=\frac{\pi}{3}.
\end{align*} 
Für $f,g\in S_k$ definieren wir
\begin{align*}
(f,g)&=\int_{\Gamma \backslash H} f(\tau)\overline{g(\tau)} \Imt(\tau)^k ~\mathrm{d}\mu(\tau)\\
&=\int_F f(\tau)=\overline{g}(\tau)\Imt(\tau)^k~\mathrm{d}\mu(\tau),
\end{align*}
wobei $F$ ein beliebiger Fundamentalbereich von $\Gamma$ ist.
\end{defi}

\begin{thm}
$(\cdot,\cdot)$ definiert ein inneres Produkt auf $S_k$.
\end{thm}

\begin{thm}
Der Heckeoperator $T_\alpha$ für $\alpha \in \Sigma$ ist bezüglich $(\cdot,\cdot)$ selbstadjungiert.
\end{thm}
\begin{proof}
z.b. in [Automorphe Formen] von Anton Deitmar.
\end{proof}

Definiere $\Delta_n=\{\alpha \in M_2(\Z)\mid \det(\alpha)=n\}$ sowie
$T_n=\sum_{\alpha \in \Gamma \backslash \Delta_n /\Gamma} T_\alpha$.
Indem wir $\Delta_n=\cup \Gamma \gamma_i$ schreiben, gilt
\begin{align*}
f\mid_{T_n} =n^{\frac{k}{2}-1} \sum f\mid_{\gamma_i},
\end{align*}
also
\begin{align*}
f\mid_{T_n}=n^{\frac{k}{2}-1} \sum_{\gamma \in \Gamma \backslash \Delta_n} f\mid_\gamma.
\end{align*}

Solange wir keine explizite Beschreibung der Menge haben, über die wir summieren, ist die Formel noch nicht sehr nützlich.
Es gilt $\Delta_n=\cup_{\substack{ad=n \\ b \mod d}} \Gamma(\begin{pmatrix}
a&b\\
0&d
\end{pmatrix})$.
%b mod d heißt b=0,\dots,d-1
Damit erhalten wir
\begin{align*}
f\mid_{T_n}(\tau)=n^{k-1} \sum_{\substack{ad=n\\ b \mod d}} d^{-k} f\left( \frac{a\tau+b}{d}\right).
\end{align*}
Als nächstes betrachten wir die Fourierkoeffizienten der neuen Modulform.

\begin{prop}
Sei $f(\tau)=\sum_{m=0}^\infty c_mq^m \in M_k$.
Dann gilt $f\mid_{T_n}(\tau)=\sum_{m=0}^\infty b_mq^m \in M_k$,
wobei die Koeffizienten $b_m$ durch
\begin{align*}
b_m=\sum_{d \mid (m,n)} d^{k-1} c_{\frac{mn}{d^2}}
\end{align*}
gegeben sind.
Insbesondere gilt $b_0=c_0\sigma_{k-1}(n)$ und $b_1=c_n$.
\end{prop}
\begin{proof}
Es gilt
\begin{align*}
f\mid_{T_n}(\tau)&=n^{k-1} \sum_{ad=n} \sum_{b\mod d} d^{-k} f\left(\frac{a \tau+b}{d}\right)\\
&=n^{k-1} \sum_{ad=n}\sum_b d^{-k} \sum_{m=0}^\infty c_m \mathrm{e}^{2 \pi \mathrm{i} m \frac{a\tau +b}{d}}\\
&=n^{k-1} \sum_{ad=n} \sum_m c_m \mathrm{e}^{2\pi \mathrm{i} m a \frac{\tau}{d}} \sum_{b \mod d} \mathrm{e}^{2\pi \mathrm{i} m  \frac{b}{d}}\\
&=\sum_{ad=n} \left(\frac{n}{d}\right)^{k-1} \sum_{m'=0}^\infty c{dm'} q^{am'}
\end{align*}
was durch umschreiben die gewünschte Formel ergibt.
%im vorletzen schritt braucht man wie sich gauss summen verhalten.
\end{proof}

%TODO muss weiter nach oben! Dies ist alter Ansatz von Hecke und vorher ist von Shimura, welcher besser ist.

%folgende formel erklärt angesprochene multiplikativität
\begin{prop}
Sei $f(\tau)=\sum_{m=0}^\infty c_mq^m \in M_k\setminus \{0\}$ für $k>0$ eine "simultane"
%TODO fixen
Eigenform für alle Heckeoperatoren, das heißt
\begin{align*}
f\mid_{T_n}=\lambda_n f
\end{align*}
für geeignete $\lambda_n \in \C$ für alle $n>0$.
Dann gilt
\begin{enumerate}
\item $c_1\not =0$,
\item $c_n=\lambda_n c_1,n>0$,
\item $c_mc_n=c_1\sum_{d\mid (m,n)} d^{k-1} c_{\frac{mn}{d^2}}$. 
\end{enumerate}
Insbesondere gilt
$c_mc_n=c_1c_{mn}$ falls $(m,n)=1$.
\end{prop}
\begin{proof}
%T_a sind selbstadjungiert, kommutieren miteinander T_n sind auch selbstadjungiert, LA: simultan diagonalisierbar und dann machen wir eine aussage über eine nicht leere Menge.
Es gilt $f\mid_{T_n}(\tau)=\sum_{m=0}^\infty b_mq^m$ mit
\begin{align*}
b_m=\sum_{d \mid (m,n)} d^{k-1} c_{\frac{mn}{d^2}}.
\end{align*}
Dann impliziert $f\mid_{T_n}=\lambda_n f$ dass
\begin{align*}
\lambda_n c_m=\sum_{d\mid (m,n)} d^{k-1} c_{\frac{mn}{d^2}}
\end{align*}
gilt.
Daraus können wir bereits alles folgern.
Setzen wir $m=1$ so erhalten wir $\lambda=nc_1=c_n$.
Da wir den Grad positiv gewählt haben, folgt $c_1\not =0$ und somit die ersten beiden Aussagen.
Die letzte Aussage folgt, wenn man die untere in die obere Formel einsetzt.
\end{proof}

Wir können den letzten Satz nun auf $S_{12}$ und $\Delta$ anwenden.

\begin{bsp}
Da $S_{12}=\C\Delta$ gilt, wissen wir eine Eigenform aller $T_n$ ist.
Da es normalisiert ist $(c_1=1)$, sind die Fourierkoeffizienten multiplikativ.
\end{bsp}

\begin{prop}
In $H_\Gamma$ gelten die folgenden Beziehungen.
\begin{enumerate}
\item $T_n T_m=T_m=T_n=T{mn}$ falls $(m,n)=1$,
\item $T_p T_{p^n}=T_{p^{n+1}}+pT_{(p,p)}T_{p^{n-1}}$ falls $p$ prim ist.
%T_{(p,p)} heißt T_{\begin{pmatrix}
%p&0\\
%0&p
%\end{pmatrix}}
\end{enumerate}
\end{prop}
\begin{proof}
Shimura, p.63.
\end{proof}

\begin{bem}
Für $f\in M_k$ gilt
\begin{align*}
f\mid_{pT_{(p,p)}}=p f \mid_{T_{(p,p)}}=p^{k-1} f.
\end{align*}
Für $\sigma_k(m)=\sum_{d\mid m}d^k$ gilt
\begin{align*}
\sigma_k(m)\sigma_k(n)=\sum_{d \mid (m,n)} d^k \sigma_k\left(\frac{mn}{d^2}\right).
\end{align*}
\end{bem}

\begin{prop}
Sei $k\geq 4$ gerade.
Dann gilt
\begin{align*}
E_k \mid_{T_n}=\sigma_{k-1}(n)E_k
\end{align*}
für alle $n\geq 1$.
\end{prop}
\begin{proof}
Es gilt
\begin{align*}
E_k(\tau)=1-\frac{2k}{B_k}\sum_{m=1}^\infty \sigma_{k-1}(m)q^m
\end{align*}
und
\begin{align*}
E_k\mid_{T_n} (\tau)=\sum_{m=0}^\infty b_mq^m
\end{align*}
mit
\begin{align*}
b_0&=\sigma_{k-1}(n)\\
b_m&=\sum_{d\mid(m,n)}d^{k-1}c_{\frac{mn}{d^2}}\\
&=-\frac{2k}{B_k} \sum_{d \mid (m,n)} d^{k-1} \sigma_{k-1}\left(\frac{mn}{d^2}\right)\\
&=-\frac{2k}{B_k} \sigma_{k-1}(m)\sigma_{k-1}(n)\\
&=\sigma_{k-1}(n)c_m,
\end{align*}
woraus die Behauptung folgt.
\end{proof}

\begin{prop}
Sei $f\in S_k$ eine Eigenfunktion, also $f\not \equiv 0$ mit $f\mid_{T_n}=\lambda_n f$ für geeignete $\lambda_n \in \C$.
Dann gilt
\begin{align*}
\abs{\lambda_n}\leq n^{\frac{k}{2}-1}\sigma_1(n).
\end{align*}
\end{prop}
\begin{proof}
Wir wissen bereits, dass die Funktion $h(\tau)\coloneqq \abs{f(\tau)} y^{\frac{k}{2}}$ beschränkt in $H$ ist.
%TODO verweis suchen.
Insbesondere existiert ein $\nu \in H$ mit $h(\tau)\leq h(\nu)$.
Dann gilt
\begin{align*}
f\mid_{T_n}(\tau)=n^{k-1} \sum_{\substack{ad=n \\ b \mod d}}d^{-k} f\left(\frac{a\tau+b}{d}\right)
\end{align*}
sodass
\begin{align*}
\lambda_n f(\tau)=n^{k-1} \sum_{\substack{ad=n \\ b \mod d}}d^{-k} f\left(\frac{a\tau+b}{d}\right)
\end{align*}
folgt.
Damit gilt
\begin{align*}
\abs{\lambda_n f(\tau)y^{\frac{k}{2}}} =n^{k-1} \abs{\sum_{\substack{ad=n \\ b \mod d}}d^{-k} f\left(\frac{a\tau+b}{d}\right) y^{\frac{k}{2}}}.
\end{align*}
Mit der Dreiecksungleichung folgt dann
\begin{align*}
\abs{\lambda_n} \abs{f(\tau)}y^{\frac{k}{2}} \leq n^{k-1} \sum_{\frac{1}{(ad)^{\frac{k}{2}}}} \abs{f\left(\frac{a\tau+b}{d}\right)} \left(\frac{ay}{d}\right)^{\frac{k}{2}}.
\end{align*}
Wenn wir $\tau=\nu$ wählen, erhalten wir
\begin{align*}
\abs{\lambda_n} h(\nu)\leq n^{k-1} \frac{1}{n^{\frac{k}{2}}} h(\nu) \sum_{ad=n} \sum_{b\mod d} 1.
\end{align*}
Teilen durch $h(\nu)$ liefert dann die Behauptung.
\end{proof}

\begin{prop}
Sei $k\geq 4$ und $f(\tau)=\sum_{m=0}^\infty c_mq^m\in M_k$ mit $c_0=1$.
Falls $f\mid_{T_n}=\lambda_n f$ für geeignete $\lambda_n$ und alle $n>1$, so gilt $f=E_k$.
\end{prop}
\begin{proof}
Es gilt $f\mid_{T_n}(\tau)=\sum_{m=0}^\infty b_mq^m$ mit $b_0=\sigma_{k-1}(n)$.
Also gilt $\lambda_n=\sigma_{k-1}(n)$.
Wir nehmen $f\not =E_k$ an.
Dann gilt $f-E_k \eqqcolon g \in S_k$ und $g$ verschwindet nicht überall.
Weiterhin gilt $g\mid_{T_n} \sigma_{k-1}(n)g$.
Aus der letzten Proposition folgt dann
\begin{align*}
\sigma_{k-1}(n)\leq n^{\frac{k}{2}-1 }\sigma_1(n).
\end{align*}
Allerdings gilt
\begin{align*}
2\sigma_{k-1}(n)-2n^{\frac{k}{2}-1} \sigma_1(n)&=\sum_{d\mid n}  \left(d^{k-1}+\left(\frac{n}{d}\right)^{k-1}-\frac{n^{\frac{k}{2}}}{n}(d+\frac{n}{d})\right)\\
&=\sum_{d \mid n} \frac{n^{\frac{k}{2}}}{d}\left(\frac{d^k}{n^{\frac{k}{2}}}+\frac{n^{\frac{k}{2}}-1}{d^{k-2}}-\left(\frac{d^2}{n}+1\right) \right)\\
&=\sum_{d \mid n} \frac{n^{\frac{k}{2}}}{d} \left(1-\left(\frac{n^{\frac{k}{2}}}{d}\right)^{k-2}\right) \left(\left(\frac{d}{n^{\frac{k}{2}}}\right)^k-1\right)\\
&>0
\end{align*}
da in der letzten Summe "quasi alle Summanden" positiv sind.
\end{proof}