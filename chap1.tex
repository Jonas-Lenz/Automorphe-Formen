\chapter{Absolutbeträge}
\begin{defi}
Sei $K$ ein Körper. Ein \emph{Absolutbetrag auf $K$} ist eine Abbildung $\abs{\cdot}\colon K \to \R$ mit
\begin{enumerate}[label=\roman*)]
\item $\abs{x}\geq 0$ und $\abs{x}=0 \Leftrightarrow x=0$,
\item $\abs{xy}=\abs{x}\abs{y}$,
\item $\abs{x+y}\leq \abs{x}+\abs{y}$
\end{enumerate}
für alle $x,y\in K$.
Gilt zusätzlich 
\begin{align*}
\abs{x+y}\leq \max(\abs{x},\abs{y}),
\end{align*}
so heißt $\abs{\cdot}$ \emph{nicht-archimedisch}.
Ist $\abs{\cdot}$ ein Betrag auf $K$, so definiert
\begin{align*}
d(x,y)\coloneqq \abs{x-y}
\end{align*}
eine Metrik auf $K$.
Zwei Absolutbeträge heißen \emph{äquivalent}, wenn sie dieselbe Topologie auf $K$ erzeugen.
\end{defi}

\begin{satz}
Seien $\abs{\cdot}_1,\abs{\cdot}_2$ zwei Absolutbeträge auf $K$.
Dann sind äquivalent
\begin{enumerate}[label=\roman*)]
\item $\abs{\cdot}_1$ und $\abs{\cdot}_2$ sind äquivalent,
\item es existiert $c>0$ mit $\abs{x}_1=\abs{x}_2^c$ für alle $x\in K$.
\end{enumerate}
\end{satz}

\begin{satz}
Sei $\abs{\cdot}$ ein Absolutbetrag auf $K$. Dann sind die folgenden Abbildungen stetig:
\begin{enumerate}[label=\roman*)]
\item $K\to \R, x \mapsto \abs{x}$,
\item $K\to K, x\mapsto -x$,
\item $K \times K \to K, (x,y)\mapsto x+y$,
\item $K^\ast \to K^\ast, x \mapsto x^{-1}$,
\item $K \times K \to K, (x,y)\mapsto xy$.
\end{enumerate}
\end{satz}

\begin{satz}
Sei $K$ ein Körper mit Absolutbetrag $\abs{\cdot}$.
Dann gibt es eine Körpererweiterung $\hat{K}/K$ und eine Fortsetzung von $\abs{\cdot}$ auf $\hat{K}$, sodass $K$ dicht in $\hat{K}$ und $\hat{K}$ vollständig bezüglich $\abs{\cdot}$ ist.

Ist $\tilde{K}$ ein weiterer Erweiterungskörper von $K$ mit obigen Eigenschaften,
so existiert genau ein Körperisomorphismus $\tilde{K}\to \hat{K}$, der auf $K$ die Identität ist
und die Absolutbeträge ineinander überführt.
\begin{proof}
Sei $K'\coloneqq\{(x_n)_{n\in \N}\subseteq K \mid (x_n)_{n\in \N} \text{ Cauchy-Folge bezüglich } \abs{\cdot}\}$ die Menge der Cauchy-Folgen in $K$.
Dann ist $K'$ ein kommutativer Ring mit $1$ und $N\coloneqq\{(x_n)_{n\in \N}\mid (x_n)_{n\in \N} \text{ Nullfolge}\}$
ein maximales Ideal.
Somit ist $\hat{K}=K'/N$ ein Körper. Die Abbildung
%TODO quotienten ordentlich macehn
\begin{align*}
K &\to \hat{K}\\
x &\mapsto (x)+N
\end{align*}
ist ein Körperhomomorphismus und somit injektiv.
Der Absolutbetrag auf $\hat{K}$ wird definiert durch
\begin{align*}
\abs{(x_n)}=\lim_{n\to \infty} \abs{x_n}
\end{align*}
wobei $\abs{x_n}$ aufgrund der umgekehrten Dreiecksungleichung eine Cauchy-Folge in $\R$ ist.
Dieser ist wohldefiniert, $\hat{K}$ ist vollständig und $K$ ist dicht in $\hat{K}$.
Die Einbettung $K \hookrightarrow \hat{K}$ ist auf einer dichten Teilmenge von $\tilde{K}$ definiert.
Da die Einbettung die Absolutbeträge erhält, ist sie stetig und lässt sich somit eindeutig zu einer stetigen Abbildung $\tilde{K}\to \hat{K}$ fortsetzen.
Diese Abbildung ist ein Isomorphismus, der die Absolutbeträge erhält.
Die Eindeutigkeit ist aus der Konstruktion klar.
\end{proof}
\end{satz}

\begin{bem}
Im Gegensatz ist der algebraische Abschluss nicht eindeutig bis auf \emph{eindeutigen} Isomorphismus.
\end{bem}