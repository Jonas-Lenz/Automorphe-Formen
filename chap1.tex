\chapter{Absolutbeträge}

Wir wollen die sogenannten $p$-adischen Zahlen betrachten, die als
Abschluss der rationalen Zahlen bezüglich einer geeigneten Topologie entstehen.
Dafür betrachten wir zunächst den allgemeinen Begriff des Absolutbetrags sowie einige Eigenschaften.

\begin{defi}
Sei $K$ ein Körper. Ein \emph{Absolutbetrag auf $K$} ist eine Abbildung $\abs{\cdot}\colon K \to \R$ mit
\begin{enumerate}[label=\roman*)]
\item $\abs{x}\geq 0$ und $\abs{x}=0 \Leftrightarrow x=0$,
\item $\abs{xy}=\abs{x}\abs{y}$,
\item $\abs{x+y}\leq \abs{x}+\abs{y}$
\end{enumerate}
für alle $x,y\in K$.
Gilt zusätzlich die starke Dreiecksungleichung
\begin{align*}
\abs{x+y}\leq \max(\abs{x},\abs{y}),
\end{align*}
so heißt $\abs{\cdot}$ \emph{nicht-archimedisch}.
Ist $\abs{\cdot}$ ein Betrag auf $K$, so definiert
\begin{align*}
d(x,y)\coloneqq \abs{x-y}
\end{align*}
eine Metrik auf $K$.
Zwei Absolutbeträge heißen \emph{äquivalent}, wenn sie die selbe Topologie auf $K$ erzeugen.
\end{defi}

Zunächst bemerken wir eine besonders einfache Charakterisierung äquivalenter Absolutbeträge.
Danach geben wir Beispiel für stetige Abbildungen an.
Die Beweise der nächsten beiden Sätze bleiben dem Leser als Übungsaufgabe überlassen.


\begin{satz}
Seien $\abs{\cdot}_1,\abs{\cdot}_2$ zwei Absolutbeträge auf $K$.
Dann sind äquivalent
\begin{enumerate}[label=\roman*)]
\item $\abs{\cdot}_1$ und $\abs{\cdot}_2$ erzeugen die selbe Topologie,
\item es existiert $c>0$ mit $\abs{x}_1=\abs{x}_2^c$ für alle $x\in K$.
\end{enumerate}
\end{satz}
\begin{proof}
Dies ist die Äquivalenz von i) und iii) in \cite[Lemma 3.1.2]{Gouvea}.
\end{proof}

\begin{satz}
Sei $\abs{\cdot}$ ein Absolutbetrag auf $K$. Dann sind die folgenden Abbildungen stetig:
\begin{enumerate}[label=\roman*)]
\item $K\to \R, x \mapsto \abs{x}$,
\item $K\to K, x\mapsto -x$,
\item $K \times K \to K, (x,y)\mapsto x+y$,
\item $K^\ast \to K^\ast, x \mapsto x^{-1}$,
\item $K \times K \to K, (x,y)\mapsto xy$.
\end{enumerate}
\end{satz}
%TODO Beweise, was ist Absolutbetrag auf K\times K?!

In der Analysis wurden die reellen Zahlen als Vervollständigung der rationalen Zahlen bezüglich des Betrages eingeführt.
Das Konzept der Vervollständigung lässt sich auch auf allgemeine Absolutbeträge verallgemeinern.

\begin{satz}
Sei $K$ ein Körper mit Absolutbetrag $\abs{\cdot}$.
Dann gibt es eine Körpererweiterung $\hat{K}/K$ und eine Fortsetzung von $\abs{\cdot}$ auf $\hat{K}$, sodass $K$ dicht in $\hat{K}$ und $\hat{K}$ vollständig bezüglich $\abs{\cdot}$ ist.\footnote{Wir bezeichnen auch die Fortsetzung auf $\hat{K}$ mit $\abs{\cdot}$.}
Ist $\tilde{K}$ ein weiterer Erweiterungskörper von $K$ mit obigen Eigenschaften,
so existiert genau ein Körperisomorphismus $\tilde{K}\to \hat{K}$, der auf $K$ die Identität ist
und die Absolutbeträge ineinander überführt.
\begin{proof}
Sei $K'\coloneqq\{(x_n)_{n\in \N}\subseteq K \mid (x_n)_{n\in \N} \text{ Cauchy-Folge bezüglich } \abs{\cdot}\}$ die Menge aller Cauchy-Folgen in $K$.
Dann ist $K'$ ein kommutativer Ring mit $1$ und man prüft leicht nach, dass $N\coloneqq\{(x_n)_{n\in \N}\mid (x_n)_{n\in \N} \text{ Nullfolge}\}$
ein maximales Ideal ist.
Somit ist $\hat{K}=\sfrac{K'}{N}$ ein Körper. Die Abbildung
%TODO quotienten ordentlich macehn
\begin{align*}
K &\to \hat{K}\\
x &\mapsto (x)+N
\end{align*}
ist ein Körperhomomorphismus und somit injektiv.
Der Absolutbetrag auf $\hat{K}$ wird durch
\begin{align*}
\abs{(x_n)}=\lim_{n\to \infty} \abs{x_n}
\end{align*}
definiert, wobei $\abs{x_n}$ aufgrund der umgekehrten
Dreiecksungleichung eine Cauchy-Folge in $\R$ ist.
Aufgrund der Vollständigkeit von $\R$ ist obiger Grenzwert wohldefiniert.
Nach Konstruktion ist $\hat{K}$ vollständig und $K$ ist dicht in $\hat{K}$.
Sei $\tilde{K}$ ein weiterer Erweiterungskörper von $K$ mit obigen Eigenschaften.
Die Einbettung $K \hookrightarrow \hat{K}$ ist auf einer dichten Teilmenge von $\tilde{K}$ definiert.
Da die Einbettung die Absolutbeträge erhält, ist sie stetig und lässt sich somit eindeutig zu einer stetigen Abbildung $\tilde{K}\to \hat{K}$ fortsetzen.
Diese Abbildung ist ein Isomorphismus, der die Absolutbeträge erhält.
Die Eindeutigkeit ist aus der Konstruktion klar.
\end{proof}
\end{satz}

\begin{bem}
Im Gegensatz ist der algebraische Abschluss nicht eindeutig bis auf \emph{eindeutigen} Isomorphismus.
\end{bem}