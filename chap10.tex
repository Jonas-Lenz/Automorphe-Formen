\chapter{Automorphe Formen auf $\GL(1,\A)$}
\begin{defi}
Sei $\omega \colon \A^\ast /\Q^\ast \to T$ ein Charakter, der nicht notwendigerweise endlich ist.
Eine \emph{automorphe Form} auf $\GL(1,\A)$ mit Charakter $\omega$
ist eine Funktion
\begin{align*}
\Phi \colon \GL(1,\A) \to \C
\end{align*}
mit den Eigenschaften
\begin{enumerate}[label=(\roman*)]
\item $\Phi(\gamma g)=\Phi(g)$ für alle $\gamma \in \GL(1,\Q)$ und $g \in \GL(1,\A)$,
\item $\Phi(zg)=\omega(z)\Phi(g)$ für alle $z,g \in \A^\ast$,
\item $\Phi$ wächst moderat.
\end{enumerate}
\end{defi}

Die automorphen Formen mit Charakter $\omega$ bilden einen komplexen
Vektorraum $S_\omega$.
Setzen wir in ii) $g=(1,1,\dots)$, so erhalten wir 
\begin{align*}
\Phi(z)=\omega(z)\Phi(g)=c \omega(z).
\end{align*}
Also ist $\dim_\C S_\omega=1$.

\begin{thm}
Sei $\Phi$ eine autormorphe Form auf $\GL(1,\A)$.
Dann kann $\Phi$ als
\begin{align*}
\Phi(g)=c \chi_{\text{idelic}}(g)\abs{g}^{\mathrm{i}t}
\end{align*}
mit $c \in \C$, $t \in \R$ und $\chi$ ein primitiver Dirichlet Charakter geschrieben werden kann.
Diese Darstellung ist eindeutig.
Wir können $\zeta$-Integrale für beliebige automorphe Formen
auf $\GL(1,\A)$ definieren.
Diese haben ähnliche Eigenschaften wie die, die zu endlichen Charakteren von $\A^\ast /\Q^\ast$ korrespondieren.
\end{thm}