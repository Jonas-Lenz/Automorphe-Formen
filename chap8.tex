\chapter{Die Riemannsche $\zeta$-Funktion}

Wir werden die analytischen Eigenschaften der Riemannschen $\zeta$-Funktion mit Hilfe adelischer Methoden zeigen.
Natürlich geht dies auch mit klassischen Methoden aus der Funktionentheorie,
aber unsere Methode wird alles in einen größeren Kontext stellen
und auch auf andere Situationen anwendbar sein.

\begin{defi}
Für $s\in \C$ mit $\Ret(s)>1$ ist die Riemannsche $\zeta$-Funktion durch
\begin{align*}
	\zeta(s)\coloneqq \sum_{n=1}^\infty \frac{1}{n^s}
\end{align*}
definiert.
\end{defi}

\begin{lem}
Die Reihe konvergiert lokal gleichmäßig und absolut.
\end{lem}

Wir betrachten nun die Poisson-Summation für Schwartzfunktionen.

\begin{defi}
Für $f \in \mathcal{S}(\A)$ definieren wir
\begin{align*}
	E(f)(x)=\sum_{t \in \Q^\ast} f(tx)
\end{align*}
für $x \in \A^\ast$.
\end{defi}

\begin{prop}
$E(f)$ konvergiert lokal gleichmäßig und absolut und definiert eine
stetige Funktion auf $\A^\ast/\Q^\ast$.
Weiterhin ist $E(f)$ schnell fallend, das heißt für alle $\Z\ni N >0$ existiert ein $C_N>0$ mit
\begin{align*}
	\abs{E(f)(x)}\leq \frac{C_N}{\abs{x}^N}
\end{align*}
für $\abs{x}\geq 1$.
Außerdem gilt
\begin{align*}
	E(f)(x)=\frac{1}{\abs{x}}\left(E(\hat{f})(\frac{1}{x})+\hat{f}(0)\right) -f(0).
\end{align*}
\end{prop}
\begin{proof}
Wir zeigen zunächst die Konvergenz der Reihe.
Es reicht diese für Funktionen der Form
\begin{align*}
	f(x)=f_{\infty}(x_\infty)\chi_{a+N\hat{Z}}(x_f)
\end{align*}
mit $f\in \mathcal{S}(\R)$, $a \in \A_f$, $\Z \ni N>0$.
Da $f_\infty$ eine Schwartzfunktion ist, existiert ein $C>0$, sodass
\begin{align*}
	\abs{f_\infty(x_\infty)}\leq \frac{C}{1+x_\infty^2}
\end{align*}
für alle $x_\infty \in \R$ gilt.
Sei $x_f \in \A_f^\ast$. Dann ist $x_f$ von der Form
\begin{align*}
	x_f=r u
\end{align*}
für eindeutige $r \in \Q_{\geq 0}^\ast$, $u \in \hat{Z} ^\ast$.
Insbesondere ist dann $r \hat{Z} ^\ast$ eine kompakte offene Teilmenge von $\A_f^\ast$.
Für $t \in \Q^\ast$ gilt
\begin{align*}
tx_f &\in a+N\hat{Z}\\
\Rightarrow tru &\in a+ N \hat{Z}\\
\Rightarrow t &\in r^{-1}\left(u^{-1}a+N\hat{Z}\right)\\
\Rightarrow t &\in r^{-1}s (\hat{Z}\cap \Q^\ast)\\
\Rightarrow t &\in \frac{1}{n}\hat{Z}
\end{align*}
für ein $Z \ni n>0$.
Somit gilt nun
\begin{align*}
\abs{E(f)(x)}&\leq \sum_{t \in \Q^\ast} \abs{f(tx)}\\
&\leq \sum_{t \in \Q^\ast} \abs{f_\infty(tx_\infty)}\chi_{a+N\hat{Z}}(tx_f)\\
&\leq \sum_{t \in \Q^\ast, tx_f \in a+N\hat{Z}} \abs{f_\infty(tx_\infty)}\\
&\leq c\sum_{t \in \frac{1}{n}\Z} \frac{1}{1+(tx_\infty)^2}\\
&\leq c\sum_{t\in \Z} \frac{1}{1+t^2x_\infty^2/n^2}.
\end{align*}
Die rechte Seite verhält sich wie $\zeta(2)$.
Daher konvergiert $E(f)$ gleichmäßig auf Mengen der Form $I \times r \hat{Z} ^\ast$, wobei $I \subseteq \R^\ast$ kompakt ist.

Ebenso kann man zeigen, dass $E(f)$ schnell fallend ist.
Die Darstellung über die Fouriertransformation folgt aus der Poissonschen Summationsformel.
Für $x\in \A^\ast$ definieren wir
\begin{align*}
	f_x(y)\colon f(xy) \in \mathcal{S}(\A).
\end{align*}
Damit gilt
\begin{align*}
	\hat{f}_x(y)&=\int_\A f_x(z)\overline{\ex(zy)}~\mathrm{d}z\\
	&=\int_\A f(xz)\ex(-xzy/x)~\mathrm{d}z\\
	&=\frac{1}{\abs{x}}\hat{f}(y/x).
\end{align*}
Daraus folgt nun
\begin{align*}
	E(f)(x)&=\sum_{t \in \Q} f(tx)-f(0)\\
	&=\sum_{t\in \Q} f_x(t)-f(0)\\
	&=\sum_{t\in \Q} \hat{f}_x(t)-f(0)\\
	&=\frac{1}{\abs{x}}\sum_{t \in Q} \hat{f}(t/x)-f(0)\\
	&=\frac{1}{\abs{x}}\left(\sum_{t \in \Q^\ast} \hat{f}(t/x)+\hat{f}(0)\right)-f(0)\\
	&=\frac{1}{\abs{x}} \left(E(\hat{f})(1/x)+\hat{f}(0)\right)-f(0),
\end{align*}
was die Behauptung zeigt.
\end{proof}

\begin{defi}
Für $f \in \mathcal{S}(\A)$ definieren wir das $\zeta$-Integral von $f$ durch
\begin{align*}
	\zeta(f,s)=\int_{\A^\ast} f(x)\abs{x}^s~\mathrm{d}^\ast x,
\end{align*}
wobei $\mathrm{d}^\ast x$ das normalisierte Haarmaß auf $\A^\ast$ ist.
\end{defi}

Wenn man die richtige Funktion einsetzt erhält man die Riemannsche $\zeta$-Funktion.
Allerdings kann man mit dieser Methode auch Funktionalgleichungen für andere holomorphe Funktionen zeigen.

\begin{prop}
Sei $g=g_\infty g_f \in \mathcal{S}(\A)$ mit
\begin{align*}
g_\infty(x_\infty)&=\ex^{-\pi x_\infty^2} \in \mathcal{S}(\R),\\
g_f(x_f)&=\chi_{\hat{Z}}(x_f) \in \mathcal{S}(\A_f).
\end{align*}
Dann gilt für $\Ret(s)>1$
\begin{align*}
\zeta(g,s)=\pi^{-s/2} \Gamma(\frac{s}{2})\zeta(s),
\end{align*}
das heißt $\zeta(g,s)$ ist die vervollständigte Riemannsche $\zeta$-Funktion.
\end{prop}
\begin{proof}
Es gilt
\begin{align*}
\zeta(g,s)=&\int_{\R^\ast} g_\infty(x_\infty)\abs{x_\infty}^s~\mathrm{d}^\ast x_\infty \cdot\\
&\int_{\A_f^\ast} g_f(x_f)\abs{x_f}^s~\mathrm{d}_f^\ast x_f.
\end{align*}
Das erste Integral ergibt
\begin{align*}
\int_{\R^\ast} g_\infty(t) \abs{t}^s \frac{\mathrm{d}t}{\abs{t}}&=2\int_0^\infty \mathrm{e}^{-\pi t^2} t^{s-1}~\mathrm{d}t\\
&=\pi^{-s/2} \int_0^\infty \ex^{-x} x^{s/2-1}~\mathrm{d}x\\
&=\pi^{-s/2} \Gamma(s/2).
\end{align*}
Das zweite Integral ergibt
\begin{align*}
\int_{\A_f^\ast} g_f(x_f)\abs{x_f}^s~\mathrm{d}_f^\ast x_f&=\prod_{p<\infty} \int_{\Q_p^\ast} g_p(x_p)\abs(x_p)_p^s ~\mathrm{d}^\ast x_p\\
&=\prod_{p<\infty} \int_{\Z_p \setminus \{0\}} \abs{x_p}_p^s~\mathrm{d}^\ast x_p\\
&=\prod_{p<\infty} \frac{1}{1+p^{-s}}=\zeta(s).
\end{align*}
%TODO woher wissen wir das?!
Wenn wir die beiden Rechnung kombinieren, folgt die Behauptung.
\end{proof}

Wir zeigen nun folgende allgemeine Aussage.
Als Spezialfall wird daraus die Funktionalgleichung der Riemannschen $\zeta$-Funktion folgen.
\begin{thm}
Es gilt
\begin{align*}
	\zeta(f,s)=\int_{\A^\ast/\Q^\ast} E(f)(x)\abs{x}^s~\mathrm{d}^\ast x.	
\end{align*}
Das Integral konvergiert lokal gleichmäßig für $\Ret(s)>1$ und definiert eine holomorphe Funktion.
Weiterhin hat es eine meromorphe Fortsetzung nach $\C$ und ist holomorph auf $\C\setminus\{0,1\}$.
In $s=0,1$ liegt ein Pol der Ordnung $\leq 1$ mit Residuum $-f(0)$ bzw.\, $\hat{f}(0)$ vor.
Außerdem erfüllt das $\zeta$-Integral die Funktionalgleichung
\begin{align*}
	\zeta(f,s)=\zeta(\hat{f},1-s).
\end{align*}
\end{thm}
\begin{proof}
Zunächst zeigen wir die Konvergenz für $\Ret(s)>1$.
Es gilt
\begin{align*}
\abs{f(x)} \leq \frac{c}{1+\abs{x_\infty}^N} \chi_{\frac{1}{n}\hat{Z}} (x_f)
\end{align*}
für geeignetes $c>0$, $\Z \ni N,n>0$ und $N>\Ret(s)$.
Damit können wir nun leicht die Konvergenz zeigen, denn es gilt
\begin{align*}
\zeta(\abs{f},s)&=\int_{\A^\ast} \abs{f(x)} \abs{x}^s~\mathrm{d}^\ast x\\
&=\int_{\R^\ast} \abs{f_\infty(t)}\abs{t}^{s} \frac{\mathrm{d}t}{\abs{t}} \int_{\A_f} \abs{f(x_f)}\abs{x_f}^s ~\mathrm{d}_f^\ast x_f\\
&\leq 2C \int_{\R_{\geq 0}^\ast} \frac{t^{s-1}}{1+t^N}~\mathrm{d}t\int_{\frac{1}{n}\hat{Z}\cap \A_f^\ast} \abs{x_f}^s~\mathrm{d}_f^\ast x_f.
\end{align*}
Der erste Faktor ist endlich, da wir $N$ groß genug gewählt haben.
Der zweite Faktor konvergiert wie $\zeta(s)$ für $\Ret(s)>1$.
Diese Abschätzungen gelten lokal gleichmäßig in $s$ und dies zeigt die Konvergenz.

Als nächstes zeigen wir, dass wir $\zeta(f,s)$ auch durch obiges Integral darstellen können.
Dazu verwenden wir, dass die Isomorphie
\begin{align*}
\A^\ast &\to \A^1\times \R_{>0}^\ast\\
(a_\infty,a_f)&\mapsto ((\frac{a_\infty}{\abs{a}},a_f),\abs{a})
\end{align*}
einen Isomorphismus
\begin{align*}
\A^\ast/\Q\ast  &\to \A^1/\Q^\ast \times \R_{>0}^\ast\\
(a_\infty,a_f)\Q^\ast&\mapsto ((\frac{a_\infty}{\abs{a}},a_f)\Q^\ast,\abs{a})
\end{align*}
induziert.
In Kombination mit dem Isomorphismus
\begin{align*}
\A^1/\Q^\ast &\to \hat{Z} ^\ast\\
(1,a_f)\Q^\ast &\mapsto a_f
\end{align*}
erhalten wir einen Isomorphismus
\begin{align*}
\A^\ast/\Q^\ast \to \hat{Z} ^\ast \times \R_{>0}^\ast \eqqcolon \mathcal{F}.
\end{align*}
Für $s\in \C$ mit $\Ret(s)>1$ gilt (da $\mathrm{d}^\ast x$ ein invariantes Maß ist)
\begin{align*}
\zeta(f,s)&=\int_{\A^\ast} f(x)\abs{x}^s~\mathrm{d}^\ast x\\
&=\sum_{t\in \Q^\ast}\int_{t\mathcal{F}} f(x)\abs{x}^s~\mathrm{d}^\ast x\\
&=\sum_{t \in \Q^\ast} \int_\mathcal{F} f(tx)\abs{tx}^s~\mathrm{d}^\ast x\\
%was passiert hier?!
&=\sum_{t \in \Q^\ast} \int_\mathcal{F} f(tx)\abs{x}^s~\mathrm{d}^\ast x\\
&=\int_\mathcal{F} \sum_{t \in \Q^\ast} f(tx)\abs{x}^s ~\mathrm{d}^\ast x\\
&=\int_{\A^\ast/\Q^\ast} E(f)(x)\abs{x}^s ~\mathrm{d}^\ast x,
\end{align*}
wobei wir verwendet haben, dass das Integral absolut konvergiert, damit wir Integral und Summe (Integral bzgl.\, Zählmaß) vertauschen dürfen.
Da $\{1\}\subseteq \R_{>0}\ast$ eine Nullmenge ist und $A^1/\Q^\ast$ kompakt ist, hat auch
\begin{align*}
\A^1/\Q^\ast \times \{1\}\subseteq \A^\ast/\Q^\ast
\end{align*}
Maß $0$.
Damit folgt (da Nullmengen für das Integral unwichtig sind)
\begin{align*}
\zeta(f,s)=\int_{\A^\ast /\Q^\ast \abs{x}>1} E(f)(x)\abs{x}^s~\mathrm{d}^\ast x + \int_{\A^\ast /\Q^\ast \abs{x}<1} E(f)(x)\abs{x}^s~\mathrm{d}^\ast x
\end{align*}
%TODO zweizeilige Integralgrenzen machen
Da $E(f)$ schnell fallend ist, konvergiert das erste Integral für $s\in \C$, da
\begin{align*}
\abs{\int_{\A^\ast /\Q^\ast \abs{x}>1} E(f)(x)\abs{x}^s~\mathrm{d}^\ast x} &\leq \int_{\abs{x}>1} \abs{E(f)(X)} \abs{\abs{x}^s} ~\mathrm{d}^\ast x\\
&\leq \int_{\abs{y}>1} \frac{C_n}{\abs{x}^N} \abs{x}^{\Ret(s)}~\mathrm{d}^\ast x\\
&\leq C_N \int_{A^1/\Q^\ast} \int_1^\infty t^{\Ret(s)-N} \frac{\mathrm{d}t}{t}~\mathrm{d}^\ast x\\
&\leq C_N \int_1^\infty t^{\Ret(s)-N-1}~\mathrm{d}t
\end{align*}
mit Fubini folgt.
Wenn wir $N$ groß genug wählen ($\Ret(s)<N$), konvergiert das letzte Integral
und die Konvergenz ist lokal gleichmäßig.
Daraus folgt, dass $\int_{\abs{x}>1} E(f)(x)\abs{x}^s~\mathrm{d}^\ast x$ eine holomorphe Funktion auf $\C$ definiert.

Für das zweite Integral verwenden wir die Darstellung von $E(f)$ aus
Proposition 8.4.
Damit erhalten wir
\begin{align*}
\int_{\abs{x}<1} E(f)(x)\abs{x}^s~\mathrm{d}^\ast x &= \int_{\abs{x}<1}E(\hat{f})(\frac{1}{x})\abs{x}^{s-1}~\mathrm{d}^\ast x\\
&+ \hat{f}(0)\int_{\abs{x}<1} \abs{x}^{s-1} ~\mathrm{d}^\ast x -f(0)\int_{\abs{x}<1}\abs{x}^s ~\mathrm{d}^\ast x.
\end{align*}
Wir betrachten nun jedes Integral einzeln und beginnen mit dem zweiten.
Es gilt
\begin{align*}
\int_{\abs{x}<1}\abs{x}^{s-1}~\mathrm{d}^\ast x &=\int_{\A^1/\Q^\ast} \int_0^1 t^{s-1}\frac{\mathrm{d}t}{t}~\mathrm{d}^\ast x\\
&=\int_0^1 t^{s-2}~\mathrm{d}t=\frac{1}{s-1}.
\end{align*}
Analog folgt
\begin{align*}
\int_{\abs{x}<1} \abs{x}^s~\mathrm{d}^\ast x =\frac{1}{s}.
\end{align*}
Für das erste Integral verwenden wir die Substitution $y=\frac{1}{x}$ und erhalten
\begin{align*}
\int_{\abs{x}<1}E(\hat{f})(\frac{1}{x})\abs{x}^{s-1}~\mathrm{d}^\ast x&=\int_{\abs{y}>1} E(\hat{f})(y)\abs{y}^{1-s}~\mathrm{d}^\ast y.
\end{align*}
Das schwierige bei dieser Substitution ist, wie sich das Maß verhält.
Für $\nu=\infty$ gilt für die Substitution $y=\frac{1}{x}$ mit $\frac{\mathrm{d}y}{\mathrm{d}x}=-\frac{1}{x^2}$ und somit
\begin{align*}
\int_0^\infty f(\frac{1}{x})~\frac{\mathrm{d}x}{x}=-\int_{\infty}^0 f(y)\frac{\mathrm{d}y}{y}.
\end{align*}
Der Fall $\nu<\infty$ verbleibt als Übung.
Insgesamt erhalten wir
\begin{align*}
\zeta(f,s)=\int_{\abs{x}>1}E(f)(x)\abs{x}^s +E(\hat{f})(x)\abs{x}^{1-s}~\mathrm{d}^\ast x +\frac{\hat{f}(0)}{s-1}-\frac{f(0)}{s}.
\end{align*}
Dies zeigt die meromorphe Fortsetzung sowie die Aussage über die Residuen.
\end{proof}
Dieses Theorem ist eines der Hauptresultate der Vorlesung und liefert eine schöne Beschreibung der Eigenschaften des $\zeta$-Integrals.
Für $g \in \mathcal{S}(A)$ aus Proposition 8.6 gilt $g=\hat{g}$,
sodass $\zeta(g,s)=\zeta(g,1-s)$ folgt.
\begin{prop}
Sei $\zeta^\ast(s)=\pi^{-s/2} \Gamma(s/2)\zeta(s)$ für $\Ret(s)>1$.
Dann definiert $\zeta^\ast$ ist eine meromorphe Funktion auf $\C$ mit einfachen Polen in $0$ und $1$.
Weiterhin gilt $\zeta^\ast(s)=\zeta^\ast(1-s)$.
\end{prop}