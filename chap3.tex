\chapter{Integration}
\begin{defi}
Sei $X$ ein Hausdorffraum. Die \emph{Borel $\sigma$-Algebra} $\sigma(X)$ von $X$ ist die kleinste $\sigma$-Algebra, die die offenen Mengen von $X$ enthält.
Ein \emph{Radonmaß} auf $X$ ist ein Maß $\mu \colon \sigma(X) \to [0,\infty]$ sodass
\begin{enumerate}[label=\roman*)]
\item $\mu$ ist lokal endlich, das heißt für jedes $x\in X$ existiert eine offene Umgebung $U$ von $x$ die endliches Maß hat.
\item $\mu$ ist \emph{regulär von innen}, das heißt
\begin{align*}
\mu(A)=\sup\{\mu(K)\mid K\subset A \text{ kompakt }\}
\end{align*}
gilt für alle $A \in \sigma(X)$.
\end{enumerate}
\end{defi}

\begin{bsp}
Das Lebesguemaß auf $\Rn$ ist ein Radonmaß.
\end{bsp}

\begin{thm}[Riesz]
Sei $X$ ein lokal kompakter Hausdorffraum und
\begin{align*}
I \colon C_c(X) \to \C
\end{align*}
eine \emph{positive lineare Form}, das heißt $I(f)\geq 0$ für alle $0\leq f \in \C_c(X)$.
Dann existiert ein eindeutiges Radonmaß $\mu \colon \sigma(X)\to [0,\infty]$ mit $I(f)=\int_X f ~\mathrm{d}\mu$.
Außerdem gilt für kompakte Mengen $K$
\begin{align*}
\mu(K)=\inf\{I(f)\mid f\in \C_c(X),f \geq \chi_K\}
\end{align*}
sowie
\begin{align*}
\mu(A)=\sup\{\mu(K)\mid K\subseteq A \text{ kompakt } \}
\end{align*}
für alle $A \in \sigma(X)$.
\end{thm}
\begin{proof}
Elstrodt, Kapitel 8 von Maß- und Integrationstheorie.
\end{proof}

\begin{thm}[Haar]
Sei $G$ eine lokal kompakte Hausdorffgruppe.
Dann gibt es ein linksinvariantes Radonmaß $\mu$ auf $G$.
Dieses ist eindeutig bis auf positive Konstanten und heißt \emph{Haarmaß} auf $G$.
\end{thm}

\begin{bsp}
\begin{enumerate}[label=\roman*)]
\item Auf einer diskreten Gruppe ist das Zählmaß ein Haarmaß.
\item Auf $(\Rn,+)$ ist das Lebesguemaß ein Haarmaß.
\item Auf $(\R^\ast,\cdot)$ ist $\frac{\mathrm{d}x}{\abs{x}}$, wobei $\mathrm{d}x$ das Lebesguemaß auf $\R$ ist, ein Haarmaß.
\item Ein Haarmaß auf $\GL(2,\R)$ ist durch
\begin{align*}
\frac{\mathrm{d}x_{11}\mathrm{d}x_{12}\mathrm{d}x_{21}\mathrm{d}x_{22}}{\abs{\det(M)}^2}
\end{align*}
gegeben. (M wie man es erwartet).
\end{enumerate}
\end{bsp}

\begin{thm}
Sei $G$ eine lokal kompakte Hausdorffgruppe und $K\subseteq G$ kompakt. Dann gibt es ein eindeutiges Haarmaß $\mu$ auf $G$, das $\mu(K)=1$ erfüllt.
\end{thm}


\begin{defi}
Sei $G$ eine lokal kompakte Hausdorffgruppe und $\mu$ ein Haarmaß auf $G$.
Dann definiert für $x\in G$
\begin{align*}
\mu_x(A)=\mu(Ax)
\end{align*}
ein links invariantes Maß auf $G$.
\end{defi}

\begin{defi}
Die Eindeutigkeit impliziert $\mu_x=\delta(x) \mu$ für ein $\delta(x)\in \R_{\geq 0}$.
Insbesondere ist $\delta(x)$ unabhängig von der Wahl von $\mu$.
Die Funktion
\begin{align*}
\delta \colon G &\to \R_{\geq 0}\\
x &\mapsto \delta(x)
\end{align*}
heißt \emph{modulare Funktion} von $G$.
\end{defi}

\begin{prop}
Die Abbildung $\delta$ ist ein Morphismus topologischer Gruppen.
\end{prop}
\begin{proof}
Für $x,y\in G$ gilt
\begin{align*}
\delta(xy)\mu(A)&=\mu_{xy}(A)=\mu(Axy)\\
&=\mu_y(Ax)=\delta(y)\delta(x)\mu(A).
\end{align*}
Der Rest verbleibt als Übungsaufgabe.
\end{proof}

\begin{bsp}
Man kann zeigen, dass jeder stetiger Gruppenhomomorphismus $h \colon \SL(2,\R)\to \R_{\geq 0}$ konstant sein muss.
Daraus folgt, dass jedes Haarmaß auf $\SL(2,\R)$ auch rechts invariant ist.
\end{bsp}

Wir betrachten nun die lokal kompakten Hausdorffgruppen $\Q_p$ sowie $\Q_p^\ast$.
Wir normieren das Haarmaß $\mu$ auf $\Q_p$ durch die Forderung $\mu(\Z_p)=1$.

\begin{prop}
Für eine messbare Menge $A \subseteq \Q_p$ und $x \in \Q_p^\ast$ gilt
\begin{align*}
\mu(xA)=\abs{x}_p\mu(A).
\end{align*}
\end{prop}

\begin{proof}
Für $x \in \Q_p^\ast$ definiert
\begin{align*}
\mu_x(A)=\mu(xA)
\end{align*}
ein Radonmaß auf $\Q_p$.
Dies ist links invariant, da
\begin{align*}
\mu_x(y+A)=\mu((y+A)x)=\mu(yx+Ax)=\mu_x(A)
\end{align*}
gilt.
Also folgt aus der Eindeutigkeit, dass $\mu_x(A)=c(x)\mu(A)$ mit $c(x)\in \R_{\geq 0}$ unabhängig von $A$ gilt.
Sei zunächst $\abs{x}_p=p^{-n}$ für ein $n>0$.
Dann gilt $x\Z_p=p^n\Z_p$ sowie 
\begin{align*}
\Z_p=\bigcup_{j=0}^{p^n-1}(j+p^n\Z_p)
\end{align*}
als disjunkte Vereinigung.
Aus $\sigma$-Additivität folgt daher
\begin{align*}
\mu(\Z_p)&=\sum_{j=0}^{p^n-1} \mu(j+p^n\Z_p)\\
&=p^n \mu(p^n \Z_p),
\end{align*}
was $\abs{x}_p\mu(\Z_p)=\mu(x\Z_p$ impliziert.
Also ist in diesem Fall $c(x)=\abs{x}_p$.
Als nächstes betrachten wir den Fall $\abs{x}_p=p^n$ mit $n>0$.
Dann gilt
$\mu(A)=\mu(x^{-1}xA)=\abs{x^{-1}}_p\mu(xA)$,
also folgt $\mu(xA)=\abs{x}_p\mu(A)$.
\end{proof}

\begin{cor}
Sei $f$ integrierbar. Dann gilt für $a \in \Q_p^\ast$
\begin{align*}
\int_{\Q_p}f(a^{-1}x)~\mathrm{d}\mu(x)=\abs{a}_p\int_{\Q_p}f_(x)~\mathrm{d}\mu(x).
\end{align*}
\end{cor}
\begin{proof}
Wir wollen die letzte Proposition verwenden
Es gilt 
\begin{align*}
\int_{\Q_p}f(a^{-1}x)~\mathrm{d}\mu(x)&=\int_{\Q_p} f(x)~\mathrm{d}\mu_a(x)\\
&=\abs{a}_p \int_{\Q_p} f(x)~\mathrm{d}\mu(x).
\end{align*}
Wir werden das normierte Haarmaß auf $\Q_p$ mit $\mathrm{d}x$ notieren.
\end{proof}

\begin{bsp}
Wir können jetzt das Volumen von $\Z_p^\ast=\bigcup_{j=1}^{p-1}(j+p\Z_p)$ berechnen.
Es gilt
\begin{align*}
\mu_{\mathrm{d}x}(\Z_p^\ast)&=\sum_{j=1}^{p-1} \mu_{\mathrm{d}x}(j+p\Z_p)\\
&=(p-1)\mu_{\mathrm{d}x}(p \Z_p)\\
&=\frac{p-1}{p}\mu_{\mathrm{d}x}(\Z_p)\\
&=\frac{p-1}{p}.
\end{align*}
\end{bsp}

Wir konstruieren nun ein Haarmaß auf $\Q_p^\ast$.
Für $A \subseteq \Q_p^\ast\subseteq \Q_p$ definieren wir
\begin{align*}
\mu_{\frac{\mathrm{d}x}{\abs{x_p}}}(A)\coloneqq \int_{\Q_p} \chi_A(x)\frac{\mathrm{d}x}{\abs{x_p}}.
\end{align*}
Da dies in $0$ nicht definiert ist, setzen wir den Integranden in diesem Punkt auf $0$.
%warum auch immer das gut geht.
Dann gilt für $y\in \Q_p^\ast$, dass
\begin{align*}
\mu_{\frac{\mathrm{d}x}{\abs{x_p}}}(yA)&=\int_{\Q_p} \chi_{yA}(x)\frac{\mathrm{d}x}{\abs{x}_p}\\
&=\abs{y^{-1}}_p \int_{\Q_p}\chi_A(y^{-1}x)\frac{\mathrm{d}x}{\abs{y^{-1}x}_p}\\
&=\int_{\Q_p} \chi_A(x)\frac{\mathrm{d}x}{\abs{x}_p}\\
&=\mu_{\frac{\mathrm{d}x}{\abs{x_p}}}(A).
\end{align*}
Also ist $\mu_{\frac{\mathrm{d}x}{\abs{x_p}}}$ links invariant.
Das Volumen von $\Z_p^\ast$ bezüglich $\mu_{\frac{\mathrm{d}x}{\abs{x_p}}}$ ist
\begin{align*}
\mu_{\frac{\mathrm{d}x}{\abs{x_p}}}(\Z_p^\ast)&=\int_{\Q_p}\chi(\Z_p^\ast) \frac{\mathrm{d}x}{\abs{x}_p}\\
&=\int_{\Q_p}\chi(\Z_p^\ast)~\mathrm{d}x=\mu_{\mathrm{d}x}(\Z_p^\ast)\\
&=\frac{p-1}{p}.
\end{align*}

\begin{thm}
${\frac{\mathrm{d}x}{\abs{x_p}}}$ definiert ein Haarmaß auf $\Q_p^\ast$.
Das normierte Haarmaß
\begin{align*}
\mathrm{d}^\ast x =\frac{p}{p-1}{\frac{\mathrm{d}x}{\abs{x_p}}}
\end{align*}
erfüllt $\mu_{\mathrm{d}^\ast x}(\Z_p^\ast)=1$.
\end{thm}

Für $a \in \R_{\geq 0}$ und $s\in \C$ setzen wir
\begin{align*}
a^s=\mathrm{e}^{s \log(a)}.
\end{align*}

Die Riemannsche $\zeta$-Funktion ist durch
\begin{align*}
\zeta(s)\coloneqq \sum_{n=1}^\infty \frac{1}{n^s}
\end{align*}
definiert.
Die Reihe konvergiert für $\Ret(s)>1$ und besitzt eine meromorphe Fortsetzung auf $\C$ mit einer Polstelle vom Grad $1$ in $s=1$.
Für $\Ret(s)>1$ gilt
\begin{align*}
\zeta(s)=\prod_p \frac{1}{1-p^{-2}}.
\end{align*}
Für $\Ret(s)>0$ werden wir
\begin{align*}
\int_{\Z_p \setminus \{0\}} \abs{x}_p^s ~\mathrm{d}^\ast x
\end{align*}
berechnen.
Wir können das Integrationsgebiet durch
\begin{align*}
\Z_p\setminus \{0\}=\bigcup_{k=0}^\infty p^k \Z_p^\ast
\end{align*}
in disjunkte Mengen zerlegen.
Somit gilt
\begin{align*}
\int_{\Z_p \setminus \{0\}} \abs{x}_p^s ~\mathrm{d}^\ast x&=\sum_{k=0}^\infty \int_{p^k \Z_p^\ast} \abs{x}_p^s ~\mathrm{d}^\ast x\\
&=\sum_{k=0}^\infty \left(p^{-k}\right)^s \int_{p^k \Z_p^\ast}\mathrm{d}^\ast x \\
&=\sum_{k=0}^\infty \left(p^{-k}\right)^s=\sum_{k=0}^\infty \left(p^{-2}\right)^k\\
=\frac{1}{1-p^{-s}}.
\end{align*}
Dies scheint zunächst \glqq zufällig\grqq\, zu sein, aber wir werden später sehen, dass es hier einen tieferen Zusammenhang gibt.