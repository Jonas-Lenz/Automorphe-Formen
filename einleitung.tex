\chapter*{Einleitung}
Sei $G$ eine Gruppe, die auf einem topologischen Raum $X$ operiert.
Eine Funktion $f\colon X \to \C$ heißt \emph{automorphe Form auf $X$}, wenn
\begin{align*}
f(gx)=  \Psi(g,x) f(x)
\end{align*}
für alle $g\in G,x\in X$ für eine geeignete Funktion $\Psi \colon G \times X \to \C $ gilt.
Häufig werden noch weitere Bedingungen zum Beispiel an das Wachstum gefordert.
Einer automorphen Form kann man eine automorphe Darstellung zuordnen und dieser wiederum eine $L$-Reihe.
%TODO wie schreibt man L-Reihe richtig?!
Es wird vermutet, dass diese automorphen $L$-Reihen dieselben sind, die man in der arithmetischen algebraischen Geometrie findet (Langlands-Programm).
Ein Spezialfall dieser Korrespondenz ist der Modularitätssatz:
\begin{satz}[Breuil, Conrad, Diamond, Taylor (2001), Wiles (1995) vermutet von Taniyama und Shimura (1958)]
Sei $E$ eine rationale elliptische Kurve mit Führer $N$.
Dann gibt es eine Neuform $f\in S_2(\Gamma_0(N))$ mit
\begin{align*}
L_f=L_E.
\end{align*}
\end{satz}
In dieser Vorlesung betrachten wir automorphe Formen auf $\GL(n,\mathds{A}_\Q)$ für $n=1$ und wenn es die Zeit erlaubt für $n=2$.
%TODO strich für A ist auf der falschen Seite
\section*{Literatur}
Goldfeld, Hundley: Automorphic representations and $L$-functions for the general linear group, Cambridge University Press.\\
Skript von Herrn Brunier