\chapter{Adele}
Wir wollen alle Vervollständigungen von $\Q$ gleichzeitig betrachten.
Dafür betrachten wir den lokalkompakten Ring 
\begin{align*}
\A=\{(x_\infty,x_2,x_3,x_5,\dots,)\mid x_\nu \in \Q_\nu \text{ und } x_p \in \Z_p \text{ a.e.}\}
\end{align*}
%TODO a.e. schön machen 
von Adelen.

Sei $(X_i)_{i\in I}$ eine Familie topologischer Räume.
Die Produkttopologie auf $\prod_{i \in I} X_i$ ist die gröbste Topologie auf $\prod_{i \in I} X_i$, bezüglich der alle Projektionen stetig sind.
Eine Basis für diese Topologie ist durch $\{\prod_{i\in I}O_i\}$ gegeben, wobei $O_i$ für alle $i \in I$ offen ist und $O_i=X_i$ für fast alle $i \in I$ gilt.

Mengen dieser Form heißen \emph{offene Rechtecke}.

Wir benötigen folgenden klassischen Satz aus der Topologie.
\begin{thm}[Tychonoff]
Sei $(X_i)_{i \in I}$ eine Familie von topologischen Räumen. Dann ist das Produkt der $X_i$ genau dann kompakt, wenn alle $X_i$ kompakt sind.
\end{thm}

\begin{thm}
Sei $(X_i)_{i \in I}$ eine Familie von Hausdorffräumen. Dann ist $\prod_{i \in I} X_i$ lokalkompakt genau dann wenn alle $X_i$ lokalkompakt und alle bis auf endlich viele sogar kompakt sind.
\end{thm}
\begin{proof}
\glqq $\Rightarrow$\grqq : Da die Projektionen $p_j \colon \prod_{i \in I} X_i \to X_j$ stetig sind, folgt, dass alle $X_i$ lokalkompakt sind.
Sei $x=(x_i)_{i\in I} \in \prod_{i \in I} X_i$ und $C=(C_i)_{i \in I}$ eine kompakte Umgebung von $x$.
$C$ enthält eine Vereinigung offener Rechtecke. Da alle bis auf endlich viele Komponenten eines offenen Rechtecks der ganze Raum sind, gilt das gleich auch für $C$, das heißt
\begin{align*}
\pi_i(C)=X_i
\end{align*}
für alle bis auf endlich viele $i \in I$.\\
\glqq $\Leftarrow$\grqq : Da mit $J=\{i\in I \mid X_i \text{ kompakt }\}$, die Menge $I \setminus J$ endlich ist, folgt mit
\begin{align*}
X=\prod_{i \in J} X_i \times \prod_{i \in I\setminus J}X_i,
\end{align*}
dass $X$ als Produkt einer kompakten und einer lokalkompakten Menge lokalkompakt ist.
\end{proof}

\begin{bsp}
$\prod_{p <\infty} \Q_p$ ist nicht lokalkompakt und daher gibt es dort kein Haarmaß.
\end{bsp} 
%TODO wissen wir wirklich dass es keins gibt?!

\begin{defi}
Sei $(X_i)_{ i \in I}$ eine Familie lokalkompakter Hausdorffräume und für jedes $i \in I$ sei $\emptyset \not =K_i\subseteq X_i$ eine kompakte offene Menge.
Diese muss es nicht geben, betrachte dafür zum Beispiel $\Rn$.
Dann definieren wir das eingeschränkte Produkt
\begin{align*}
X=\widehat{\prod_{i \in I}}\vphantom{\prod}^{K_i} X_i \coloneqq \{(x_i)_{i \in I} \in \prod_{i \in I} X_i \mid x_i \in K_i \text{ für fast alle } i \in I\}.
\end{align*}
%TODO eine gute art finden, dies zu texen, danke Peter, an allen Stellen einfügen
%TODO Was muss man hier wirklich an K_i fordern?!
Falls $K_i$ aus dem Kontext klar ist, lassen wir diese in der Schreibweise weg.
Ein eingeschränktes offenes Rechteck ist eine Menge der Form
\begin{align*}
\prod_{i\in I} U_i
\end{align*}
wobei $U_i \subseteq X_i$ offen ist und $U_i=K_i$ für alle bis auf endlich viele $i  \in I$ gilt.
Eine Teilmenge $U\subseteq X$ heißt \emph{offen}, wenn es eine Vereinigung solcher Mengen ist.
Die dazugehörige Topologie heißt die \emph{eingeschränkte Produkttopologie}.
\end{defi}

\begin{prop}
Sei $(X_i)_{i\in I}$ eine Familie lokalkompakter Hausdorffräume.
Für jedes $i \in I$ sei $K_i\subseteq X_i$ eine kompakte offene Menge.
Dann ist
\begin{align*}
X=\widehat{\prod_{i \in I}}\vphantom{\prod}^{K_i} X_i
\end{align*}
ausgestattet mit der eingeschränkten Produkttopologie ein lokalkompakter Hausdorffraum.
\end{prop}

\begin{proof}
Sei $x=(x_i)_{i \in I} \in X$. Definiere $J=\{i \in I \mid x_i \in K_i\}$. Dann ist $I\setminus J$ endlich. Für jedes $i \in I\setminus J$ wählen wir eine kompakte Umgebung $U_i$ von $x_i$.
Dann ist
\begin{align*}
\prod_{i\in I\setminus J} U_i \times \prod_{i\in J} K_i
\end{align*}
eine kompakte Umgebung von $x$.
Die Hausdorffeigenschaft folgt da, die $X_i$ Hausdorff sind.
\end{proof}

\begin{defi}
Der Ring
\begin{align*}
\mathbb{A}_f \coloneqq \widehat{\prod_{p<\infty}}\vphantom{\prod}^{\Z_p} \Q_p
\end{align*}
heißt der Ring der \emph{endlichen Adele}.
Nach dem letzten Satz ist $\mathbb{A}_f$ ein lokalkompakter Hausdorffraum.
\end{defi}


\begin{prop}
Der Ring $\mathbb{A}_f$ ist ein topologischer Ring, das heißt, die Ringverknüpfungen sind stetig.
\end{prop}

\begin{prop}
Die Menge
\begin{align*}
\hat{\Z}=\prod_{p <\infty} \Z_p \subseteq \mathbb{A}_f
\end{align*}
ist kompakt und offen.
\end{prop}
%TODO Beweis steht im Skript bzw ist leicht, per Definition bzw Tychonoff

%das ist Beweis für folgenden Satz
Sei $N\in \Z$ sowie $N>0$. Dann gilt
\begin{align*}
N=\prod_{p<\infty} p^{\nu_p}
\end{align*}
und $\nu_p=0$ für alle bis auf endlich viele $p$.
Weiterhin ist
\begin{align*}
N \hat{\Z}=\prod_{p <\infty} p^{\nu_p}\Z_p
\end{align*}
eine kompakte offene Umgebung der $0 \in \mathbb{A}_f$.
Eine beliebige offene Umgebung der $0 \in \mathbb{A}_f$ ist eine Vereinigung von Mengen der Form
\begin{align*}
\prod_{i \in I\setminus J} U_i \times \prod_{j \in J} K_j
\end{align*}
wobei $I\setminus J$ endlich ist und die $U_i$ für $i \in I \setminus J$ offen sind.
In $\Q_p$ gilt $K(0,p^m)=p^{-m}\Z_p$ für alle $m \in \Z$.

\begin{prop}
Jede offene Umgebung der $0\in \mathbb{A}_f$ enthält eine Menge der Form $N \hat{Z}$ für ein $N \in \Z$ mit $N>0$.
\end{prop}

\begin{prop}
Die Einbettung $\Q \hookrightarrow \mathbb{A}_f$ hat dichtes Bild.
\end{prop}

\begin{defi}
Der Ring $\mathbb{A} \coloneqq \R \times \mathbb{A}_f$ heißt \emph{Ring der Adele}.
\end{defi}

\begin{bem}
$\mathbb{A}$ ist ein lokalkompakter Hausdorffraum.
\end{bem}

\begin{prop}
$\mathbb{A}$ ist ein topologischer Ring.
Die Abbildung
\begin{align*}
Q \to \mathbb{A}\\
a &\mapsto (a,a,\dots)
\end{align*}
liefert eine Einbettung von $\Q$ nach $\mathbb{A}$.
\end{prop}

\begin{prop}
$\Q$ ist eine diskrete abgeschlossene Untergruppe von $\mathbb{A}$.
\end{prop}
\begin{proof}
Wir müssen zeigen, dass $\mathbb{A}$ die diskrete Topologie auf $\Q$ erzeugt, das heißt für jedes $a\in \Q$ existiert eine offene Umgebung $U$ mit $\Q \cap U=\{a\}$.
Aufgrund der Stetigkeit der Addition können wir ohne Beschränkung der Allgemeinheit $a=0$ annehmen.
Wähle
\begin{align*}
U\coloneqq [-\frac{1}{2},\frac{1}{2}] \times \prod_{p <\infty} \Z_p.
\end{align*}
Für $r \in \Q \cap U$ gilt
\begin{align*}
\abs{r}_p \leq 1
\end{align*}
für alle $p<\infty$.
Also folgt $r\in \Z$ und somit $r=0$.\\
Es bleibt zu zeigen, dass $\Q$ abgeschlossen in $\mathbb{A}$ ist.
Sei dazu $a\in \overline{\Q}$. Wir nehmen $a \not \in \Q$ an.
Es existiert eine offene Menge $U \subseteq \mathbb{A}$ mit $U \cap \Q=\{0\}$.
Wähle eine offene Umgebung $V$ von $0\in \mathbb{A}$, die $V+V \subseteq U$ und $-V=V$ erfüllt.
Die Stetigkeit der Addition impliziert, dass es eine offene Umgebung $W$ von $0$ gibt, die $W+W\subseteq U$ erfüllt. Wenn man $V=W \cap (-W)$ setzt, erhält man die gewünschte Menge.
Dann gilt $(a+V)\cap \Q\not=\emptyset$.
Sei $x\in (a+V)\cap \Q$. Dann $x\not =a$.
Da $\mathbb{A}$ Hausdorff ist, existiert eine offene Umgebung $W$ von $a$ mit $x\not \in W$.
Dann ist $(a+V)\cap W$ eine offene Umgebung von $a$.
Wähle $y \in (a+V)\cap W \cap \Q$.
Dann gilt
\begin{align*}
x-y\in ((a+V)\cap \Q)-((a+V)\cap W \cap Q.
\end{align*}
Dies impliziert $x-y \in  (V-V)\cap \Q$ also $x-y\in (U\cap \Q)=\{0\}$.
Aber $x=y$ ist ein Widerspruch.
\end{proof}
%hier verwendet man für die nichtleerheit der schnitte, dass a im Abschluss von \Q liegt.


Wir betrachten die kanonische Projektion
\begin{align*}
\pi \colon \A \to \A/\Q
\end{align*}
und statten $\A/\Q$ mit der Quotiententopologie aus, das heißt $U \subseteq \A/\Q$ ist genau dann offen wenn $\pi^{-1}(U)$ offen in $\A$ ist.
Insbesondere ist $\pi$ stetig.
\begin{prop}
Die kanonische Projektion ist offen.
\end{prop}
\begin{proof}
Sei $U\subseteq \mathbb{A}$ offen. Wir müssen zeigen, dass $\pi^{-1}(\pi(U))$ offen ist.
Es gilt
\begin{align*}
\pi^{-1}(\pi(U))&=\{a \in \mathbb{A} \mid \pi(a)\in \pi(U)\}\\
&=\bigcup_{a \in U} (a+\Q)\\
&=\bigcup_{x\in \Q} (x+U),
\end{align*}
somit ist die Offenheit gezeigt.
%TODO warum gilt die letzte Gleichheit.
\end{proof}

\begin{prop}
$\A/\Q$ ist ein kompakter Hausdorffraum.
\end{prop}
\begin{proof}
Zunächst zeigen wir, dass $\A/\Q$ Hausdorff ist.
Sei dafür $a+\Q\not = b+\Q$. Dann gilt $a-b \in \mathbb{A}\setminus \Q$, welches eine offene Menge ist.
Dann existiert eine offene symmetrische Umgebung der $0$ mit
\begin{align*}
((a-b)+U+U)\cap \Q=\emptyset.
\end{align*}
%TODO wieso? wegen Offenheit
Somit folgt
\begin{align*}
((a+U)-(b+U))\cap \Q &= \emptyset\\
((a-b)+U)\cap (Q+ U)&=\emptyset\\
(a+U)\cap (b+U+\Q)&=\emptyset\\
(a+U+\Q)\cap (b+U+\Q)&=\emptyset.
\end{align*}
Daher sind $(a+U)$ und $(b+U)$ disjunkt in $\A/\Q$ sind.
Es bleibt die Kompaktheit zu zeigen.
Dafür zeigen wir, dass es eine kompakte Teilmenge $K \subseteq \A$ mit $\pi(K)=\A$ gibt.
Sei $K\coloneqq [0,1]\times \hat{\Z}\subseteq \A$.
Wir zeigen, dass jede Restklasse in $\A/\Q$ einen Repräsentanten in $K$ besitzt.
Sei $a=(a_\nu)_{\nu \leq \infty} \in \A$. Für $p<\infty$ schreiben wir
\begin{align*}
a_p=\sum_{n=n_p} a_p(n)p^n
\end{align*}
Dann gilt $n_p=0$ für fast alle $p$.
Definiere
\begin{align*}
b\coloneqq a- \sum_{p<\infty}\sum_{n=n_p}^{-1} a_p(n)p^n,
\end{align*}
wobei der zweite Summand ein Element von $\Q \subseteq \A$ ist.
Dann gilt $b \in \R \times \A$, da
\begin{align*}
\abs{a_p-\sum_{q<\infty} \sum_{n=n_q}^{-1} a_q(n)q^n}_p&
=\abs{\sum_{n=0}^\infty a_p(n)p^n-\sum_{\infty>q\not =p} \sum_{n=n_q} a_q(n)q^n}_p\\
&=\max \left(\abs{\sum_{n=0}^\infty a_p(n)p^n}_p,\abs{\sum_{\infty>q\not =p} \sum_{n=n_q}^{-1} a_q(n)q^n}_p\right) \leq 1.
\end{align*}
%TODO schön untereinander schreiben
Indem wir $b$ um eine geeignete ganze Zahlen verschieben, erhalten wir einen Repräsentanten in $K$.
\end{proof}

Da $\A$ eine lokalkompakte Hausdorffgruppe ist, können wir das Haarmaß auf $\A$ betrachten, sodass $[0,1]\times \hat{\Z}$ Maß $1$ hat.

\begin{defi}
Eine \emph{einfache Funktion auf $\A$} ist eine Funktion der Form
\begin{align*}
f=\prod_{\nu \leq \infty} f_\nu,
\end{align*}
wobei $f_\nu \in \CC_c(\Q_\nu)$ und $f_p=\chi_{\Z_p}$ für fast alle $p<\infty$.
\end{defi}


\begin{bem}
Für $x=(x_\nu)_{\nu \leq \infty}\in \A$ gilt
\begin{align*}
f(x)=\prod_{\nu \leq \infty} f_\nu(x),
\end{align*}
wobei nur endlich viele Faktoren ungleich $1$ sind.
Die einfachen Funktionen auf $\A$ sind stetig und haben kompakten Träger (Notation $f \in \CC_c(\A)$).
\end{bem}

\begin{prop}
Sei $f=\prod_{\nu \leq \infty} f_\nu$ eine einfache Funktion auf $\A$.
Dann gilt
\begin{align*}
\int_A f(x)~\mathrm{d}x=\prod_{\nu \leq \infty} \int_{\Q_p} f_\nu(x)~\mathrm{d}x_\nu.
\end{align*}
\end{prop}

\begin{proof}
Betrachte die endliche Menge $I \coloneqq \{p <\infty \mid f_p \not = \chi_{\Z_p}\} \cup \{\infty\}$.
Definiere $\A_I\coloneqq\prod_{\nu \in I}\Q_\nu$.
Dann gilt $\A=\A_I \times \A^I$, wobei
\begin{align*}
\A^I \coloneqq\widehat{\prod_{p \in I}}~\vphantom{\prod}^{\Z_p} \Q_p
\end{align*}
ist.
%TODO wie will man damit umgehen, dass das p "zu weit" nach links geht und dadurch der hat so riesig wird.
$\A_I$ ist als endliches Produkt lokalkompakter Hausdorffräume wieder ein lokalkompakter Hausdorffraum.
Das Haarmaß auf $\A_I$ ist durch das Produkt der Haarmaße auf den Faktoren gegeben und der Satz von Fubini gilt.
%TODO "das" haarmaß ist eher suboptimal.
Wir schreiben
\begin{align*}
f=\prod_{\nu \in I} f_\nu \cdot \prod_{p \not \in I} f_p.
\end{align*}
%TODO erster Faktor f_I und zweiter f^I
Mit Fubini gilt somit
\begin{align*}
\int_Af(x)~\mathrm{d}x&=\int_{\A_I}f_I(x)~\mathrm{d}x \int_{\A^I}f^I(x)~\mathrm{d}x\\
&=\prod_{\nu \in I} \int_{\Q_\nu} f_\nu(x)~\mathrm{d}x\\
&=\prod_{\nu \leq \infty} \int_{\Q_\nu} f_\nu(x)~\mathrm{d}x,
\end{align*}
was zu zeigen war.
\end{proof}

Sei $x\in \Z_p$. Dann existiert für jedes $n \in \Z, n>0$ genau ein $\alpha_n \in \Z$, $0 \leq \alpha_n<p^n$ mit
\begin{align*}
\abs{x-\alpha_n}_p \leq p^{n-}\\
\Leftrightarrow \alpha_n \in x+K(0,p^{-n})\\
\Leftrightarrow \alpha_n \in x+p^n \Z_p.
\end{align*}
%TODO formatierung schoen machen

\begin{thm}
Sei nun $x\in \hat{\Z}=\prod_{\nu \leq \infty} \Z_p$. Dann existiert für jedes $N \in \Z$, $N>0$ existiert ein eindeutiges $\alpha_N\in \Z$, $0 \leq \alpha<N$ mit
\begin{align*}
\alpha_N \in x + N\hat{\Z}.
\end{align*}
\end{thm}
%nu statt p, da p nur für endliche primzahlen?! wie ist \hat{\Z} definiert oben nur für endliche primzahlen sonst keine Teilmenge von \A_f
\begin{proof}
Sei $x=(x_p)_{p\leq \infty}$ mit $x_p=\sum_{n=0}^\infty x_p(n)p^n$ sowie $N=\prod_{p\leq \infty} p^{\nu_p}$.
Dann gilt $\nu_p=0$ für fast alle $p$ und $N\hat{\Z}=\prod_{p<\infty}p^{\nu_p} \Z_p$.
Der Chinesische Restsatz existiert ein eindeutiges $\alpha_N \in \Z$, $0\leq \alpha <N$ mit 
\begin{align*}
\alpha_N=\sum_{n=0}^{\nu_p-1} x_p(n)p^n~~\mod p^{\nu_p}
\end{align*}
für alle $\nu_p\not =0$. Für dieses $\alpha_N$ gilt $\alpha_N \in x+N \hat{\Z}$.
Diese Approximation werden für wachsendes $N$ immer besser.
Insbesondere gilt $\alpha_M=\alpha_N ~\mod N$ für alle Vielfachen $M$ von $N$. Also definiert $x$ eine Folge ganzer Zahlen $(\alpha_N)_{N>0}$ mit $0\leq \alpha_N <N$ sowie $\alpha_M=\alpha_N \mod N$ falls $N\mid M$.
\end{proof}

Umgekehrt definiert eine solche Folge ein Element in $\hat{\Z}$.

\begin{defi}
Eine Folge von Restklassen $(\alpha_N)_{N>0}$ mit $\alpha_N\in\Z /N\Z$, die die Kompatibilitätsbedingung $\alpha_M=\alpha_N \mod N$ für $N \mid M$ erfüllt, heißt \emph{kompatibles System}.
Der projektive Limes
\begin{align*}
\lim_{\longleftarrow} \Z /N \Z=\left\lbrace(\alpha_N) \in \prod_{N>0} (\Z /N \Z \mid (\alpha_N)_{N>0} \text{ ist kompatibel}\right\rbrace
\end{align*}
ist auf natürliche Art und Weise ein Ring.
Wir können diesen mit der Topologie des projektiven Limes ausstatten.
%TODO was ist dieser Limes?!
\end{defi}

\begin{thm}
Die Abbildung
\begin{align*}
\hat{\Z}&\mapsto \lim_{\longleftarrow} \Z /N \Z\\
x&\mapsto (\alpha_N \mod N)
\end{align*}
ist ein Isomorphismus topologischer Ringe.
\end{thm}

Die Abbildung
\begin{align*}
\hat{\Z} &\to \Z /N\Z\\
x&\mapsto \alpha_N \mod N
\end{align*}
ist surjektiv mit Kern $N\hat{\Z}$. Also folgt
\begin{align*}
\hat{\Z}/N\hat{\Z}=\Z /N \Z.
\end{align*}