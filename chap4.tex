\chapter{Adele}
Wir wollen alle Vervollständigungen von $\Q$ gleichzeitig betrachten.
Dafür betrachten wir den lokal kompakten Ring 
\begin{align*}
\A_Q=\{(x_\infty,x_2,x_3,x_5\dots,)\mid x_\nu \in \Q_\nu \text{ und } x_p \in \Z_p \text{ a.e.}\}
\end{align*}
von Adelen.

Sei $(X_i)_{i\in I}$ eine Familie topologischer Räume.
Die Produkttopologie auf $\prod_{i \in I} X_i$ ist die gröbste Topologie auf $\prod_{i \in I} X_i$, bezüglich der alle Projektionen stetig sind.
Eine Basis für diese Topologie ist durch $\{\prod_{i\in I}O_i\}$ gegeben, wobei $O_i$ für alle $i \in I$ offen ist und $O_i=X_i$ für fast alle $i \in I$ gilt.

Mengen dieser Form heißen \emph{offene Rechtecke}.

Wir benötigen folgenden klassischen Satz aus der Topologie.
\begin{thm}[Tychonoff]
Sei $(X_i)_{i \in I}$ eine Familie von topologischen Räumen. Dann ist das Produkt der $X_i$ genau dann kompakt, wenn alle $X_i$ kompakt sind.
\end{thm}

\begin{thm}
Sei $(X_i)_{i \in I}$ eine Familie von Hausdorffräumen. Dann ist $\prod_{i \in I} X_i$ lokal kompakt genau dann wenn alle $X_i$ lokal kompakt und alle bis auf endlich viele sogar kompakt sind.
\end{thm}

\begin{proof}
\glqq $\Rightarrow$\grqq : Da die Projektionen $p_j \colon \prod_{i \in I} X_i \to X_j$ stetig sind, folgt, dass alle $X_i$ lokal kompakt sind.
Sei $x=(x_i)_{i\in I} \in \prod_{i \in I} X_i$ und $C=(C_i)_{i \in I}$ eine kompakte Umgebung von $x$.
$C$ enthält eine Vereinigung offener Rechtecke. Da alle bis auf endlich viele Komponenten eines offenen Rechtecks der ganze Raum sind, gilt das gleich auch für $C$, das heißt
\begin{align*}
\pi_i(C)=X_i
\end{align*}
für alle bis auf endlich viele $i \in I$.
\glqq $\Leftarrow$\grqq : Da mit $J=\{i\in I \mid x_i \text{ compact }\}$, die Menge $I \setminus J$ endlich ist, folgt mit
\begin{align*}
X=\prod_{i \in J} X_i \times \prod_{i \in I\setminus J}X_i,
\end{align*}
dass $X$ als Produkt einer kompakten und einer lokal kompakten Menge lokal kompakt ist.
\end{proof}

\begin{bsp}
$\prod_{p <\infty} \Q_p$ ist nicht lokal kompakt und deswegen wissen wir nicht, ob es ein Haarmaß gibt.
\end{bsp}
%TODO wissen wir wirklich dass es keins gibt?!

\begin{defi}
Sei $(x_i)_{ i \in I}$ eine Familie lokal kompakter Hausdorffräume und für jedes $i \in I$ sei $K_i\subseteq X_i$ eine kompakte offene Menge.
%TODO warum geht das immer?
Dann definieren wir das eingeschränkte Produkt
\begin{align*}
X=\hat{\prod}_{i\in I}^{K_i} X_i \coloneqq \{(x_i)_{i \in I} \in \prod_{i \in I} X_i \mid x_i \in K_i \text{ für fast alle } i \in I\}.
\end{align*}
Falls $K_i$ aus dem Kontext klar ist, lassen wir diese in der Schreibweise weg.
Ein eingeschränktes offenes Rechteck ist eine Menge der Form
\begin{align*}
\prod_{i\in I} U_i
\end{align*}
wobei $U_i \subseteq X_i$ offen ist und $K_i=U_i$ für alle bis auf endlich viele $i  \in I$ gilt.
Eine Teilmenge $U\subseteq X$ heißt offen, wenn es eine Vereinigung solcher Mengen ist.
Die dazugehörige Topologie heißt die \emph{eingeschränkte Produkttopologie}.
\end{defi}

\begin{prop}
Sei $(X_i)_{i\in I}$ eine Familie lokal kompakter Hausdorffräume.
Für jedes $i \in I$ sei $K_i\subseteq X_i$ eine kompakte offene Menge.
Dann ist
\begin{align*}
X=\hat{\prod}_{i \in I}^{K_i} X_i
\end{align*}
ausgestattet mit der eingeschränkten Produkttopologie ein lokal kompakter Hausdorffraum.
\end{prop}

\begin{proof}
Sei $x=(x_i)_{i \in I} \in X$. Definiere $J=\{i \in I \mid x_i \in K_i\}$. Dann ist $I\setminus J$ endlich. Für jedes $i \in I\setminus J$ wählen wir eine kompakte Umgebung $U_i$ von $x_i$.
Dann ist
\begin{align*}
\prod_{i\in I\setminus J} U_i \times \prod_{i\in J} K_i
\end{align*}
eine kompakte Umgebung von $x$.
Die Hausdorffeigenschaft verbleibt als Übung.
\end{proof}

\begin{defi}
Der Ring
\begin{align*}
\mathbb{A}_f \coloneqq \widehat{\prod_{p<\infty}}^{\Z_p} \Q_p
\end{align*}
heißt der Ring der \emph{endlichen Adele}.
Nach dem letzten Satz ist $\mathbb{A}_f$ ein lokal kompakter Hausdorffraum.
\end{defi}
%TODO A_f schoen machen, Problem: ds gibt doppelstrich auf falscher Seite...

\begin{prop}
Der Ring $\mathbb{A}_f$ ist ein topologischer Raum, das heißt, die Ringverknüpfungen sind stetig.
\end{prop}

\begin{prop}
Die Menge
\begin{align*}
\hat{\Z}=\prod_{p <\infty} \Z_p \subseteq \mathbb{A}_f
\end{align*}
ist kompakt und offen.
\end{prop}

Sei $N\in \Z$ sowie $N>0$. Dann gilt
\begin{align*}
N=\prod_{p<\infty} p^{\nu_p}
\end{align*}
und $\nu_p=0$ für alle bis auf endlich viele $p$.
Weiterhin haben wir, dass
\begin{align*}
N \hat{\Z}=\prod_{p <\infty} p^{\nu_p}\Z_p
\end{align*}
eine kompakte offene Umgebung der $0 \in \mathbb{A}_f$.
Eine beliebige offene Umgebung der $0 \in \mathbb{A}_f$ ist eine Vereinigung von Mengen der Form
\begin{align*}
\prod_{i \in I\setminus J} U_i \times \prod_{j \in J} K_j
\end{align*}
wobei $I\setminus J$ endlich ist und die $U_i$ für $i \in I \setminus J$ offen sind.
In $\Q_p$ gilt $K(0,p^m)=p^{-m}\Z_p$ für alle $m \in \Z$.

\begin{prop}
Jede offene Umgebung der $0\in \mathbb{A}_f$ enthält eine Menge der Form $N \hat{Z}$ für ein $N \in \Z$ mit $N>0$.
\end{prop}

\begin{prop}
Die Einbettung $\Q \hookrightarrow \mathbb{A}_f$ hat dichtes Bild.
\end{prop}