\chapter{Modulare Formen für $\SL_2(\Z)$}

Nun können wir modulare Formen definieren, dies sind sehr besondere Funktionen.
Erinnerung: $\Gamma=\SL_2(\Z)$ operiert auf der oberen Halbebene $H$ durch
\begin{align*}
\begin{pmatrix}
a&b\\
c&d
\end{pmatrix} z=\frac{az+b}{cz+d}.
\end{align*}

\begin{defi}
Sei $k \in \Z$. Eine meromorphe Funktion $f \colon H \to \C$ heißt
\emph{meromorphe modulare Form von Gewicht $k$} für $\Gamma$ falls
\begin{enumerate}[label=\arabic*)]
\item $f(M \tau)=(c\tau +d)^k f(\tau)$ gilt für alle $M =\begin{pmatrix}
a&b\\
c&d
\end{pmatrix} \in \Gamma$.
\item $f$ ist meromorph im unendlichen, das heißt, es $f$ besitzt
eine Fourierentwicklung der Form
\begin{align*}
f(\tau)=\sum_{n \in \Z} a_n q^n,
\end{align*}
wobei $q=\mathrm{e}^{2\pi \mathrm{i} \tau}$ und $a_n=0$ für $n$ hinreichend klein.
\end{enumerate}
\end{defi}

Tatsächlich kann man die Voraussetzung 1) abschwächen.
Man muss dies nicht für alle Matrizen in $\SL_2(Z)$ testen,
dies würde sehr lange dauern, es reicht sich auf die Erzeuger zu beschränken.
Dies ist nicht vollkommen offensichtlich.
\begin{bem}
\begin{enumerate}[label=\arabic*)]
\item Da $\Gamma$ von $S$ und $T$ erzeugt wird, ist es ausreichend,
die Transformationseigenschaft für $S$ und $T$ zu überprüfen.
%TODO Beweis?!
\item Die Abbildung
\begin{align*}
H&\to \{q \in \C \mid 0<\abs{q}<1\}\\
\tau &\mapsto \mathrm{e}^{2 \pi \mathrm{i}\tau}
\end{align*}
ist holomorph und surjektiv.
Da $f$ wegen 1) für $T$ $1$-periodisch ist, ist die Funktion
\begin{align*}
g(q)\coloneqq f\left( \frac{\log(q)}{2 \pi \mathrm{i}} \right)
\end{align*}
wohldefiniert und meromorph.
Falls $g$ zu einer meromorphen Funktion auf dem offenen Einheitsball fortgesetzt werden kann,
so besitzt $g$ eine Laurententwicklung der Form
\begin{align*}
g(q)=\sum_{n\in \Z} a_nq^n
\end{align*}
in einer Umgebung der $0$ mit $a_n=0$ für hinreichend kleines $n$. Dann gilt
\begin{align*}
f(\tau)=\sum_{n\in \Z} a_n \mathrm{e}^{2 \pi \mathrm{i} \tau n}
\end{align*}
mit $a_n=0$ für hinreichend kleines $n$.
In diesem Fall sagen wir, dass $f$ meromorph im unendlichen ist.
Die Fourierkoeffizienten sind durch
\begin{align*}
a_n =\int_w^{w+1} f(\tau)\mathrm{e}^{-2\pi \mathrm{i} n \tau}~\mathrm{d}\tau
\end{align*}
für beliebiges $w$ mit möglicherweise hinreichendem großem Imaginärteil (sodass $g$ im korrespondieren Ball keine Pole hat).
\end{enumerate}
\end{bem}
%TODO qs ordentlich machen

\begin{defi}
Eine holomorphe Funktion heißt \emph{holomorphe modulare Form}, falls
\begin{enumerate}[label=\arabic*)]
\item $f(M\tau)=(c\tau +d)^k f(\tau)$ für alle $M\begin{pmatrix}
a&b\\
c&d
\end{pmatrix} \in \Gamma$.
\item $f$ holomorph im unendlichen ist, das heißt $f$ hat eine Fourierentwicklung der Form
\begin{align*}
f(\tau)=\sum_{n=0}^\infty a_n q^n.
\end{align*}
\end{enumerate}
Falls zusätzlich $a_0=0$ gilt, dann heißt $f$ \emph{Spitzenform}.
\end{defi}

Der Raum aller modularer Formen von Gewicht $k$ wird mit $M_k$
bezeichnet und der Raum der Spitzenform mit $S_k$.
\begin{thm}[Hecke]
Sei $f(\tau)=\sum_{n=1}^\infty a_n q^n$ einen Spitzenform von
Gewicht $k>0$.
Dann gilt
\begin{align*}
\abs{a_n} \leq c n^{k/2}.
\end{align*}
\end{thm}
Diese Schranke heißt \emph{triviale Schranke}. Man kann diese Schranke noch verbessern, dafür gab es die Fields-Medaille und den Abel-Preis.
%TODO Name raussuchen.