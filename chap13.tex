\chapter{Modulare Formen für $\SL_2(\Z)$}

Nun können wir modulare Formen definieren, dies sind sehr besondere Funktionen.
Erinnerung: $\Gamma=\SL_2(\Z)$ operiert auf der oberen Halbebene $H$ durch
\begin{align*}
\begin{pmatrix}
a&b\\
c&d
\end{pmatrix} z=\frac{az+b}{cz+d}.
\end{align*}

\begin{defi}
Sei $k \in \Z$. Eine meromorphe Funktion $f \colon H \to \C$ heißt
\emph{meromorphe modulare Form von Gewicht $k$} für $\Gamma$ falls
\begin{enumerate}[label=\arabic*)]
\item $f(M \tau)=(c\tau +d)^k f(\tau)$ gilt für alle $M =\begin{pmatrix}
a&b\\
c&d
\end{pmatrix} \in \Gamma$.
\item $f$ ist meromorph im unendlichen, das heißt, es $f$ besitzt
eine Fourierentwicklung der Form
\begin{align*}
f(\tau)=\sum_{n \in \Z} a_n q^n,
\end{align*}
wobei $q=\mathrm{e}^{2\pi \mathrm{i} \tau}$ und $a_n=0$ für $n$ hinreichend klein.
\end{enumerate}
\end{defi}

Tatsächlich kann man die Voraussetzung 1) abschwächen.
Man muss dies nicht für alle Matrizen in $\SL_2(Z)$ testen,
dies würde sehr lange dauern, es reicht sich auf die Erzeuger zu beschränken.
Dies ist nicht vollkommen offensichtlich.
\begin{bem}
\begin{enumerate}[label=\arabic*)]
\item Da $\Gamma$ von $S$ und $T$ erzeugt wird, ist es ausreichend,
die Transformationseigenschaft für $S$ und $T$ zu überprüfen.
%TODO Beweis?!
\item Die Abbildung
\begin{align*}
H&\to \{q \in \C \mid 0<\abs{q}<1\}\\
\tau &\mapsto \mathrm{e}^{2 \pi \mathrm{i}\tau}
\end{align*}
ist holomorph und surjektiv.
Da $f$ wegen 1) für $T$ $1$-periodisch ist, ist die Funktion
\begin{align*}
g(q)\coloneqq f\left( \frac{\log(q)}{2 \pi \mathrm{i}} \right)
\end{align*}
wohldefiniert und meromorph.
Falls $g$ zu einer meromorphen Funktion auf dem offenen Einheitsball fortgesetzt werden kann,
so besitzt $g$ eine Laurententwicklung der Form
\begin{align*}
g(q)=\sum_{n\in \Z} a_nq^n
\end{align*}
in einer Umgebung der $0$ mit $a_n=0$ für hinreichend kleines $n$. Dann gilt
\begin{align*}
f(\tau)=\sum_{n\in \Z} a_n \mathrm{e}^{2 \pi \mathrm{i} \tau n}
\end{align*}
mit $a_n=0$ für hinreichend kleines $n$.
In diesem Fall sagen wir, dass $f$ meromorph im unendlichen ist.
Die Fourierkoeffizienten sind durch
\begin{align*}
a_n =\int_w^{w+1} f(\tau)\mathrm{e}^{-2\pi \mathrm{i} n \tau}~\mathrm{d}\tau
\end{align*}
für beliebiges $w$ mit möglicherweise hinreichendem großem Imaginärteil (sodass $g$ im korrespondieren Ball keine Pole hat).
\end{enumerate}
\end{bem}
%TODO qs ordentlich machen

\begin{defi}
Eine holomorphe Funktion heißt \emph{holomorphe modulare Form} oder auch \emph{Modulform}, falls
\begin{enumerate}[label=\arabic*)]
\item $f(M\tau)=(c\tau +d)^k f(\tau)$ für alle $M=\begin{pmatrix}
a&b\\
c&d
\end{pmatrix} \in \Gamma$.
\item $f$ holomorph im unendlichen ist, das heißt, $f$ hat eine Fourierentwicklung der Form
\begin{align*}
f(\tau)=\sum_{n=0}^\infty a_n q^n
\end{align*}
für $q=\mathrm{e}^{2\pi \mathrm{i} \tau}$.
\end{enumerate}
Falls zusätzlich $a_0=0$ gilt, dann heißt $f$ \emph{Spitzenform}.
\end{defi}

Der Raum aller modularer Formen von Gewicht $k$ wird mit $M_k$
bezeichnet und der Raum der Spitzenform mit $S_k$.
\begin{thm}[Hecke]
Sei $f(\tau)=\sum_{n=1}^\infty a_n q^n$ einen Spitzenform von
Gewicht $k>0$.
Dann gilt
\begin{align*}
\abs{a_n} \leq c n^{k/2}.
\end{align*}
\end{thm}
Diese Schranke heißt \emph{triviale Schranke}. Man kann diese Schranke noch verbessern, dafür gab es die Fields-Medaille und den Abel-Preis.
%TODO Name raussuchen.
\begin{proof}
Mit
\begin{align*}
h(\tau)=\abs{f(\tau)} \Imt(\tau)^{k/2}
\end{align*}
gilt
\begin{align*}
h(M\tau)=h(\tau)
\end{align*}
für alle $M \in \Gamma$.
%TODO wieso ist das so?
Außerdem gilt
\begin{align*}
f(\tau)=\sum_{n=1}^\infty a_n q^n=q\sum_{n=1}^\infty a_n q^{n-1}.
\end{align*}
Deshalb ist $\mathrm{e}^{-2\pi \mathrm{i}t} f(\tau)$ für alle $\gamma>0$ auf
dem Gebiet $\{\gamma \in H \mid \Imt(\tau)\geq \gamma\}$ beschränkt.
%TODO wieso?
Also ist
\begin{align*}
\abs{f(\tau) \mathrm{e}^{-2\pi \mathrm{i}t}} \leq \abs{f(\tau)} \mathrm{e}^{2\pi y}
\end{align*}
%TODO y einfach aufgrund der schreibweise z=x+iy
und
\begin{align*}
h(T))\abs{f(\tau)}y^{k/2}
%TODO \tau?
\end{align*}
auf $\overline{\mathbb{D}}$ beschränkt.
Da $h$ invariant unter $\Gamma$ ist, gilt
\begin{align*}
0 \leq h(\tau)\leq c'
\end{align*}
für ein $c'>0$ und alle $\tau \in H$.
Nun schätzen wir die Fourierkoeffizienten ab
\begin{align*}
\abs{a_n}&=\abs{\int_0^1 f(x+iy) \ex^{-2\pi \mathrm{i} n(x+\mathrm{i}}} ~\mathrm{d}x\\
&\leq \ex^{2\pi ny} \int_0^1 \abs{f(x+\mathrm{i}y)}~\mathrm{d}x\\
&\leq \ex^{2\pi ny} y^{-k/2} \int_0^1 h(x+\mathrm{i}y)~\mathrm{d}x\\
&\leq c' y^{-k/2} \mathrm{e}^{2\pi ny}.
\end{align*}
%TODO ah das ist der parametrisierte Weg schon eingesetzt, das möchte man tendenziell hinschreiben, 1/n sollte groß genug sein, da wir holomorph sind.
Wenn wir $y=\frac{1}{n}$ wählen, erhalten wir die Behauptung.
\end{proof}

\begin{bem}
Eine Konsequenz des Beweises von Deligne\footnote{\url{http://www.numdam.org/article/SB_1968-1969__11__139_0.pdf}} der Ramanujan-Petersson Vermutung \footnote{\url{https://en.wikipedia.org/wiki/
Ramanujan-Petersson_conjecture}}
ist
\begin{align*}
\abs{a_n} \leq c(\varepsilon) n^{\frac{k-1}{2}+\varepsilon}
\end{align*}
für alle $\varepsilon>0$.
\end{bem}

\begin{defi}
Die einfachsten Beispiele für Modulformen sind die \emph{Eisensteinreihen}\footnote{\url{https://de.wikipedia.org/wiki/Gotthold_Eisenstein}}
\begin{align*}
G_k(\tau)\coloneqq \sum_{\substack{(m,n)\in \Z^2\\ (m,n)\not = (0,0)}} \frac{1}{(m\tau +n )^k}.
\end{align*}
\end{defi}

Zum Beweis der Wohldefiniertheit der Eisensteinreihen benötigen wir
das folgende Lemma.

\begin{lem}
Sei $K\subseteq H$ kompakt.
Dann gibt es Konstanten $\gamma,\delta>0$, sodass
\begin{align*}
\gamma \abs{m\mathrm{i}+n} \leq \abs{m\tau +n} \leq \delta \abs{m\mathrm{i}+n}
\end{align*}
für alle $m,n \in \R$ und $\tau \in K$ gilt.
\end{lem}
\begin{proof}
Dies verbleibt als Übung.
Sollte sofort aus Stetigkeit und Positivität der mittleren Funktion folgen.
\end{proof}

\begin{thm}
Sei $k\geq 3$. Dann konvergiert die Eisensteinreihe $G_k$
absolut und lokal gleichmäßig.
Insbesondere ist $G_k$ holomorph auf $H$.
\end{thm}
\begin{proof}
Aufgrund des vorherigen Lemmas reicht es zu zeigen, dass
\begin{align*}
\sum_{\substack{(mn,n)\in \Z^2\\ (m,n)\not =(0,0)}} \frac{1}{(m^2+n^2)^\alpha}
\end{align*}
für $\alpha>1$ konvergiert. An dieser Stelle ist $k\geq 3$ wichtig,
da man sonst $\alpha\leq 1$ erhalten würde.
Im weiteren betrachten wir die Reihe für eine endliche Teilmenge $E \subseteq \Z^2$ und schätzen diese unabhängig von der Menge ab.
\end{proof}

\begin{bem}
Da die Reihen absolut konvergieren, können wir die Summationsreihenfolge der $G_k$ vertauschen,
ohne dass sich der Wert ändert.
\end{bem}

Als nächstes zeigen wir, dass die $G_k$ wirklich von Gewicht $k$ sind.

\begin{thm}
Für $k\geq 3$ gilt
\begin{align*}
G_k(M\tau)=(c\tau +d)^k G_k(\tau).
\end{align*}
\end{thm}
\begin{proof}
rechnen, wird nachgereicht.
\end{proof}

\begin{prop}
Für $\tau \in H$ und $k\geq 2$ gilt
\begin{align*}
\sum_{n \in \Z} \frac{1}{(\tau+n)^k}=\frac{(-2\pi \mathrm{i})^k}{(k-1)!}  \sum_{n=1}^\infty n^{k-1} q^n
\end{align*}
mit $q=\mathrm{e}^{2\pi \mathrm{i} \tau}$.
\end{prop}

Nun zeigen wir die Fourierreihendarstellung, was wir benötigen, um zu zeigen, dass Eisensteinreihen Modulformen definieren.
\begin{thm}
Für alle $\tau \in H$ und $k \in 2\Z, k\geq 4$ gilt
\begin{align*}
G_k(\tau)=2 \zeta(k)+2\frac{(2\pi \mathrm{i})^k}{(k-1)!} \sum_{n=1}^\infty \sigma_{k-1}(n)q^n
\end{align*}
mit $\sigma_{k-1}(n)=\sum_{d \mid n} k^{d-1}$.
Insbesondere gilt $G_k \in M_k$.
\end{thm}

\begin{defi}
Die \emph{normalisierte Eisensteinreihe} wird als
\begin{align*}
E_k\coloneqq  \frac{1}{2\zeta(k)} G_k
\end{align*}
definiert.
Für $k \in 2\Z, k\geq 4$ gilt
\begin{align*}
2 \zeta(k)=-\frac{(2\pi \mathrm{i})^k}{k!}B_k
\end{align*}
wobei $B_k$ Bernoullizahlen\footnote{\url{https://de.wikipedia.org/wiki/Bernoulli-Zahl}} sind.
Also gilt
\begin{align*}
E_k(\tau)=1-\frac{2k}{B_k} \sum_{n=1}^\infty \sigma_{k-1}(n)q^n.
\end{align*}
Man kann außerdem zeigen, dass
\begin{align*}
E_k(\tau)=\frac{1}{2} \sum_{(m,n)=1} \frac{1}{(m\tau +n)^k}
\end{align*}
gilt.
\end{defi}

\begin{defi}
Sei $k\in \Z$.
Für eine meromorphe Funktion $f \colon H \to \C \cup \{\infty\}$
definieren wir eine Operation von $M=\begin{pmatrix}
a&b\\
c&d
\end{pmatrix} \in \GL_2(\R^+)$
durch
\begin{align*}
(f\mid_{k,M})(\tau)=\det(M)^{k/2}(c\tau +d)^{-k} f(M\tau).
\end{align*}
Dann gilt für $M,N \in \GL_2(\R)$
%TODO muss da noch ein plus hin?!
\begin{align*}
(f\mid_{k,M})\mid_{k,N} =f \mid_{k,MN}.
\end{align*}
Der Index wird oft weggelassen, da wenn $M \in \Gamma$ auch
\begin{align*}
f\mid_M=f
\end{align*}
gilt.
%TODO warum steht nur M und kein k dabei?
\end{defi}

\begin{defi}
Sei $f$ eine meromorphe Funktion auf $H$.
Die Ordnung von $f$ in $\omega \in H$ ist definiert als die Zahl $n$, sodass
\begin{align*}
\frac{f(\tau)}{(\tau-\omega)^n}
\end{align*}
holomorph und ungleich $0$ in $\omega$ ist.
Wir schreiben $n=\nu_\omega(f)$.
Wenn $f$ eine meromorphe Modulform ist, dann gilt
\begin{align*}
\nu_\omega(f)=\nu_{M\omega}(f)
\end{align*}
für alle $M \in \Gamma$.
Wenn $f(\tau)=\sum_{n\geq n_0} a_nq^n$ mit $a_{n_0}\not =0$ ist, dann gilt $\nu_\infty(f)=n_0$.
\end{defi}

\begin{thm}[$\frac{k}{12}$-Formel, Gewichtsformel]
Sei $f\not =0$ eine meromorphe Modulform von Gewicht $k$.
Dann hat $f$ nur endlich viele Polstellen und Nullstellen modulo $\Gamma$ und es gilt
\begin{align*}
\nu_\infty(f)+\sum_{\tau \in \Gamma \ H} \frac{1}{\frac{1}{2}\abs{G_t}} \nu_t(f)=\frac{k}{12}.
\end{align*}
%TODO was heißt das?, \Gamma modulo H 
\end{thm}
\begin{proof}
rechnen und cauchyintegral formel + Residuensatz.
\end{proof}

\begin{bem}
Die Transformationsformel impliziert, dass $M_k=\{0\}$, wenn $k$ ungerade ist.
Sei $0 \not =f \in M_k$
\end{bem}

\begin{prop}
\begin{enumerate}[label=(\roman*)]
\item $M_k=\{0\}$ für $k<0$ oder $k=2$.
\item $M_k=\C$ für $k=0$.
\item $M_k=\C E_k$ für $k=4,6,8,10,14$.
\end{enumerate}
\end{prop}

\begin{defi}
Wir definieren
\begin{align*}
\bigtriangleup(\tau)=\frac{1}{1728} (E_4(\tau)^3-E_6(\tau)^2)\not =0.
\end{align*}
Dann ist $\bigtriangleup\in S_{12}$, $\nu_\infty(\bigtriangleup)=1$ und $\bigtriangleup(\tau)\not=0$ für alle $\tau \in H$.
\end{defi}

\begin{prop}
\begin{enumerate}[label=(\roman*)]
\item $S_k=\{0\}$ für $k<12$ oder $k=14$.
\item $S_k=\bigtriangleup M_{k-12}$ für $k>14$.
\item $M_k=\C E_k\oplus S_k$ für $k>2$.
\end{enumerate}
\end{prop}

\begin{prop}
Sei $0\leq k\in 2\Z$. Dann gilt
\begin{align*}
\dim M_k=\begin{cases}
\left[\frac{k}{12}\right],~~&\text{wenn } k \equiv 2 \mod 12\\
\left[\frac{k}{12}\right]+1,&\text{ wenn } k \not \equiv 2 \mod 12.
\end{cases}
\end{align*}
\end{prop}

\begin{thm}
Die Abbildung $\C[X,Y]\to M$ definiert durch
\begin{align*}
X &\mapsto E_4\\
Y &\mapsto E_6
\end{align*}
ist ein Isomorphismus von Ringen mit $M =\bigoplus_{k=0}^\infty M_k$.
Die \emph{Hilbertfunktion} ist durch
\begin{align*}
\sum_{k=0}^\infty \dim(M_k)t^k&=\frac{1}{(1-t^4)(1-t^6}\\
&=\left(\sum_{n=0}^\infty t^{4n}\right) \left(\sum_{m=0}^\infty t^{6m}\right)
\end{align*}
definiert.
\end{thm}
\begin{proof}
Der Beweis verbleibt als Übung.
\end{proof}