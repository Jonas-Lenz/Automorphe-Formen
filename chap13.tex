\chapter{Modulare Formen für $\SL_2(\Z)$}

Nun können wir modulare Formen definieren, dies sind sehr besondere Funktionen.
Erinnerung: $\Gamma=\SL_2(\Z)$ operiert auf der oberen Halbebene $H$ durch
\begin{align*}
\begin{pmatrix}
a&b\\
c&d
\end{pmatrix} z=\frac{az+b}{cz+d}.
\end{align*}

\begin{defi}
Sei $k \in \Z$. Eine meromorphe Funktion $f \colon H \to \C$ heißt
\emph{meromorphe modulare Form von Gewicht $k$} für $\Gamma$ falls
\begin{enumerate}[label=\arabic*)]
\item $f(M \tau)=(c\tau +d)^k f(\tau)$ gilt für alle $M =\begin{pmatrix}
a&b\\
c&d
\end{pmatrix} \in \Gamma$.
\item $f$ ist meromorph im unendlichen, das heißt, es $f$ besitzt
eine Fourierentwicklung der Form
\begin{align*}
f(\tau)=\sum_{n \in \Z} a_n q^n,
\end{align*}
wobei $q=\mathrm{e}^{2\pi \mathrm{i} \tau}$ und $a_n=0$ für $n$ hinreichend klein.
\end{enumerate}
\end{defi}

Tatsächlich kann man die Voraussetzung 1) abschwächen.
Man muss dies nicht für alle Matrizen in $\SL_2(Z)$ testen,
dies würde sehr lange dauern, es reicht sich auf die Erzeuger zu beschränken.
Dies ist nicht vollkommen offensichtlich.
\begin{bem}
\begin{enumerate}[label=\arabic*)]
\item Da $\Gamma$ von $S$ und $T$ erzeugt wird, ist es ausreichend,
die Transformationseigenschaft für $S$ und $T$ zu überprüfen.
%TODO Beweis?!
\item Die Abbildung
\begin{align*}
H&\to \{q \in \C \mid 0<\abs{q}<1\}\\
\tau &\mapsto \mathrm{e}^{2 \pi \mathrm{i}\tau}
\end{align*}
ist holomorph und surjektiv.
Da $f$ wegen 1) für $T$ $1$-periodisch ist, ist die Funktion
\begin{align*}
g(q)\coloneqq f\left( \frac{\log(q)}{2 \pi \mathrm{i}} \right)
\end{align*}
wohldefiniert und meromorph.
Falls $g$ zu einer meromorphen Funktion auf dem offenen Einheitsball fortgesetzt werden kann,
so besitzt $g$ eine Laurententwicklung der Form
\begin{align*}
g(q)=\sum_{n\in \Z} a_nq^n
\end{align*}
in einer Umgebung der $0$ mit $a_n=0$ für hinreichend kleines $n$. Dann gilt
\begin{align*}
f(\tau)=\sum_{n\in \Z} a_n \mathrm{e}^{2 \pi \mathrm{i} \tau n}
\end{align*}
mit $a_n=0$ für hinreichend kleines $n$.
In diesem Fall sagen wir, dass $f$ meromorph im unendlichen ist.
Die Fourierkoeffizienten sind durch
\begin{align*}
a_n =\int_w^{w+1} f(\tau)\mathrm{e}^{-2\pi \mathrm{i} n \tau}~\mathrm{d}\tau
\end{align*}
für beliebiges $w$ mit möglicherweise hinreichendem großem Imaginärteil (sodass $g$ im korrespondieren Ball keine Pole hat).
\end{enumerate}
\end{bem}
%TODO qs ordentlich machen

\begin{defi}
Eine holomorphe Funktion heißt \emph{holomorphe modulare Form} oder auch \emph{Modulform}, falls
\begin{enumerate}[label=\arabic*)]
\item $f(M\tau)=(c\tau +d)^k f(\tau)$ für alle $M=\begin{pmatrix}
a&b\\
c&d
\end{pmatrix} \in \Gamma$.
\item $f$ holomorph im unendlichen ist, das heißt, $f$ hat eine Fourierentwicklung der Form
\begin{align*}
f(\tau)=\sum_{n=0}^\infty a_n q^n
\end{align*}
für $q=\mathrm{e}^{2\pi \mathrm{i} \tau}$.
\end{enumerate}
Falls zusätzlich $a_0=0$ gilt, dann heißt $f$ \emph{Spitzenform}.
\end{defi}

Der Raum aller modularer Formen von Gewicht $k$ wird mit $M_k$
bezeichnet und der Raum der Spitzenform mit $S_k$.
\begin{thm}[Hecke]
Sei $f(\tau)=\sum_{n=1}^\infty a_n q^n$ einen Spitzenform von
Gewicht $k>0$.
Dann gilt
\begin{align*}
\abs{a_n} \leq c n^{k/2}.
\end{align*}
\end{thm}
Diese Schranke heißt \emph{triviale Schranke}. Man kann diese Schranke noch verbessern, dafür gab es die Fields-Medaille und den Abel-Preis.
%TODO Name raussuchen.
\begin{proof}
Mit
\begin{align*}
h(\tau)=\abs{f(\tau)} \Imt(\tau)^{k/2}
\end{align*}
gilt
\begin{align*}
h(M\tau)=h(\tau)
\end{align*}
für alle $M \in \Gamma$.
%TODO wieso ist das so?
Außerdem gilt
\begin{align*}
f(\tau)=\sum_{n=1}^\infty a_n q^n=q\sum_{n=1}^\infty a_n q^{n-1}.
\end{align*}
Deshalb ist $\mathrm{e}^{-2\pi \mathrm{i}t} f(\tau)$ für alle $\gamma>0$ auf
dem Gebiet $\{\gamma \in H \mid \Imt(\tau)\geq \gamma\}$ beschränkt.
%TODO wieso?
Also ist
\begin{align*}
\abs{f(\tau) \mathrm{e}^{-2\pi \mathrm{i}t}} \leq \abs{f(\tau)} \mathrm{e}^{2\pi y}
\end{align*}
%TODO y einfach aufgrund der schreibweise z=x+iy
und
\begin{align*}
h(T))\abs{f(\tau)}y^{k/2}
%TODO \tau?
\end{align*}
auf $\overline{\mathbb{D}}$ beschränkt.
Da $h$ invariant unter $\Gamma$ ist, gilt
\begin{align*}
0 \leq h(\tau)\leq c'
\end{align*}
für ein $c'>0$ und alle $\tau \in H$.
Nun schätzen wir die Fourierkoeffizienten ab
\begin{align*}
\abs{a_n}&=\abs{\int_0^1 f(x+iy) \ex^{-2\pi \mathrm{i} n(x+\mathrm{i}}} ~\mathrm{d}x\\
&\leq \ex^{2\pi ny} \int_0^1 \abs{f(x+\mathrm{i}y)}~\mathrm{d}x\\
&\leq \ex^{2\pi ny} y^{-k/2} \int_0^1 h(x+\mathrm{i}y)~\mathrm{d}x\\
&\leq c' y^{-k/2} \mathrm{e}^{2\pi ny}.
\end{align*}
%TODO ah das ist der parametrisierte Weg schon eingesetzt, das möchte man tendenziell hinschreiben, 1/n sollte groß genug sein, da wir holomorph sind.
Wenn wir $y=\frac{1}{n}$ wählen, erhalten wir die Behauptung.
\end{proof}

\begin{bem}
Eine Konsequenz des Beweises von Deligne\footnote{\url{http://www.numdam.org/article/SB_1968-1969__11__139_0.pdf}} der Ramanujan-Petersson Vermutung \footnote{\url{https://en.wikipedia.org/wiki/
Ramanujan-Petersson_conjecture}}
ist
\begin{align*}
\abs{a_n} \leq c(\varepsilon) n^{\frac{k-1}{2}+\varepsilon}
\end{align*}
für alle $\varepsilon>0$.
\end{bem}

\begin{defi}
Die einfachsten Beispiele für Modulformen sind die \emph{Eisensteinreihen}\footnote{\url{https://de.wikipedia.org/wiki/Gotthold_Eisenstein}}
\begin{align*}
G_k(\tau)\coloneqq \sum_{\substack{(m,n)\in \Z^2\\ (m,n)\not = (0,0)}} \frac{1}{(m\tau +n )^k}.
\end{align*}
\end{defi}

Zum Beweis der Wohldefiniertheit der Eisensteinreihen benötigen wir
das folgende Lemma.

\begin{lem}
Sei $K\subseteq H$ kompakt.
Dann gibt es Konstanten $\gamma,\delta>0$, sodass
\begin{align*}
\gamma \abs{m\mathrm{i}+n} \leq \abs{m\tau +n} \leq \delta \abs{m\mathrm{i}+n}
\end{align*}
für alle $m,n \in \R$ und $\tau \in K$ gilt.
\end{lem}
\begin{proof}
Dies verbleibt als Übung.
Sollte sofort aus Stetigkeit und Positivität der mittleren Funktion folgen.
\end{proof}

\begin{thm}
Sei $k\geq 3$. Dann konvergiert die Eisensteinreihe $G_k$
absolut und lokal gleichmäßig.
Insbesondere ist $G_k$ holomorph auf $H$.
\end{thm}
\begin{proof}
Aufgrund des vorherigen Lemmas reicht es zu zeigen, dass
\begin{align*}
\sum_{\substack{(mn,n)\in \Z^2\\ (m,n)\not =(0,0)}} \frac{1}{(m^2+n^2)^\alpha}
\end{align*}
für $\alpha>1$ konvergiert. An dieser Stelle ist $k\geq 3$ wichtig,
da man sonst $\alpha\leq 1$ erhalten würde.
Im weiteren betrachten wir die Reihe für eine endliche Teilmenge $E \subseteq \Z^2$ und schätzen diese unabhängig von der Menge ab.
\end{proof}
%TODO im beweis kommt die 4 daher, dass wir nur auf einen quadranten gehen

\begin{bem}
Da die Reihen absolut konvergieren, können wir die Summationsreihenfolge der $G_k$ vertauschen,
ohne dass sich der Wert ändert.
\end{bem}

Als nächstes zeigen wir, dass die $G_k$ wirklich von Gewicht $k$ sind.

\begin{thm}
Für $k\geq 3$ gilt
\begin{align*}
G_k(M\tau)=(c\tau +d)^k G_k(\tau).
\end{align*}
\end{thm}
\begin{proof}
rechnen, wird nachgereicht.
\end{proof}

\begin{prop}
Für $\tau \in H$ und $k\geq 2$ gilt
\begin{align*}
\sum_{n \in \Z} \frac{1}{(\tau+n)^k}=\frac{(-2\pi \mathrm{i})^k}{(k-1)!}  \sum_{n=1}^\infty n^{k-1} q^n
\end{align*}
mit $q=\mathrm{e}^{2\pi \mathrm{i} \tau}$.
\end{prop}

Nun zeigen wir die Fourierreihendarstellung, was wir benötigen, um zu zeigen, dass Eisensteinreihen Modulformen definieren.
\begin{thm}
Für alle $\tau \in H$ und $k \in 2\Z, k\geq 4$ gilt
\begin{align*}
G_k(\tau)=2 \zeta(k)+2\frac{(2\pi \mathrm{i})^k}{(k-1)!} \sum_{n=1}^\infty \sigma_{k-1}(n)q^n
\end{align*}
mit $\sigma_{k-1}(n)=\sum_{d \mid n} k^{d-1}$.
Insbesondere gilt $G_k \in M_k$.
\end{thm}

\begin{defi}
Die \emph{normalisierte Eisensteinreihe} wird als
\begin{align*}
E_k\coloneqq  \frac{1}{2\zeta(k)} G_k
\end{align*}
definiert.
Für $k \in 2\Z, k\geq 4$ gilt
\begin{align*}
2 \zeta(k)=-\frac{(2\pi \mathrm{i})^k}{k!}B_k
\end{align*}
wobei $B_k$ Bernoullizahlen\footnote{\url{https://de.wikipedia.org/wiki/Bernoulli-Zahl}} sind.
Also gilt
\begin{align*}
E_k(\tau)=1-\frac{2k}{B_k} \sum_{n=1}^\infty \sigma_{k-1}(n)q^n.
\end{align*}
Man kann außerdem zeigen, dass
\begin{align*}
E_k(\tau)=\frac{1}{2} \sum_{(m,n)=1} \frac{1}{(m\tau +n)^k}
\end{align*}
gilt.
\end{defi}

\begin{defi}
Sei $k\in \Z$.
Für eine meromorphe Funktion $f \colon H \to \C \cup \{\infty\}$
definieren wir eine Operation von $M=\begin{pmatrix}
a&b\\
c&d
\end{pmatrix} \in \GL_2(\R^+)$
durch
\begin{align*}
(f\mid_{k,M})(\tau)=\det(M)^{k/2}(c\tau +d)^{-k} f(M\tau).
\end{align*}
Dann gilt für $M,N \in \GL_2(\R)$
%TODO muss da noch ein plus hin?!
\begin{align*}
(f\mid_{k,M})\mid_{k,N} =f \mid_{k,MN}.
\end{align*}
Der Index wird oft weggelassen, wenn er aus dem Kontext klar ist.
\end{defi}

\begin{defi}
Sei $f$ eine meromorphe Funktion auf $H$.
Die Ordnung von $f$ in $\omega \in H$ ist definiert als die Zahl $n$, sodass
\begin{align*}
\frac{f(\tau)}{(\tau-\omega)^n}
\end{align*}
holomorph und ungleich $0$ in $\omega$ ist.
Wir schreiben $n=\nu_\omega(f)$.
Wenn $f$ eine meromorphe Modulform ist, dann gilt
\begin{align*}
\nu_\omega(f)=\nu_{M\omega}(f)
\end{align*}
für alle $M \in \Gamma$.
Wenn $f(\tau)=\sum_{n\geq n_0} a_nq^n$ mit $a_{n_0}\not =0$ ist, dann gilt $\nu_\infty(f)=n_0$.
\end{defi}

\begin{thm}[$\frac{k}{12}$-Formel, Gewichtsformel]
Sei $f\not =0$ eine meromorphe Modulform von Gewicht $k$.
Dann hat $f$ nur endlich viele Polstellen und Nullstellen modulo $\Gamma$ und es gilt
\begin{align*}
\nu_\infty(f)+\sum_{\tau \in \Gamma \ H} \frac{1}{\frac{1}{2}\abs{G_t}} \nu_t(f)=\frac{k}{12}.
\end{align*}
%TODO was heißt das?, \Gamma modulo H 
\end{thm}
\begin{proof}
rechnen und cauchyintegral formel + Residuensatz.
\end{proof}

\begin{bem}
Die Transformationsformel impliziert, dass $M_k=\{0\}$, wenn $k$ ungerade ist.
Sei $0 \not =f \in M_k$
\end{bem}

\begin{prop}
\begin{enumerate}[label=(\roman*)]
\item $M_k=\{0\}$ für $k<0$ oder $k=2$.
\item $M_k=\C$ für $k=0$.
\item $M_k=\C E_k$ für $k=4,6,8,10,14$.
\end{enumerate}
\end{prop}

\begin{defi}
Wir definieren
\begin{align*}
\bigtriangleup(\tau)=\frac{1}{1728} (E_4(\tau)^3-E_6(\tau)^2)\not =0.
\end{align*}
Dann ist $\bigtriangleup\in S_{12}$, $\nu_\infty(\bigtriangleup)=1$ und $\bigtriangleup(\tau)\not=0$ für alle $\tau \in H$.
\end{defi}

Die \glqq zufällig\grqq\ erscheinende Wahl der Potenzen erklärt sich
durch Theorem 13.22.
Die Wahl des Faktors dient der Normalisierung.

\begin{prop}
\begin{enumerate}[label=(\roman*)]
\item $S_k=\{0\}$ für $k<12$ oder $k=14$.
\item $S_k=\bigtriangleup M_{k-12}$ für $k>14$.
\item $M_k=\C E_k\oplus S_k$ für $k>2$.
\end{enumerate}
\end{prop}

In der Zerlegung in (iii) ist der erste Summand derjenige, der sehr gut verstanden ist.
Alle Mysterien verstecken sich im zweiten Summanden.

\begin{prop}
Sei $0\leq k\in 2\Z$. Dann gilt
\begin{align*}
\dim M_k=\begin{cases}
\left[\frac{k}{12}\right],~~&\text{wenn } k \equiv 2 \mod 12\\
\left[\frac{k}{12}\right]+1,&\text{ wenn } k \not \equiv 2 \mod 12.
\end{cases}
\end{align*}
\end{prop}

Die folgende Aussage ist sehr wichtig, dadurch sieht man dass $M$
eine sehr einfache Struktur hat.

\begin{thm}
Die Abbildung $\C[X,Y]\to M$ definiert durch
\begin{align*}
X &\mapsto E_4\\
Y &\mapsto E_6
\end{align*}
ist ein Isomorphismus von Ringen mit $M =\bigoplus_{k=0}^\infty M_k$.
Die \emph{Hilbertfunktion} ist durch
\begin{align*}
\sum_{k=0}^\infty \dim(M_k)t^k&=\frac{1}{(1-t^4)(1-t^6}\\
&=\left(\sum_{n=0}^\infty t^{4n}\right) \left(\sum_{m=0}^\infty t^{6m}\right)
\end{align*}
definiert.
\end{thm}
\begin{proof}
Der Beweis verbleibt als Übung.
\end{proof}

\begin{defi}
Wir definieren $j=\frac{E_4^3}{\bigtriangleup}$.
\end{defi}
Wir teilen nicht durch $0$, da $\bigtriangleup\not = 0$.

\begin{bem}
Dann gilt $j(\tau)=q^{-1} +744 + 196884q+ \dots$.
Es gilt $196884=196883+1$ wobei der erste Summand mit der Monstergruppe zu tun hat.
Die $j$-Funktion liefert einen Zusammenhang zwischen Gruppentheorie und Modulformen.
Für weitere Details verweisen wir auf den Wikipediaartikel und die dort angegebenen Quellen\footnote{\url{https://de.wikipedia.org/wiki/J-Funktion}}.
\end{bem}

\begin{prop}
$j$ ist eine meromorphe modulare Form von Gewicht $0$ und hat einen
einfachen Pol in $\infty$.
Außerdem definiert $j$ eine Bijektion zwischen $\Gamma\backslash H$ und $\C$.
\end{prop}
\begin{proof}
Die erste Aussage ist klar.
Für die zweite müssen wir zeigen, dass für $\lambda \in \C$ die Funktion
\begin{align*}
f_\lambda\coloneqq E_4^3-\lambda \bigtriangleup
\end{align*}
genau eine Nullstelle in $H$ hat.
Aus der Gewichtsformel folgt
\begin{align*}
\nu_\infty(f_\lambda)+\frac{1}{3}\nu_\rho(f_\lambda)+\frac{1}{12}\nu_{\mathrm{i}}(f_\lambda)+\sum_{\substack{\tau \in \Gamma\backslash H\\ \tau \not = \rho,\mathrm{i}}} \nu_\tau(f_\lambda)=1.
\end{align*}
Dies vereinfacht sich durch Einführen geeigneter Variablen zu
\begin{align*}
\frac{n_\rho}{3}+\frac{n_{\mathrm{i}}{2}}+n_\tau=1
\end{align*}
mit nicht negativen $n_\rho,n_\mathrm{i},n_\tau \in \Z$.
Diese Gleichung hat Lösungen $(3,0,0),(0,2,0)$ und $(0,0,1)$.
Dies zeigt die Behauptung.
\end{proof}

\begin{prop}
Die meromorphen modularen Formen von Gewicht $0$ auf $H$ sind genau
die rationalen Funktionen in $j$.
\end{prop}
\begin{proof}
Sei $f$ eine meromorphe modulare Form von Gewicht $0$.
Wir nehmen an, dass $f$ Pole der Ordnung $m_1,\dots,m_n$
in $\tau_1,\dots,\tau_n$ hat.
Dann ist
\begin{align*}
f(z)\prod_{i=1}^n (j(z)-j(\tau_i))^{m_i}
\end{align*}
holomorph in $H$.
Daher können wir ohne Einschränkung annehmen, dass $f$ holomorph in $H$ ist.
Für geeignetes $0\leq k \in \Z$ gilt $f\bigtriangleup^k \in M_{12k}$ und wir können Funktion als
\begin{align*}
f\bigtriangleup^k=\sum_{4i+6j=12k} \alpha_{ij} E_4^i E_6^j
\end{align*}
schreiben.
Damit gilt
\begin{align*}
f=\sum_{i+j=k} \alpha_{ij} \frac{E_4^{3i}}{\bigtriangleup^i} \frac{E_6^{2j}}{\bigtriangleup^j},
\end{align*}
was die Behauptung beweist, da $\frac{E_4^3}{\bigtriangleup}=j$ und
$\frac{E_6^2}{\bigtriangleup}=j+c$ für ein geeignetes $c$ gilt, da auch $E_4^3$ und $\bigtriangleup$ eine Basis von $M_{12}$ bilden.
\end{proof}

Bevor wir uns die Eisensteinreihe von Gewicht $2$ betrachten, wiederholen wir noch einmal die Definition.
Mit Lemma 13.7 gilt für alle $\tau \in K$, wobei $K$ kompakt ist, dass
\begin{align*}
\abs{G_k(\tau)}\leq \sum_{\substack{(m,n)\in \Z^2\\ (m,n)\not = (0,0)}} \frac{1}{(m\tau +n )^k}\\
&\leq\frac{1}{\gamma} \sum_{\substack{(m,n)\in \Z^2\\ (m,n)\not = (0,0)}}\frac{1}{\abs{m\mathrm{i}+n}^k} \\
&\leq \frac{1}{\gamma} \sum_{\substack{(m,n)\in \Z^2\\ (m,n)\not = (0,0)}} \frac{1}{(m^2+n^2)^{k/2}}\\
&\leq \frac{1}{\gamma} \left(4 \sum_{m\geq 1} \frac{1}{m^k}+4\sum_{m,n\geq 1} \frac{1}{(m^2+n^2)^{k/2}} \right)\\
&\leq \frac{1}{\gamma} \left( 4 \sum_{m\geq 1} \frac{1}{m^k}+4 \sum_{m\geq 1} \frac{1}{m^{k/2}} \sum_{n\geq 1} \frac{1}{n^{k/2}}\right)\\
&<\infty,
\end{align*}
falls $k\geq 3$.
Wir betrachten nun den Fall $k=2$, in dem die Reihe erstmal nicht
absolut konvergieren muss.
Daher müssen wir eine Additionsreihenfolge vorgeben.
\begin{defi}
Wir definieren
\begin{align*}
G_2(\tau)=\sum_{n \not =0} \frac{1}{n^2} +\sum_{m \not =0} \left( \sum_{n\in \Z} \frac{1}{(m\tau+n)^2}\right).
\end{align*}
\end{defi}

Das folgende Resultat beschreibt die Wohldefiniertheit und einen Zusammenhang zum Fall $k\geq 3$.
\begin{prop}
$G_2$ definiert einen holomorphe Funktion auf $H$ und es gilt
\begin{align*}
G_2(\tau)=\frac{\pi^2}{3}(1-24\sum_{n=1}^\infty \sigma_i(n)q^n)
\end{align*}
\end{prop}
\begin{proof}
Für $m\not =0$ gilt
\begin{align*}
\sum_{n \in \Z} \frac{1}{(m\tau+n)^2} =(2\pi \mathrm{i})^2 \sum_{n=1}^\infty nq^{mn},
\end{align*}
sodass
\begin{align*}
\sum_{m\not =0} \sum_{n \in \Z} \frac{1}{(m \tau+n)^2}=-8\pi^2 \sum_{m=1}^\infty \sum_{n=1}^\infty n q^{mn}
\end{align*}
gilt.
Wenn wir zeigen, dass die letzte Summe absolut konvergiert, dürfen wir umordnen.
Es gilt
\begin{align*}
\sum_{m=1}^\infty \sum_{n=1}\infty \abs{nq^{nm}} &= \sum_{n=1}^\infty n \sum_{m=1}^\infty \abs{q^n}^m\\
&=\sum_{n=1}^\infty n \frac{\abs{q^n}}{1-\abs{q^n}}\\
&=\sum_{n=1}^\infty n \frac{\abs{q}^n}{1-\abs{q}^n}\\
&\leq C+2\sum_{n=1}^\infty n \abs{q}^n
\end{align*}
wobei wir im letzten Schritt verwendet haben, dass $1-\abs{q}^n \geq \frac{1}{2}$ für fast alle $n$ gilt.
Da $n \abs{q}^n \leq \abs{q}^{n/2}$ für alle bis auf endliche viele $n$ gilt (aufgrund des exponentiellen Abfalls), dass wir den letzten Term nach oben gegen $D+2\sum_{n=1}^\infty \abs{q}^{n/2}$ abschätzen.
Dann dürfen wir umordnen und erhalten die Behauptung.

Für einen alternativen Beweis bemerken wir zunächst
\begin{align*}
n^j \leq \sigma_j(n)\leq \zeta(j)n^j.
\end{align*}
Die untere Abschätzung ist klar und die obere gilt wegen
\begin{align*}
\sigma_j(n)&=\sum_{d\mid n} d^j \\
&=n^j \sum_{d\mid n} \left(\frac{d}{n}\right)^j\\
&=n^j \sum_{d\mid n} \left(\frac{1}{n}\right)^j\\
&\leq n^j \zeta(j).
\end{align*}
Damit folgt $(\sigma_j(n))^{1/n} \to 1$ für $n \to \infty$.
\end{proof}

Es gilt
\begin{align*}
G_2(-\frac{1}{\tau})&=\sum_{n\not =0} \frac{1}{n^2} + \sum_{m \not= 0} \sum_{n\in \Z} \frac{1}{(-m\frac{1}{\tau}+n)^2}\\
&=\tau^2 \left(\frac{\pi^2}{3}+\sum_{n \in \Z}\sum_{m \not =0} \frac{1}{(m\tau+n)^2}\right).
\end{align*}
Die wichtige Beobachtung ist, dass falls $G_2$ eine modulare Form von Gewicht $2$ wäre, dann der letzte Summand wieder $\tau^2 G_2$ wäre.
Dies ist aber nicht der Fall, da die Reihen in der \glqq falschen\grqq\ Reihenfolge genommen werden.
Den Unterschied zwischen den beiden Termen beschreibt die folgende Proposition.

\begin{prop}
Es gilt $G_2(-\frac{1}{\tau})=\tau^2 G_2(\tau)-2\pi \mathrm{i}\tau$.
\end{prop}
\begin{proof}
Wir definieren
\begin{align*}
a_{m,n}(\tau)\coloneqq\frac{1}{(m\tau+n-1)(m\tau+n)}=\frac{1}{m\tau +n-1}-\frac{1}{m\tau +n}.
\end{align*}
Dann gilt
\begin{align*}
\frac{1}{(m\tau+n)^2}-a_{m,n}(\tau)=-\frac{1}{(m\tau+n)^2(m\tau+n-1)}.
\end{align*}
Es gilt
\begin{align*}
G_2^\ast(\tau)&=\frac{\pi^2}{3}+\sum_{m\not=0} \sum_{n\in \Z} \left(\frac{1}{(m\tau+n)^2}-a_{m,n}(\tau)\right)\\
&=\frac{\pi^2}{3}+\sum_{m\not=0} \sum_{n\in \Z} \frac{1}{(m\tau+n)^2}-\sum_{m\not=0}\sum_{n\in \Z} \left(\frac{1}{m\tau+n-1}-\frac{1}{m\tau +n}\right)=G_2(\tau),
\end{align*}
wobei wir verwendet haben, dass die Doppelsumme absolut konvergiert
und wir daher umordnen dürfen.
Der letzte Summand verschwindet als Teleskopsumme.
Da die Doppelsumme absolut konvergiert, erhalten wir
\begin{align*}
G_2(\tau)&= \frac{pi^2}{3}+\sum_{n\in \Z} \sum_{m\not =0} \left(\frac{1}{(m\tau +n)^2}-a_{m,n}(\tau)\right)\\
&=\tau^{-2} G_2\left(-\frac{1}{\tau}\right)- \sum_{n \in \Z} \sum_{m \not =0} a_{m,n}(\tau).
\end{align*}
Da die äußere Summe in $G_2(\tau)$ absolut konvergiert, gilt dies auch für $\sum_{n \in \Z}\left( \sum_{m\not =0} a_{m,n}(\tau)\right)$ (da die Terme \glqq nicht weit auseinander sind\grqq) und es gilt
\begin{align*}
\sum_{n\in \Z} \sum_{m \not=0} a_{m,n}(\tau)
&=\lim_{N\to \infty} \sum_{-N+1}^N \sum_{m\not =0} a_{m,n}(\tau)\\
&=\lim_{N \to \infty} \sum_{m\not =0} \sum_{-N+1}^N a_{m,n}(\tau)\\
&=\lim_{N \to \infty} \frac{2}{\tau} \sum_{n=1}^\infty \left(\frac{1}{-\frac{N}{\tau}+m}+ \frac{1}{-\frac{N}{\tau}-m}\right)\\
&=\lim_{N\to \infty} \frac{2}{\tau} \left(\pi \coth\left(-\frac{\pi N}{\tau}\right) +\frac{\tau}{N}\right)\\
&=\frac{2\pi}{\tau} \lim_{N\to \infty} \mathrm{i} \frac{\ex^{-2\pi \mathrm{i}\frac{N}{\tau}}+1}{\ex^{-2\pi \mathrm{i}\frac{N}{\tau}}-1}\\
&=-\frac{2\pi \mathrm{i}}{\tau}.
\end{align*}
\end{proof}

\begin{defi}
Wir definieren nun die Dedekindsche $\eta$-Funktion.
\begin{align*}
\eta(\tau)=q^{\frac{1}{24}} \prod_{n=1}^\infty(1-q^n)
\end{align*}
mit $q=\ex^{2\pi \mathrm{i}\tau}$.
\end{defi}
Da die Reihe $\sum_{n=1}^\infty q^n$ absolut und kompakt gleichmäßig für $\abs{q}<1-\varepsilon$ konvergiert, folgt,
dass das Produkt konvergiert und holomorph ist (für weitere Details siehe Freitag Funktionentheorie)
%TODO Quelle gut machen mit Verweis ins literaturverzeichnis

\begin{prop}
Es gilt
\begin{align*}
\eta(-\frac{1}{\tau})=\sqrt{\frac{\tau}{\mathrm{i}}} \eta(\tau)
\end{align*}
\end{prop}
\begin{proof}
Wir berechnen nun die logarithmische Ableitung
\begin{align*}
\frac{\eta'(\tau)}{\eta(\tau)}&=\left(\log(\eta(\tau))\right)'\\
&=\frac{2\pi \mathrm{i}}{24}\left(1-24 \sum_{n=1}^\infty \frac{nq^n}{1-q^n}\right)\\
&=\frac{2\pi\mathrm{i}}{24} \left(1-24 \sum_{n=1}^\infty nq^n \sum_{k=0}^\infty q^{kn} \right)\\
&=\frac{2\pi \mathrm{i}}{24}\left(1-24 \sum_{n=1}^\infty \sum_{k=0}^\infty nq^{n(k+1} \right)\\
&=\frac{2\pi \mathrm{i}}{24}\left(1-24 \sum_{n=1}^\infty \sigma_i(n)q^n\right)\\
&=\frac{2\pi \mathrm{i}}{24} \frac{3}{\pi^2}G_2(\tau)\\
&=\frac{1}{4 \pi} G_2(\tau).
\end{align*}
Als nächstes berechnen wir die logarithmische Ableitung der rechten Seite.
\begin{align*}
\frac{\left(\sqrt{\frac{\tau}{\mathrm{i}}}\eta(\tau)\right)'}{\left(\sqrt{\frac{\tau}{\mathrm{i}}}\eta(\tau)\right)}&= \frac{\frac{1}{2 \sqrt{\frac{\tau}{\mathrm{i}}}}\frac{1}{\mathrm{i}}\eta(\tau)+\sqrt{\frac{\tau}{\mathrm{i}}}\eta'(\tau)}{\sqrt{\frac{\tau}{\mathrm{i}}}\eta(\tau)}\\
&=\frac{1}{2 \tau} +\frac{\eta'(\tau)}{\eta(\tau)}\\
&=\frac{\mathrm{i}}{4\pi \tau^2}\left(\tau^2 G_2(\tau)-2\pi \mathrm{i}\tau\right)\\
&=\frac{1}{\tau^2}\frac{\mathrm{i}}{4\pi} G_2\left(-\frac{1}{\tau}\right)\\
&=\frac{1}{\tau^2} \frac{\eta'\left(-\frac{1}{\tau}\right)}{\eta\left(-\frac{1}{\tau}\right)}\\
&=\frac{\left(\eta\left(-\frac{1}{\tau}\right)\right)'}{\eta\left(-\frac{1}{\tau}\right)}
\end{align*}
Dies impliziert, dass $\eta\left(-\frac{1}{\tau}\right)=c\sqrt{\frac{\tau}{\mathrm{i}}} \eta(\tau)$ für eine geeignete Konstante.
Indem man $\tau=1$ setzt, erhält man $c=1$.
\end{proof}
%24 ist eine sehr merkwürdige Zahl, sie taucht an mehreren Stellen auf, kritische Dimension Boson-Higgs

\begin{cor}
Es gilt
\begin{align*}
\bigtriangleup{\tau}&=\frac{1}{1728}=(E_4^3(\tau)\\
&=q\prod_{n=1}^\infty (1-q^n)^{24}\\
&=\eta^{24}(\tau).
\end{align*}
\end{cor}
%Dies sieht man, da beides modulformen von gewicht 12 sind.

Man kann Modulformen über verschiedene Ansätze erhalten.
Wir haben bereits gesehen, dass dies über Eisensteinreihen und Produktansätze geht.
Als nächstes betrachten wir $\theta$-Funktionen.

Sei $V$ ein euklidischer Vektorraum von Dimension $n$.
Ein \emph{Gitter} ist eine Teilmenge $L\subseteq V$, falls diese von der Form
\begin{align*}
L=\Z v_1+\dots + \Z v_n
\end{align*}
für eine Basis $(v_1,\dots,v_n)$ ist.
Die Zahl $\det((v_i,v_j))$ ist unabhängig von der gewählten Basis
und heißt \emph{Determinante von $L$}, das heißt $\det(L)=\det((v_i,v_j))$.
Es gilt außerdem, dass das Volumen des Fundamentalbereichs durch die Wurzel der Determinante von $L$ gegeben ist.
%TODO nachschauen wie sehr das stimmt.
Das \emph{duale Gitter} zu $L$ ist durch
\begin{align*}
L'\coloneqq \{ x\in V \mid (x,\alpha) \in \Z \forall \alpha \in L\}
\end{align*}
definiert und ist wiederum ein Gitter.
Die \emph{theta-Reihe} von $L$ ist definiert als
\begin{align*}
\theta_L(\tau)\coloneqq \sum_{\alpha \in L} q^{\frac{\alpha^2}{2}}
\end{align*}
definiert eine holomorphe Funktion auf $H$, da das Skalarprodukt
positiv definit ist.
Die Poissonsche Summationsformel impliziert, dass
\begin{align*}
\theta_L\left(-\frac{1}{\tau}\right) = \sqrt{\frac{\tau}{\mathrm{i}}}^n \frac{1}{\sqrt{\det(L)}} \theta_{L'}(\theta).
\end{align*}
Ein Gitter $L$ heißt \emph{gerade}, falls $\alpha^2=(\alpha,\alpha)\in 2\Z$ für alle $\alpha \in L$ gerade ist.
In diesem Fall folgt $(\alpha,\beta)\in \Z$ für alle $\alpha,\beta \in L$ (Polarisationsformel) und somit $L\subseteq L'$.
Außerdem gilt $\abs{L'/L}=\det(L)$.
Falls $L=L'$ gilt, heißt $L$ unimodular.

Uns interessiert die Frage, wann die theta-Funktion eines Gitters eine Modulform ist.
Dies kann höchstens dann passieren wenn es unimodular ist.
\begin{thm}
Sei $L\subseteq V$ ein gerades unimodulares Gitter. Dann gilt
\begin{enumerate}[label=\roman*)]
\item $\dim(V)=0\mod 8$,
\item $\theta_L$ ist eine Modulform von Gewicht $\frac{n}{2}$.
\end{enumerate}
\end{thm}
\begin{proof}
Beide Aussagen folgen aus der Transformationsformel von $\theta_L$.
\end{proof}

\begin{bem}
Angenommen wir haben ein gerades Gitter $L$. Dann gilt
\begin{align*}
\theta_L(\tau)&=\sum_{a\in L} q^{\frac{\alpha^2}{2}}\\
&=\sum_{n=0}^\infty a_n q^n,
\end{align*}
wobei $a_n=\abs{\{\alpha \in L \mid \alpha^2=2n\}}$.
\end{bem}

\begin{bsp}
Betrachte $D_n=\{(x_1,\dots,x_n)\in \Z^n\mid \sum_{i=1}^nx_i=0\mod 2\}\subseteq \Rn$.
Dies ist gerades Gitter mit Determinante $4$.
Falls $n=0 \mod 8$ ist, definieren wir
\begin{align*}
D_n^+\coloneqq D_n \cup (s+D_n)
\end{align*}
mit $s=(\frac{1}{2},\dots,\frac{1}{2})$.
Dies ist ein gerades Gitter, da $(s,s)=\frac{n}{4} \in 2\Z$ gilt.
$D_8^+$ heißt auch $E_8$. Es gilt
\begin{align*}
\theta_{E_8}(\tau)=E_4(\tau)=1+240q+2160q^2+\dots.
\end{align*}
Man kann sich überzeugen, dass es genau $240$ Vektoren der Länge $1$ gibt.
\end{bsp}