\chapter{Zerlegungen von $\GL_2(\R)$}
Wir erinnern uns, dass $G=\SL_2(\R)$ auf der oberen Halbebene durch
\begin{align*}
\begin{pmatrix}
a&b\\
c&d
\end{pmatrix}\tau=\frac{a\tau+b}{c\tau+d}
\end{align*}
operiert.
Der Stabilisator von $\mathrm{i}$ in $G$ ist durch
\begin{align*}
K&=\left\lbrace \begin{pmatrix}
a&b\\
-b&a
\end{pmatrix}\mid a,b\in \R, a^2+b^2=1\right\rbrace\\
&=\left\lbrace \begin{pmatrix}
\cos \alpha &-\sin \alpha \\
\sin \alpha & \cos \alpha
\end{pmatrix} \mid 0\leq \alpha <2\pi\right\rbrace\\
&=\SO_2(\R)
\end{align*}
gegeben.
Weiterhin definieren wir
\begin{align*}
A=\left\lbrace \begin{pmatrix}
y&0\\
0&\frac{1}{y} 
\end{pmatrix} \mid y>0 \right\rbrace
\end{align*}
sowie
\begin{align*}
N=\left\lbrace 
\begin{pmatrix}
1&x\\
0&1
\end{pmatrix} \mid x\in \R\right\rbrace.
\end{align*}
Diese Gruppen bilden als Produkt grade $G$.
\begin{prop}[Iwasawa Zerlegung]
Sei $g\in G$. Dann kann $g$ eindeutig als Produkt
\begin{align*}
g=ank
\end{align*}
mit $a \in A$, $n\in N$ und $k \in K$ geschrieben werden.
\end{prop}
\begin{proof}
Sei $g(\mathrm{i})=x+\mathrm{i}y\in H$.
Definiere
\begin{align*}
a&=\begin{pmatrix}
\sqrt{y}&0\\
0&\frac{1}{\sqrt{y}}
\end{pmatrix}\\
n&=\begin{pmatrix}
1&\frac{x}{y}\\
0&1
\end{pmatrix}.
\end{align*}
Dann gilt
\begin{align*}
an=\frac{1}{\sqrt{y}} \begin{pmatrix}
y&x\\
0&1
\end{pmatrix},
\end{align*}
sodass $an(\mathrm{i})=g(\mathrm{i})$ gilt.
Damit folgt $(an)^{-1} g \in K$.
Dies beweist die Existenz der Zerlegung.
Die Eindeutigkeit verbleibt als Übung.
\end{proof}
Man kann auf ähnliche Art und Weise zeigen, dass auch eine Zerlegung der Form $g=nak$ existiert.


\begin{prop}[Cartan Zerlegung]
Sei $g \in G$. Dann ist $g$ von der Form
\begin{align*}
g=\begin{pmatrix}
\cos \alpha&-\sin\alpha\\
\sin  \alpha&\cos \alpha
\end{pmatrix} \begin{pmatrix}
y&0\\
0&\frac{1}{y}
\end{pmatrix}
\begin{pmatrix}
\cos\beta&-\sin\beta\\
\sin\beta&\cos\beta
\end{pmatrix}
\end{align*}
mit $y>0$, $0\leq \alpha<\pi$ sowie $0\leq \beta <2\pi$.
\end{prop}
\begin{proof}
Die Matrix $s=g^Tg$ ist symmetrisch und positiv definit.
Also existiert $k \in \Oo_2(\R)$ mit
\begin{align*}
k^Tsk=\begin{pmatrix}
\lambda_1&0\\
0&\lambda_2
\end{pmatrix}
\end{align*}
wobei $\lambda_1,\lambda_2$ positiv sind.
Mit
\begin{align*}
d=\begin{pmatrix}
\sqrt{\lambda_1}&0\\
0&\sqrt{\lambda_2}
\end{pmatrix}
\end{align*}
gilt
\begin{align*}
k^Tg^Tgk=d^Td
\end{align*}
und weiter
\begin{align*}
gk&=(k^Tg^T)^{-1} d^Td\\
&=\underbrace{(g^T)^{-1}kd^T}_{h} d.
\end{align*}
Wir behaupten, dass $h$ orthogonal ist. Es gilt
\begin{align*}
h^T h=dk^T g^{-1} (g^{T})^{-1} kd^T=1
\end{align*}
also $h \in \Oo_2(\R)$ und $g=k^{-1}hd$
mit $h,k\in \Oo_2(\R)$.
Für die Determinante gilt
\begin{align*}
1=\det(h)\det(d)\det(k).
\end{align*}
Wegen
\begin{align*}
\begin{pmatrix}
-1&0\\
0&1
\end{pmatrix} d
\begin{pmatrix}1
-1&0\\
0&1
\end{pmatrix}=d
\end{align*}
können wir annehmen, dass $h,k \in \SO_2(\R)$.
Außerdem gilt
\begin{align*}
\det(d)=\sqrt{\lambda_1}\sqrt{\lambda_2}=1
\end{align*}
sodass wir $d$ geeignet wählen können.
\end{proof}

Die obigen Zerlegungen existieren für beliebige reduktive Lie-Gruppen. Für weitere infos siehe [Knapp Lie groups beyond an introduction].

$G \overset{\simeq}\to K \times A \times N$ wobei $K$ maximal kompakt, $A$ abelsch  und $N$ eine nilpotent Untergruppe von $G$ ist.[Knapp, p.455]
Dies ist allgemeine Iwasawa Zerlegung.
%TODO diesen Teil ausformulieren

Cartan Zerlegung $G=KAK$ wobei $K$ maximal kompakt und $A$ abelsch ist.
Als nächstes betrachten wir $\GL_2(\Q_\nu)$ mit $\GL_2(\A)$ und deren Zerlegungen.

\begin{prop}[Iwasawa decomposition]
Sei $g\in \GL_2(\R)$.
Dann kann $g$ eindeutig geschrieben werden als
\begin{align*}
g=\begin{pmatrix}
1&x\\
0&1
\end{pmatrix}
\begin{pmatrix}
y&0\\
0&1
\end{pmatrix}
\begin{pmatrix}
\cos \alpha &-\sin \alpha\\
\sin \alpha &\cos \alpha
\end{pmatrix}
\begin{pmatrix}
\pm 1&0\\
0&1
\end{pmatrix}
\begin{pmatrix}
r&0\\
0&r
\end{pmatrix}
\end{align*}
mit $x \in \R$, $y>0$, $0\leq \alpha \leq 2\pi$ sowie $r>0$.
\end{prop}

Analog betrachten wir die Cartan Zerlegung für $\GL_2(\R)$.
\begin{prop}
Sei $g \in \GL_2(\R)$. Dann kann $g$ geschrieben werden als
\begin{align*}
g=\begin{pmatrix}
\cos \alpha &-\sin \alpha\\
\sin \alpha &\cos \alpha
\end{pmatrix}
\begin{pmatrix}
y&0\\
0&\frac{1}{y}
\end{pmatrix}
\begin{pmatrix}
\cos \beta &-\sin \beta\\
\sin \beta &\cos \beta
\end{pmatrix}
\begin{pmatrix}
\pm 1&0\\
0&1
\end{pmatrix}
\begin{pmatrix}
r&0\\
0&r
\end{pmatrix}
\end{align*}
mit $y>0$, $0\leq \alpha <\pi$ und $0\leq \beta <2\pi$.
\end{prop}