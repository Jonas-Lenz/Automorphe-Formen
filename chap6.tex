\chapter{Charaktere}
An dieser Stelle vergessen wir die Notation $\hat{}$ aus Kapitel 4.
\begin{defi}
Sei $G$ eine topologische Gruppe.
Ein \emph{Quasicharakter} von $G$ ist ein stetiger Gruppenhomomorphismus $G \to \C^\ast$.
Ein \emph{Charakter} von $G$ ist ein stetiger Gruppenhomomorphismus $G=T\coloneqq \{z \in \C \mid \abs{z}=1\}$.
\end{defi}

\begin{prop}
Sei $G$ eine kompakte Gruppe und $f \colon G \to \C^\ast$ ein Quasicharakter.
Dann ist $f$ ein Charakter.
\end{prop}
\begin{proof}
Die Abbildung
\begin{align*}
G &\to \R_{>0}^\ast\\
x&\mapsto \abs{f(x)}
\end{align*}
ist ein stetiger Gruppenhomomorphismus. Daher ist das Bild eine kompakte Untergruppe von $\R_{>0}^\ast$, nämlich $\{1\}$.
\end{proof}

\begin{defi}
Eine lokal kompakte Hausdorffgruppe heißt \emph{total umzusammenhängend}, falls jede offene Umgebung der $1$ eine offene Untergruppe enthält.
\end{defi}

\begin{bsp}
Zum Beispiel sind
\begin{enumerate}[label=\roman*)]
\item $(\Q_p,+)$, $(\Q_p^\ast,\cdot)$
\item $(\Z_p,+)$, $(\Z_p^\ast,\cdot)$
\item $(\A_f,+)$, $(\A_f^\ast,\cdot)$
\item $(\hat{Z},+)$, $(\hat{Z}^\ast,\cdot)$
\end{enumerate}
solche Gruppen.
\end{bsp}

\begin{prop}
Sei $G$ eine total unzusammenhängende lokal kompakte Hausdorffgruppe und $\chi\colon G \to \C^\ast$ ein Quasicharakter.
Dann ist $\ker(\chi)$ offen.
\end{prop}
\begin{proof}
Sei $0<\varepsilon<1$. Dann ist $V_\varepsilon\coloneqq \{z\in \C \mid \Ret(z)>\varepsilon\}$ eine offene Umgebung von $1\in \C^\ast$.
Also ist $\chi^{-1}(V_\varepsilon)$ offen in $G$ und enthält somit eine offene Untergruppe $H \subseteq G$.
Das Bild $\chi(H)$ ist eine Untergruppe von $\C^\ast$ in $V_\varepsilon$.
Für $z\in \C$ ungleich $1$ existiert ein $n\in \N$ sodass die Potenz $z^n$ außerhalb von $V_\varepsilon$ liegt.
Also folgt $\chi(H)=1$.
Sei nun $x\in \ker(\chi)$. Dann ist $xH$ eine offene Umgebung von $x$ in $\ker(\chi)$.
\end{proof}

\begin{defi}
Sei $G$ eine topologische Gruppe. Dann ist die Menge der Charaktere auf $G$ eine Gruppe bezüglich punktweiser Multiplikation.
Dieser Gruppe heißt die \emph{duale Gruppe von $G$}, wir bezeichnen sie mit $\hat{G}$.
Wir statten $\hat{G}$ mit kompakt-offenen Topologie aus.
Eine Subbasis dieser Topologie bilden die Mengen der Form
\begin{align*}
S(K,U)\coloneqq \{\chi \in C(G,T)\mid \chi(K)\subseteq U\}
\end{align*}
wobei $K\subseteq G$ kompakt und $U\subseteq T$ offen ist.
Endliche Schnitte dieser Mengen bilden eine Basis der Topologie von $\hat{G}$.
\end{defi}

\begin{prop}
Die duale Gruppe $\hat{G}$ ist eine abelsche Hausdorffgruppe.
\end{prop}
\begin{proof}
Sei $f \in \hat{G}, \varepsilon\not =f$.
Dann existiert ein $x\in G$ mit
\begin{align*}
f(x)\not =1 \varepsilon(x).
\end{align*}
Da $T$ Hausdorff ist, können wir offene Mengen $U,W \subseteq T$ mit $f(x)\in U, 1\in W$ sowie $U\cap W=\emptyset$ wählen.
Definiere $K=\{x\}$.
Dann gilt $f\in S(K,U)$, $\varepsilon\in S(K,W)$ sowie $S(K,U)\cap S(K,W)=\emptyset$.
\end{proof}

\begin{prop}
Sei $G$ eine topologische Gruppe.
\begin{enumerate}[label=\roman*)]
\item Falls $G$ endlich ist, so ist $\hat{G}$ ebenfalls endlich,
\item Falls $G$ diskret ist, so ist $\hat{G}$ kompakt.
\end{enumerate}
\end{prop}

\begin{prop}
Sei $G$ eine Hausdorffgruppe.
\begin{enumerate}[label=\roman*)]
\item Falls $G$ lokal kompakt ist, so ist $\hat{G}$ ebenfalls lokal kompakt.
\item Falls $G$ kompakt ist, so ist $\hat{G}$ diskret.
\end{enumerate}
\end{prop}

\begin{bsp}
Wir betrachten einige Beispiele von Charakteren.
\begin{enumerate}[label=\roman*)]
\item Die Abbildung
\begin{align*}
\mathrm{e}_\infty \colon \R &\to T\\
x &\mapsto \mathrm{e}^{2\pi \mathrm{i}x}\eqqcolon \mathrm{e}(x)
\end{align*}
ist ein Charakter von $\R$.
\end{enumerate}
\end{bsp}

Nun werden wir die Charaktere einiger Gruppen bestimmen, die in der Vorlesung eine Rolle spielen, z.B. $\R$, $\Q$, $\A$.

\begin{prop}
Es gilt $\chi(x)=\mathrm{e}^{2\pi \mathrm{i}a x}$.
Die Abbildung
\begin{align*}
\R &\to \hat{\R}\\
a&\mapsto (x\mapsto \mathrm{e}_\infty(ax))
\end{align*}
ist ein Isomorphismus topologischer Gruppen.
\end{prop}
\begin{proof}
Wir zeigen, dass jeder Charakter von $\R$ von der angegebenen Form ist.
Sei $\gamma \colon \R \to T$ ein Charakter. Dann gilt $\gamma(0)=1$ und wegen der Stetigkeit von $\gamma$ existiert $a>0$ mit
\begin{align*}
\int_0^a \gamma(t)~\mathrm{d}t=c\not =0.
\end{align*}
Für $x\in \R$ gilt
\begin{align*}
\int_x^{x+a}\gamma(t)~\mathrm{d}t&= \int_0^a\gamma(x+t)~\mathrm{d}t\\
&=\gamma(x)\int_0^a \gamma(t)~\mathrm{d}t=\gamma(x)c
\end{align*}
und somit folgt $\gamma(x)=\frac{1}{c}\int_x^{x+a} \gamma(t)~\mathrm{d}t$.
Dies impliziert aber, dass $\gamma$ differenzierbar ist mit
\begin{align*}
\gamma'(t)&=\frac{1}{c}\left(\gamma(x+a)-\gamma(x)\right)\\
&= \frac{1}{c}(\gamma(a)-1)\gamma(x).
\end{align*}
Aus der Differentialgleichung folgt, dass $\gamma$ von der gewünschten Gestalt ist.
\end{proof}

\begin{prop}
Die Abbildung
\begin{align*}
\Z &\to \hat{\R/\Z}\\
n &\mapsto (x \to \mathrm{e} (nx))
\end{align*}
ist ein Isomorphismus.
\end{prop}

\begin{prop}
Die dualen Gruppen der zyklischen Gruppen sind
\begin{align*}
\hat{\Z/n\Z} &\simeq \Z/n\Z\\
\hat{\Z} &\simeq \R/\Z \simeq T.
\end{align*}
\end{prop}

Sei nun $p<\infty$. Wie wir wissen, ist ein Element $a\in \Q_p$ von der Form
\begin{align*}
a=\sum_{n=n_p}^\infty a_p(n)p^n
\end{align*}
mit $a_p(n)\in \Z$, $0\leq a_p(n)<p$.
Der  \emph{Hauptteil} von $a$ ist definiert als
\begin{align*}
a^-=\sum_{n=n_p}^{-1} a_p(n)p^n.
\end{align*}
In der folgenden Definition ist eine gute Konvention $-a^-$ zu wählen, dies wird später klar werden.
Wir definieren also
\begin{align*}
e_p(a)\coloneqq  \mathrm{e}(-a^-)=\mathrm{e}^{2\pi \mathrm{i}(-a^-)}.
\end{align*}
%TODO nachschauen wie wir e(\cdot) definiert haben
Dann ist
\begin{align*}
e_p \colon \Q_p \to T
\end{align*}
ein Charakter. Die Stetigkeit folgt, daraus, dass $e_p$ auf $\Z_p$ konstant $1$, also lokal konstant ist.

\begin{prop}
Die Abbildung
\begin{align*}
\Q_p &\to \hat{\Q_p}\\
a &\mapsto (x \mapsto e_p(ax))
\end{align*}
ist ein Isomorphismus topologischer Gruppen.
\end{prop}
\begin{proof}
Wie so oft, ist die Surjektivität der schwierigste Teil.
Es ist klar, dass diese Abbildung ein Homomorphismus ist.
Für die Injektivität betrachten wir $a \in \Q_p$ mit $e_p(ax)=1$ für alle $x\in \Q_p$.
Dann gilt $ax \in \Z_p$ für alle $x\in \Q_p$. Dies impliziert aber $a=0$.
Für die Surjektivität sei $\chi\colon \Q_p \to T$ ein Charakter.
\begin{enumerate}[label=\roman*)]
\item Wir nehmen zunächst $\Z_p\subseteq \ker(\chi)$ an.
Für $m\in \Z$, $m>0$ gilt dann
$p^m \frac{1}{p^m}=1$ und somit $\chi(\frac{1}{p^m})^{p^m}=1$.
Also muss $\chi(\frac{1}{p^m})$ eine Einheitswurzel sein, im Besonderen gilt
\begin{align*}
\chi(\frac{1}{p^m})=\mathrm{e}(\frac{a_m}{p^m})
\end{align*}
für ein eindeutiges $a_m \in \Z/p^m\Z$.
Wir erhalten so eine Folge $(a_m)_{m> 0} \in \prod_{m>0} \left(\Z/p^m\Z\right)$ und behaupten, dass diese kompatibel ist.
Dies gilt, da
\begin{align*}
\mathrm{e}(\frac{a_m}{p^m})&=\chi(\frac{1}{p^m})=\chi(\frac{p}{p^{m+1}})\\
&=\chi(\frac{1}{p^{m+1}})^p=\mathrm{e}(\frac{a_{m+1}}{p^{m+1}})^p=\mathrm{e}(\frac{a_{m+1}}{p^{m}})^p,
\end{align*}
sodass $\frac{a_{m+1}}{p^m}=\frac{a_m}{p^m}\mod 1$.
Also definiert $(a_m)_{m>0}$ eine $p$-adische Zahl $a\in \Z_p$.
Es gilt $\chi(x)=e_p(-ax)$ für alle $x\in \Q_p$. Dies ist wahr für $x\in \Z_p$.
Als nächstes beweisen wir dies für $x=\frac{1}{p^m}$ mit $\Z \ni m>0$.
Dazu schreiben wir $a=\sum_{n=0}^\infty a_p(n)p^n$ modulo $p^m$ gilt
\begin{align*}
a&=\sum_{n=0}^{m-1}a_p(n)p^n \mod p^m\\
&=a_m \mod p^m.
\end{align*}
Daraus folgt
\begin{align*}
\frac{a_m}{p^m}=\frac{1}{p^m} \sum_{n=0}^{m-1} a_p(n)p^n \mod 1
\end{align*}
und
\begin{align*}
e_p(-\frac{a}{p^m})&=\mathrm{e}((\frac{a}{p^m})^-)\\
&=\mathrm{e}(\frac{1}{p^m}\sum_{n=0}^\infty a_p(n)p^n)\\
&=\mathrm{e}(\frac{a^m}{p^m})=\chi(\frac{1}{p^m}).
\end{align*}
Da $\Q_p$ als Gruppe von $\Z_p$ und den Elementen der Form $\frac{1}{p^m}$ mit $\Z \ni m>0$ erzeugt wird, folgt die Gleichheit der Charaktere.
\item Wir reduzieren den allgemeinen Fall auf Fall $1$.
Da $\Q_p$ total unzusammenhängend ist, enthält der Kern von $\chi$ einen Ball $K(0,p^{-m})=p^m \Z_p$ für $\Z \ni m>0$.
Wir definieren nun einen neuen Charakter, auf denen wir i) anwenden können.
Für $\psi(x)\coloneqq \chi(p^mx)$ gilt $\Z_p \subseteq \ker(\psi)$ und somit $\psi(x)=e_p(ax)$ für ein $a\in \Z_p$.
Damit folgt $\chi(x)=e_p(\frac{a_p}{p^m}x)$. \qedhere
\end{enumerate}
\end{proof}

Sei $a=(a_\nu)_{\nu \leq \infty}$. Dann gilt $a_p \in \Z_p$ für fast alle $p$.
Dies impliziert, dass
\begin{align*}
\mathrm{e}\colon \A &\to T\\
a &\mapsto \prod_{\nu \leq \infty} \mathrm{e}_\nu(a_\nu) 
\end{align*}
wohldefiniert ist. Außerdem ist $\mathrm{e}$ ein Charakter.

\begin{prop}
Die Abbildung
\begin{align*}
A &\to \hat{A}\\
a&\mapsto (x \mapsto \mathrm{e}(ax)
\end{align*}
ist ein Isomorphismus topologischer Gruppen.
\end{prop}
\begin{proof}
Es ist klar, dass die Abbildung ein Homomorphismus ist.
Um die Surjektivität zu zeigen, betrachten wir einen beliebigen Charakter $\chi \colon \A \to T$.
Dies induziert einen Charakter $\chi_\nu$ auf $\Q_\nu$ via
\begin{align*}
\Q_\nu \hookrightarrow \A \overset{\chi}{\to} T.
\end{align*}
Wir wissen, dass dann ein eindeutiges $a_\nu \in \Q_\nu$ mit $\chi_\nu(x_\nu)=\mathrm{e}_\nu=(a_\nu x_\nu)$ für alle $x_\nu \in \Q_\nu$ existiert.
Es gilt $a=(a_\nu)_{\nu \leq \infty}\in \A$, das heißt $a_p \in \Z_p$ für fast alle $p<\infty$.
Die natürliche Abbildung $\chi_f \colon \A_f \to T$ ist ein Charakter von $\A_f$.
Da $\A_f$ total unzusammenhängend ist, existiert $\Z \ni m >0$ mit
$m \hat{Z} \subseteq \ker(\chi_f)=\prod_{p<\infty} p^{\nu_p} \Z_p$ (hier ist $\hat{\Z}$ \emph{nicht} die duale Gruppe) mit $m=\prod p^{\nu_p}$.
Also gilt für alle $p \not \mid m$, dass $\Z_p \subseteq \ker(\chi_p)$ und somit $a_p \in \Z_p$.
Da ein beliebiges Element $x\in \A$ die Summe von endlich vielen Elementen in $\Q_\nu$ und etwas $m\hat{Z}$ ist, erhalten wir, dass $\chi(x)=\mathrm{e}(ax)$ für alle $x\in \A$ gilt.
\end{proof}

Als nächstes bestimmen wir die Charaktere der kompakten Gruppe $\A/\Q$.
Sei $r\in \Q$. Dann gilt
\begin{align*}
r=\sum_{n=n_p}^\infty r_p(n)n^p
\end{align*}
mit $n_p=0$ für fast alle $p<\infty$ und
\begin{align*}
r-\sum_{p<\infty} \sum_{n=n_p}^{-1}r_p(n) \in \Q \cap \Z_p
\end{align*}
for all $p$. Also folgt $r-\sum_{p<\infty} \sum_{n=n_p}^{-1}r_p(n) \in \Z$.
Daraus folgt dass für alle $a\in \Q$ die Abbildung
\begin{align*}
\A &\to T\\
x &\mapsto \mathrm{e}(ax)
\end{align*}
trivial auf $\Q$ ist, das heißt ein Charakter von $\A / \Q$.

\begin{thm}
Die Abbildung
\begin{align*}
\Q &\to \hat{\A/\Q}\\
a &\mapsto (x \mapsto \mathrm{e}(ax))
\end{align*}
ist ein Isomorphismus topologischer Gruppen.
\end{thm}
\begin{bem}
Wenn man dieser Isomorphie dualisiert, bekommt man eine Motiviation für die Einführung der Adele, nämlich um die Charaktere von $\Q$ beschreiben zu können.
\end{bem}
