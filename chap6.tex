\chapter{Charaktere}
\begin{defi}
Sei $G$ eine topologische Gruppe.
Ein \emph{Quasicharakter} von $G$ ist ein stetiger Gruppenhomomorphismus $G \to \C^\ast$.
Ein \emph{Charakter} von $G$ ist ein stetiger Gruppenhomomorphismus $G=T\coloneqq \{z \in \C \mid \abs{z}=1\}$.
\end{defi}

\begin{prop}
Sei $G$ eine kompakte Gruppe und $f \colon G \to \C^\ast$ ein Quasicharakter.
Dann ist $f$ ein Charakter.
\end{prop}
\begin{proof}
Die Abbildung
\begin{align*}
G &\to \R_{>0}^\ast\\
x&\mapsto \abs{f(x)}
\end{align*}
ist ein stetiger Gruppenhomomorphismus. Daher ist das Bild eine kompakte Untergruppe von $\R_{>0}^\ast$, nämlich $\{1\}$.
\end{proof}

\begin{defi}
Eine lokal kompakte Hausdorffgruppe heißt \emph{total umzusammenhängend}, falls jede offene Umgebung der $1$ eine offene Untergruppe enthält.
\end{defi}

\begin{bsp}
Zum Beispiel sind
\begin{enumerate}[label=\roman*)]
\item $(\Q_p,+)$, $(\Q_p^\ast,\cdot)$
\item $(\Z_p,+)$, $(\Z_p^\ast,\cdot)$
\item $(\A_f,+)$, $(\A_f^\ast,\cdot)$
\item $(\hat{Z},+)$, $(\hat{Z}^\ast,\cdot)$
\end{enumerate}
solche Gruppen.
\end{bsp}

\begin{prop}
Sei $G$ eine total unzusammenhängende lokal kompakte Hausdorffgruppe und $\chi\colon G \to \C^\ast$ ein Quasicharakter.
Dann ist $\ker(\chi)$ offen.
\end{prop}
\begin{proof}
Sei $0<\varepsilon<1$. Dann ist $V_\varepsilon\coloneqq \{z\in \C \mid \Ret(z)>\varepsilon\}$ eine offene Umgebung von $1\in \C^\ast$.
Also ist $\chi^{-1}(V_\varepsilon)$ offen in $G$ und enthält somit eine offene Untergruppe $H \subseteq G$.
Das Bild $\chi(H)$ ist eine Untergruppe von $\C^\ast$ in $V_\varepsilon$.
Für $z\in \C$ ungleich $1$ existiert ein $n\in \N$ sodass die Potenz $z^n$ außerhalb von $V_\varepsilon$ liegt.
Also folgt $\chi(H)=1$.
Sei nun $x\in \ker(\chi)$. Dann ist $xH$ eine offene Umgebung von $x$ in $\ker(\chi)$.
\end{proof}

\begin{defi}
Sei $G$ eine topologische Gruppe. Dann ist die Menge der Charaktere auf $G$ eine Gruppe bezüglich punktweiser Multiplikation.
Dieser Gruppe heißt die \emph{duale Gruppe von $G$}, wir bezeichnen sie mit $\hat{G}$.
Wir statten $\hat{G}$ mit kompakt-offenen Topologie aus.
Eine Subbasis dieser Topologie bilden die Mengen der Form
\begin{align*}
S(K,U)\coloneqq \{\chi \in C(G,T)\mid \chi(K)\subseteq U\}
\end{align*}
wobei $K\subseteq G$ kompakt und $U\subseteq T$ offen ist.
Endliche Schnitte dieser Mengen bilden eine Basis der Topologie von $\hat{G}$.
\end{defi}

\begin{prop}
Die duale Gruppe $\hat{G}$ ist eine abelsche Hausdorffgruppe.
\end{prop}
\begin{proof}
Sei $f \in \hat{G}, \varepsilon\not =f$.
Dann existiert ein $x\in G$ mit
\begin{align*}
f(x)\not =1 \varepsilon(x).
\end{align*}
Da $T$ Hausdorff ist, können wir offene Mengen $U,W \subseteq T$ mit $f(x)\in U, 1\in W$ sowie $U\cap W=\emptyset$ wählen.
Definiere $K=\{x\}$.
Dann gilt $f\in S(K,U)$, $\varepsilon\in S(K,W)$ sowie $S(K,U)\cap S(K,W)=\emptyset$.
\end{proof}

\begin{prop}
Sei $G$ eine topologische Gruppe.
\begin{enumerate}[label=\roman*)]
\item Falls $G$ endlich ist, so ist $\hat{G}$ ebenfalls endlich,
\item Falls $G$ diskret ist, so ist $\hat{G}$ kompakt.
\end{enumerate}
\end{prop}

\begin{prop}
Sei $G$ eine Hausdorffgruppe.
\begin{enumerate}[label=\roman*)]
\item Falls $G$ lokal kompakt ist, so ist $\hat{G}$ ebenfalls lokal kompakt.
\item Falls $G$ kompakt ist, so ist $\hat{G}$ diskret.
\end{enumerate}
\end{prop}

\begin{bsp}
Wir betrachten einige Beispiele von Charakteren.
\begin{enumerate}[label=\roman*)]
\item Die Abbildung
\begin{align*}
\mathrm{e}_\infty \colon \R &\to T\\
x &\mapsto \mathrm{e}^{2\pi \mathrm{i}x}\eqqcolon \mathrm{e}(x)
\end{align*}
ist ein Charakter von $\R$.
\end{enumerate}
\end{bsp}

\begin{prop}
Es gilt $\chi(x)=\mathrm{e}^{2\pi \mathrm{i}a x}$.
Die Abbildung
\begin{align*}
\R &\to \hat{\R}\\
a&\mapsto (x\mapsto \mathrm{e}_\infty(ax))
\end{align*}
ist ein Isomorphismus topologischer Gruppen.
\end{prop}