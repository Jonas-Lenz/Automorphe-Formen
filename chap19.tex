\chapter{Das Zentrum von $U(\GL_2(\C))$}
\begin{prop}
Sei $F\colon \GL_2(\R) \to \C$ glatt, linksinvariant unter $\GL_2(\Z)$ und rechtsinvariant $\Oo_2(\R) \cdot \R^\ast$.
Sei $D \in Z(U(g))$. Dann ist $DF$ linksinvariant unter $\GL_2(\Z)$ und rechtsinvariant unter $\Oo_2(\R) \cdot \R^\ast$.

\end{prop}
Das Zentrum $Z(U(g))$ wird durch das Harish-Chandra Theorem beschrieben.
$Z(U(g))$ ist ein Polynomring in $2$ Variablen.
Die Erzeuger können folgendermaßen gewählt werden.
Es gilt $Z(U(g))=\C[D_1,D_2]$ mit $D_1=D_z$ mit $z=\begin{pmatrix}
1 &0\\
0 &1
\end{pmatrix}$ und $D_2=D_c$ mit $c=x_{11}^2 +x_{12}x_{21} +x_{21}x_{12} +x_{22}^2$.

