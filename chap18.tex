\chapter{Die universell einhüllende Algebra von $\GL_2(\R)$}
Eine Lie-Algebra ist ein Vektorraum $g$ mit einer bilinearen Abbildung
\begin{align*}
[\cdot,\cdot] \colon g \times g &\to g
\end{align*}
die
\begin{enumerate}
\item $[x,x]=0$,
\item $[x,[y,z]]+[y,[z,x]]+[z[x,y]]$ (Jacobiidentität)
\end{enumerate}
für alle $x,y,z \in g$ erfüllt.

\begin{bsp}
Die $n\times n$ Matrizen über $\C$ bilden eine Lie-Algebra bezüglich $[A,B]=AB-BA$.
\end{bsp}

\begin{defi}
Sei $g$ eine Lie-Algebra. Eine \emph{universell einhüllende Algebra} von $g$ ist ein Paar $(U,i)$ wobei $U$ eine assoziative Algebra und
$i$ eine Abbildung $i\colon g \to U$ mit
$i([x,y])=[i(x),i(y)]$ ist, die die folgende universelle Eigenschaft erfüllt

%\begin{tikzcd}
%g \arrow[r, "j"] \arrow[rd, "i"] &A\\
%&U \arrow[u, "\Phi"]
%\end{tikzcd}

%TODO hier Diagramm einfügen
\end{defi}

\begin{prop}
Sei $g$ eine Lie-Algebra.
Dann besitzt $g$ eine universell einhüllende Algebra $U(g)$ und diese ist eindeutig bis auf eindeutigen Isomorphismus.
\end{prop}
\begin{proof}
%Übung, Tr(g)=k \oplus g \oplus g \dots.
%Teile iwas raus, tensorprodukt.
Die Eindeutigkeit ist aus dem Diagramm klar.
Die Existenz verbleibt als Übung.
\end{proof}

Wir zitieren folgendes Resultat welches 1900 von Henri Poincaré und 1937 von Birkhoff und Witt bewiesen wurde.
\begin{thm}
Sei $g$ eine Lie-Algebra mit universell einhüllender Algebra $(U,i)$.
Dann ist die Abbildung $i\colon g \to U$ injektiv.
\end{thm}

Wir konstruieren nun einen Isomorphismus von $U(\GL_2(\R))$ in eine Algebra bestehend aus Differentialoperatoren.

\begin{defi}
Sei $\alpha \in \GL_2(\R)$ und $F \colon \GL_2(\R) \to \C$ glatt. Definiere für $g \in \GL_2(\R)$
\begin{align*}
(D_\alpha F)(g)&\coloneqq \frac{\partial}{\partial t} F(g\mathrm{e}^{t\alpha}) \mid_{t=0}\\
&=\frac{\partial}{\partial t} F(g+tg\alpha)\mid_{t=0}.
\end{align*}
\end{defi}

\begin{prop}
Für $c\in \R$ und $F, G \colon \GL_2(\R) \to \C$ glatt gilt
\begin{enumerate}[label=(\roman*)]
\item $D_\alpha(F+G)=D_\alpha F+ D_\alpha G$,
\item $D_\alpha(cF)=c(D_\alpha F)$,
\item $D_\alpha(FG)=(D_\alpha F)G+F(D_\alpha G)$.
%hier sollte in t=0 wichtig sein.
\end{enumerate}
\end{prop}

\begin{defi}
Die Differentialoperatoren $D_\alpha$ erzeugen eine assoziative Algebra
über $\R$, die wir mit $D_\R^2$ bezeichnen.
Das Produkt ist durch Komposition gegeben.
\end{defi}

Wir sammeln nun einige Eigenschaften dieser Algebra.

\begin{prop}
Für $c \in \R$ und $\alpha,\beta \in \GL_2(\R)$ gilt
\begin{enumerate}[label=(\roman*)]
\item $D_{c\alpha} =c D_\alpha$,
\item $D_{\alpha+\beta}=D_\alpha +D_\beta$,
\item $D_\alpha D_\beta -D_\beta D_\alpha=D_{\alpha\beta-\beta\alpha}$.
\end{enumerate}
\end{prop}
\begin{proof}
Sei $F \colon \GL_2(\R) \to \C$ glatt. Schreibe $g \in \GL_2(\R)$ als $g=(g_{ij})$.
Sei $h=(h_{ij})\in \GL_2(\R)$ und $\alpha=(\alpha_{ij})\in \GL_2(\R)$.
Dann gilt mit der Kettenregel
\begin{align*}
(D_\alpha F)(h)&= \frac{\partial}{\partial t} F(h+th\alpha)\mid_{t=0}\\
&=\sum_{i,j} \frac{\partial F}{\partial g_{ij}} \mid_h \frac{\partial}{\partial t} (h_{ij}+t(h\alpha)_{ij})\mid_{t=0}\\
&=\sum_{i,j} \frac{\partial F}{\partial g_{ij}} \mid_h (h\alpha)_{ij}.
\end{align*}
Daraus folgen die beiden ersten Eigenschaften.
Anwenden von $D_\beta$ impliziert unter der Verwendung der Produktregel
\begin{align*}
D_\beta(D_\alpha F)(h)&=\frac{\partial }{\partial t} (D_\alpha F)(h+th\beta)\mid_{t=0}\\
&\frac{\partial}{\partial t} \left( \sum_{i,j} \frac{\partial F}{\partial g_{ij}} \mid_{h+th\beta} (h\alpha +th\beta\alpha)_{ij}\right) \mid_{t=0}\\
&=\sum_{i,j} \left(\frac{\partial F}{\partial g_{ij}} \mid_h (h\beta \alpha)_{ij} +\sum_{k,l} \frac{\partial^2 F}{\partial g_{kl} \partial g_{ij}}\mid_h (h\beta)_{kl} (h\alpha)_{ij} \right).
\end{align*}
Dies impliziert nun
\begin{align*}
(D_\beta D_\alpha-D_\alpha D_\beta)(F(h))&=\sum_{i,j} \frac{\partial F}{\partial g_{ij}}\mid_h \left( (h\beta \alpha)_{ij}-(h\alpha\beta)_{ij} \right)\\
&=D_{[\beta,\alpha]} F(h),
\end{align*}
was die dritte Behauptung zeigt.
\end{proof}

Also erhalten wir mit $V=\CC^\infty(\GL_2(\R))$ eine lineare Abbildung
\begin{align*}
j \colon \GL_2(\R) &\to \End(V)\\
a &\mapsto D_\alpha.
\end{align*}
Aufgrund der universellen Eigenschaft der universell einhüllende
Algebra folgt aus $[D_\alpha,D_\beta]=D[\alpha,\beta]$, dass es einen
eindeutigen Homomorphismus
\begin{align*}
\Phi \colon U(\GL_2(\R)) &\to \End(V)
\end{align*}
mit $\Phi(i(\alpha))=D_\alpha$ gibt.

\begin{thm}
Die Abbildung $\Phi$ ist injektiv und liefert einen Isomorphismus von $U(\GL_2(\R))$ und $D_\R^2$.
\end{thm}

Dieses Resultat lässt sich folgendermaßen verallgemeinern:
Sei $G$ eine Lie-Gruppe mit Lie-Algebra $g$.
Dann kann $U(g)$ mit der Algebra der linksinvarianten Differentialoperatoren identifiziert werden.

Wir erweitern die Operation von $\GL_2(\R)$ auf den glatten Funktionen
$F \colon \GL_2(\R) \to \C$ zu einer Operation von $\GL_2(\C)=\GL_2(\R) \otimes_\R \C$ durch $(D_{\mathrm{i}\alpha} F)(g)\coloneqq \mathrm{i} (D_\alpha F)(g)$
für $\alpha \in \GL_2(\R)$ und identifizieren $U(\GL_2(\C))$ mit $D_\C^2 \coloneqq D_\R^2 \otimes_\R \C$.
Im folgenden schreiben wir für $\GL_2(\C)$ nur $g$.

\begin{prop}
Sei $F \colon \GL_2(\R) \to \C$ glatt und linksinvariant unter $\GL_2(\Z)$
und rechtsinvariant unter dem Zentrum $Z(\GL_2(R))=\R^\ast$.
Dann ist $DF$ für alle $D \in D_\C^2$ linksinvariant unter $\GL_2(\Z)$ sowie rechtsinvariant unter $Z(\GL_2(\R))$.
\end{prop}
\begin{proof}
Seien $\alpha_1,\dots,\alpha_n \in \GL_2(\R)$ und sei $D=D_{\alpha_1} \circ \dots \circ D_{\alpha_n}$.
Dann gilt für $\gamma \in \GL_2(\Z)$
\begin{align*}
DF(\gamma h) &=\frac{\partial}{\partial t_1}\dots \frac{\partial}{\partial t_n}(F(\gamma h \ex^{t_n\alpha_n} \dots \ex^{t_1\alpha_1}))\mid_{t=0}\\
&=\frac{\partial}{\partial t_1}\dots \frac{\partial}{\partial t_n}(F( h \ex^{t_n\alpha_n} \dots \ex^{t_1\alpha_1}))\mid_{t=0}\\
&=(DF)(h).
\end{align*}
Die Invarianz unter $Z(\GL_2(\R))$ funktioniert analog.
\end{proof}
