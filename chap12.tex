\chapter{Die modulare Gruppe}

Die Gruppe $\GL_2(\C)$ operiert auf der Riemannschen Zahlenkugel
$\mathds{P}^1=\C \cup \{\infty\}$
via
\begin{align*}
\begin{pmatrix}
a&b\\
c&d
\end{pmatrix} z=\frac{az+b}{cz+d}.
\end{align*}
Diese Gruppenoperation ist transitiv.
Für $M=\begin{pmatrix}
a&b\\
c&d
\end{pmatrix} \in \GL_2(\R)$ und $z \in \C$ mit $cz+d\not =0$ gilt
\begin{align*}
\Imt(Mz)=\frac{\det(M)}{\abs{cz+d}^2}\Imt(z).
\end{align*}
$\mathds{P}^1(\C)$ wird durch die Operation von $\GL_2(\R)^+$, den Matrizen mit positiver Determinante, in drei Orbits zerlegt, nämlich
$H=\{z\in \C \mid \Imt(z)>0\}$, $-H$ und $\mathds{P}^1(\R)=\R\cup \{\infty\}$.

\begin{bsp}
\begin{enumerate}[label=(\roman*)]
\item $T=\begin{pmatrix}
1&1\\
0&1
\end{pmatrix}$ operiert durch Translation, es gilt $T^m=\begin{pmatrix}
1&m\\
0&1
\end{pmatrix}$
und $T^m=z+m$.
\item $S=\begin{pmatrix}
0&-1\\
1&0
\end{pmatrix}$ operiert durch $Sz=\frac{-1}{z}$
\item Für $z=x+\mathrm{i}y\in H$ gilt
\begin{align*}
M_z=\begin{pmatrix}
y&x\\
0&1
\end{pmatrix} \in \GL_2(\R)^+
\end{align*}
und $M_z \mathrm{i}=z$.
\item Es gilt die folgende Formel $(ST)^3=(TS)^3=S^2=-1$.
\end{enumerate}
\end{bsp}

Wir charakterisieren nun Erzeuger von $\SL_2(\R)$.
\begin{prop}
Die Gruppe $\SL_2(\R)$ wird durch die Matrizen $\begin{pmatrix}
1&x\\
0&1
\end{pmatrix}$,
$\begin{pmatrix}
y&0\\
0&\frac{1}{y}
\end{pmatrix}$, $S$
erzeugt.
\end{prop}

\begin{prop}
Der Stabilisator von $\mathrm{i}$ in $\SL_2(\R)$ ist
\begin{align*}
\SO_2(\R)=\{M \in \SL_2(\R) \mid M M^T=1\}.
\end{align*}
Insbesondere ist die Abbildung
\begin{align*}
\SL_2(\R)/\SO_2(\R) &\to H\\
M\SO_2(\R) &\mapsto M_i
\end{align*}
eine Bijektion.
\end{prop}


\begin{prop}
Die Gruppe $\Gamma \coloneqq \SL_2(\Z)$ wird von $S$ und $T$ erzeugt wird.
\end{prop}
\begin{proof}
Sei $G=\langle S,T \rangle \subseteq \Gamma$. Dann gilt $-1=S^2\in G$.
Sei $M=\begin{pmatrix}
a&b\\
c&d
\end{pmatrix}\in \Gamma$.
Wir zeigen per Induktion über $\abs{c}$, dass $M\in G$.
Falls $c=0$, dann gilt $M= \pm \begin{pmatrix}
1&n\\
0&1
\end{pmatrix}=\pm T^n$
für geeignetes $n\in \Z$.
Also ist $M \in G$. Sei nun $c \not =0$. Dann gilt
\begin{align*}
ST^m M&=\begin{pmatrix}
0&-1\\
1&0
\end{pmatrix}
\begin{pmatrix}
1&m\\
0&1
\end{pmatrix}
\begin{pmatrix}
a&b\\
c&d
\end{pmatrix}\\
&=\begin{pmatrix}
0&-1\\
1&m
\end{pmatrix}
\begin{pmatrix}
a&b\\
c&d
\end{pmatrix}\\
&=\begin{pmatrix}
\ast&\ast\\
a+mc&\ast
\end{pmatrix}.
\end{align*}
Für geeignetes $m$ gilt
\begin{align*}
0 \leq a+mc <\abs{c}.
\end{align*}
Nach Induktionsvoraussetzung ist dann aber $ST^m M \in G$, sodass $M \in G$ folgt.
\end{proof}

Definiere $D=\{z \in \C \mid \abs{\Ret(z)}<\frac{1}{2},\abs{z}>1\}$.
$D$ ist ein Fundamentalbereich für Operation von $\Gamma$ auf $H$.
%TODO Definition dafür einfügen

\begin{thm}
\begin{enumerate}[label=(\arabic*)]
\item Für jedes $z\in H$ existiert ein $\gamma \in \Gamma$ mit $\gamma y\in \overline{D}$.
\item Seien $z \not =w\in \overline{D}$ konjugiert bezüglich $\Gamma$. Dann liegen $z,w$ auf dem Rand von $D$ und $\abs{\Ret(z)}=\frac{1}{2}$ und $z=w \pm 1$ oder $\abs{z}=1$ und $z=-\frac{1}{w}$.
\item Sei $z\in \overline{D}$ und $\Gamma_z=\{\gamma \in \Gamma \mid \gamma z=z\}$ der Stabilisator von $z$.
Dann gilt mit $\rho=\mathrm{e}^{2/3\pi \mathrm{i}}$
\begin{align*}
\Gamma_i&=\langle S \rangle\\
\Gamma_\rho&=\langle ST \rangle\\
\Gamma_{-\overline{\rho}}&=\langle TS \rangle\\
\Gamma_z&=\langle -1 \rangle
\end{align*}
für alle $z\in \overline{D} \setminus  \{\mathrm{i},\rho,-\overline{\rho}\}$.
\end{enumerate}
\end{thm}

\begin{proof}
Wir beweisen lediglich 1).
Sei $z\in H$. Dann ist $\{mz+n \mid (m,n)\in \Z^2\}$ ein Gitter in $\C$.
Jeder Ball um $0$ enthält lediglich endlich viele Gitterpunkte.
Also existiert ein Paar $(c,d)\in \Z^2\setminus \{(0,0)\}$ mit 
\begin{align*}
\abs{cz+d}\leq \abs{mz+n}
\end{align*}
für alle $(m,n)\in \Z^2\setminus \{(0,0)\}$.
Das heißt, es gibt einen Gitterpunkt, der am nächsten an der Null ist.
Aufgrund der Minimalität gilt $(c,d)=1$ und es gibt eine Matrix
$\begin{pmatrix}
a&b\\
c&d
\end{pmatrix} \in \SL_2(\Z)$.
Dann gilt
\begin{align*}
T^m M &= \begin{pmatrix}
1&m\\
0&1
\end{pmatrix}\begin{pmatrix}
a&b\\
c&d
\end{pmatrix}\\
&=\begin{pmatrix}
a+mc&b+md\\
c&d
\end{pmatrix}
\end{align*}
und mit $z'\coloneq T^m Mz$
\begin{align*}
\abs{z'}=\abs{\frac{(a+mc)z+(b+md)}{cz+d}} \geq 1.
\end{align*}
Wenn man $m$ geeignet wählt, folgt $abs{\Ret(z')}\leq \frac{1}{2}$, also $z'\in \overline{D}$.
\end{proof}

